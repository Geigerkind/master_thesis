\chapter{Einleitung}
Als \textit{Lokalisierung} wird der Prozess bezeichnet die Position von einem Gerät oder Nutzer in einem Koordinatensystem zu bestimmen \cite{bulusu2000gps}.
Diese Information wird von technischen Systemen genutzt, um deren Dienste anzubieten, z. B. Tracking- oder Navigationssysteme.
Ein bekanntes Beispiel ist das \textit{Global Positioning System} (GPS) \cite{kaplan2005understanding}.
Bei GPS berechnet das empfangende Gerät seine Position basierend auf den empfangenden Signalen der Satelliten.
\newline
\newline
In Gebäuden ist die Signalstärke der GPS-Satelliten jedoch stark eingeschränkt, sodass die Ortung ungenau wird oder überhaupt nicht funktioniert.
Aus diesem Grund werden für \textit{Indoor}-Lokalisation andere Ansätze verfolgt.
Je nach Genauigkeit können Objekte mit Sendern, RFID Tags oder Barcodes markiert werden \cite{xiao2016survey}.
Meistens wird eine komplexe Infrastruktur benötigt, die Lokalisierungssysteme vergleichsweise teuer macht.
\newline
\newline
Eine weitere Art der Lokalisation ist der Orientierungssinn von Menschen und Tieren.
Anhand von Orientierungspunkten wird von einem Punkt zu einem anderen Punkt navigiert.
Beispielsweise Honigbienen navigieren auf Basis von gelernten Orientierungspunkten, um Nahrungsquelle und Nest zu finden \cite{menzel1996knowledge}.
\newline
\newline
In dieser Arbeit wird die diskrete Standortbestimmung basierend auf Sensordaten untersucht,
d.~h. eine Art der Lokalisierung, in der keine Koordinaten, sondern diskrete Standorte gefunden werden sollen.
Dabei soll eine bestimmte Anzahl von Standorten anhand verschiedener Sensorwerte unterschieden werden.
Dies ähnelt dem zuvor beschriebenen Orientierungssinn.
Mit Hilfe maschinellen Lernens sollen künstliche neuronale Netze (KNNs) und Entscheidungsbäume
auf Basis von ausgewählten Features, d.~h. Attribute und Eigenschaften der Daten, trainiert werden.
Diese Arbeit baut auf den Ergebnissen von Mian auf, der die diskrete Standortbestimmung mit Hilfe von
Feed Forward neuronalen Netzwerken (FFNNs) untersucht hatte \cite{naveedThesis}.
Mian nutzte drei Sensoren und große FFNNs, um 14 Standorte bestimmen zu können.
Zusätzliche Sensoren können das Problem potenziell vereinfachen, wodurch kleinere Modelle verwendet werden können
und mehr Standorte unterschieden werden können.
\newline
\newline
Zur Erstellung dieser Arbeit wurde ein Evaluierungsprozess der Sensordaten erstellt, mit dem Standorte und Anomalien mit Hilfe von ML-Modellen unterschieden werden.
Dafür wurden Sensordaten mit CoppeliaSim simuliert, da Echtdaten eines Prototyps zum Zeitpunkt dieser Arbeit noch nicht verfügbar sind.
Die simulativ erfassten Sensordaten wurden mit weiteren Sensordaten auf Basis vereinfachter Modelle ergänzt.
Es wurde die Klassifizierungsgenauigkeit der ML-Modelle untersucht im Hinblick auf Skalierbarkeit der Anzahl der Standorte sowie Robustheit gegenüber mögliche Fehler.
Außerdem wurde die Klassifizierungsgenauigkeit des Anomalieerkennungsmodells untersucht.
Auch wurde der Speicher- und Energiebedarf für den Betrieb auf einem batteriegestützten Mikrocontroller geschätzt.
Insgesamt wurden 128 ML-Modelle zur Standortbestimmung von 9 bis 102 Standorten und zwei ML-Modelle zur Anomalieerkennung evaluiert.
\newline
\newline
Kapitel 2 führt Entscheidungsbäume und Ensemble-Methoden ein.
Kapitel 3 führt künstliche neuronale Netze (KNN) mit dem Fokus auf FFNNs ein.
In Kapitel 4 wird auf den Stand der Forschung für die Standortbestimmung mit Hilfe von maschinellen Lernen (ML) eingegangen.
Das Trainieren der ML-Modelle mit Entscheidungsbäumen und FFNNs wird in Kapitel 5 erläutert.
Kapitel 6 geht auf die Generierung von Trainings- und Testdaten auf Basis von Simulationen ein.
Darauf folgt die Evaluation der Klassifizierungsgenauigkeit, Fehlertoleranz und Ressourcennutzung in Kapitel 7.
Kapitel 8 enthält einen kritischen Rückblick auf die Entscheidungen dieser Arbeit bevor Kapitel 9 Schlussfolgerungen zieht.