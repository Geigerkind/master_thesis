\chapter{Einleitung}
Als Standortbestimmung, oder \textit{Lokalisierung}, wird der Prozess bezeichnet die Position von einem Gerät oder Nutzer in einem Koordinatensystem zu bestimmen \cite{bulusu2000gps}.
Diese Information wird von technischen Systemen genutzt, um deren Dienste anzubieten, z. B. Tracking- oder Navigationssysteme.
Ein bekanntes Beispiel ist das \textit{Global Positioning System} (GPS) \cite{kaplan2005understanding}.
Bei GPS berechnet das empfangende Gerät seine Position basierend auf den empfangenden Signalen der Satelliten.
\newline
\newline
GPS bedarf aber direkten Kontakt zu den Satelliten, ansonsten ist die Signalstärke zu den Satelliten extrem eingeschränkt.
Aus diesem Grund werden für \textit{Indoor}-Lokalisation andere Ansätze verfolgt.
Je nach Genauigkeit können Objekte mit Sendern, RFID Tags oder Barcodes markiert werden \cite{xiao2016survey}.
Meistens wird eine komplexe Infrastruktur benötigt, die diese Lokalisierungssysteme vergleichsweise teuer macht.
\newline
\newline
In dieser Arbeit wird die diskrete Positionsbestimmung basierend auf Sensordaten untersucht.
Dabei soll eine bestimmte Anzahl an Orten anhand verschiedener Sensorwerten unterschieden werden.
Dies ist vergleichbar mit dem Orientierungssinn von Tieren und Menschen.
Zum Beispiel navigieren Honigbienen auf Basis von gelernten Orientierungspunkten, um Nahrungsquelle und Nest zu finden \cite{menzel1996knowledge}.
Mit Hilfe maschinellen Lernens sollen künstliche neuronale Netze (KNN) und Entscheidungsbäume trainiert werden.
Als Eingabedaten werden aus den gesammelten Sensordaten Features extrahiert, d. h. Attribute und Eigenschaften dieser Daten.
Dieses System bedarf keine Infrastruktur muss aber für jedes Gebäude individuell trainiert werden.
\newline
\newline
TODO: Was wurde in dieser Arbeit gemacht.
\newline
TODO: Was unterscheided diese Arbeit von anderen Arbeiten?
\newline
\newline
Kapitel 2 führt Entscheidungsbäume und Ensemble-Methoden ein.
Kapitel 3 führt künstliche neuronale Netze (KNN) ein mit dem Fokus auf Feed Forward neuronale Netze (FFNN).
In Kapitel 4 wird auf den Stand der Forschung für Standortbestimmung mit Hilfe von maschinellen Lernen (ML) eingegangen.
Die Generierung von Trainings- und Validationsdaten auf Basis von Simulationen wird in Kapitel 5 erläutert.
Kapitel 6 stellt die trainierten ML Modelle mit Entscheidungsbäumen und FFNN vor.
Darauf folgt die Evaluation der Klassifizierungsgenauigkeit, Fehlertoleranz und Resourcennutzung in Kapitel 7.
Kapitel 8 enthält einen kritischen Rückblick auf die Entscheidungen dieser Arbeit, bevor Kapitel 9 Schlussfolgerungen zieht.