\chapter{Einleitung}
Lokalisation is der Prozess die Position von einem Objekt zu bestimmen. Die Positionsbestimmung ist integral für viele technische System, z. B. Tracking, Navigation oder Überwachung.
Ein sehr bekanntes und vielseitig genutztes Positionsbestimmungssystem ist das Global Positioning System (GPS). GPS trianguliert die Position des anfragenden Geräts mit Hilfe von mehreren
Satelliten (TODO Quelle). Im freien ist eine Genauigkeit von X(TODO) m möglich. Für die Positionsbestimmung innerhalb von Gebäuden ist die Genauigkeit aber nicht ausreichend,
außerdem wird sie erschwert durch die dicke Wände und Interferenzen. Dadurch ist GPS oft nicht ausreichend für Trackingsysteme innerhalb von Gebäuden, z. B. Lagerhallen. Aus diesem Grund
werden andere Systeme für diesen Zweck verwendet. Je nach Bedarf der Genauigkeit können Objekte mit Sendern, RFID Tags oder Barcodes markiert werden (TODO Quellen). Diese Ansätze bedürfen
eine Infrastruktur, die in den Gebäuden installiert und gewartet werden muss.
\newline
\newline
In dieser Arbeit wird die diskrete Positionsbestimmung basierend auf Sensordaten untersucht. Dabei soll eine bestimmte Anzahl an Orten anhand von verschiedenen Sensorwerten unterschieden werden.
Dies ist Vergleichbar mit dem Orientierungssinn von Tieren und Menschen. Zum Beispiel navigieren Honigbienen auf Basis von gelernten Orientierungspunkten, um Nahrungsquelle und Nest zu finden (TODO Quelle).
Mit Hilfe maschinellen Lernens sollen künstliche neuronale Netze (KNN) und Entscheidungsbäume trainiert werden. Als Eingabedaten werden aus den gesammelten Sensordaten Features extrahiert,
d. h. Attribute und Eigenschaften dieser Daten. Dieses System bedarf keine Infrastruktur muss aber für jedes Gebäude individuell trainiert werden.
\newline
\newline
TODO: Was wurde in dieser Arbeit gemacht.
\newline
\newline
TODO: Kapitelübersicht

\iffalse
// Teil 1
* Leite Lokalisation ein
* Erkläre die Motivation dafür
    * Tracking
    *
* Gehe auf bestehende Systeme ein, e.g. GPS, RFID, Indoor Triangulierung
* Gehe auf deren Schwächen ein

// Teil 2
* Erkläre den untersuchten Ansatz
    * Blinden Beispiel?
    * Tier Beispiel: Honigbienen
* Gehe auf die Vorteile gegenüber den anderen Ansätze ein

// Teil 3
* Sag was in dieser Arbeit gemacht wurde

// Teil 4
* Gib eine Kapitelübersicht
\fi