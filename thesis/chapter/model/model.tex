\chapter{ML-Modelle}
In dieser Arbeit wird die Device-Based Indoor-Lokalisation auf basis von Sensorwerten untersucht.
Das ist inspiriert von dem Orientierungssinn von Mensch und Tier.
Dabei werden diskrete Standorte unterschieden, sowie ob eine Anomalie entdeckt wurde,
d. h. ob das Modell sich an einem unbekannten Standort oder auf einem unbekannten Pfad befindet.
\newline
\newline
Abbildung \ref{fig:model_idea} zeigt die Architektur des verfolgten Ansatzes.
Zunächst werden aus den Sensorwerten Features extrahiert.
Die resultierende Feature-Menge wird dann von dem ML-Modell genutzt, um den Standort zu klassifizieren.
Zuletzt wird auf basis historischer Daten und dem Klassifizierungsergebnis in einem Analyseschritt (TODO: ML Ansatz?) bestimmt, ob eine Anomalie vorliegt.
Im Vergleich zu der Architektur von Mian \cite{naveedThesis} können die Klassifizierungsergebnisse bei der Feature-Extrahierung weiter verarbeitet werden.
\begin{figure}[h!]
    \centering
    \includegraphics[width=\linewidth]{images/model_idea.png}
    \caption{Architektur des verfolgten Ansatzes.}
    \label{fig:model_idea}
\end{figure}
\newline
In dieser Arbeit Entscheidungsbaum basierte Klassifizierer mit KNN verglichen, insbesondere den von Mian verwendeten Ansatz mit FFNN.
Entscheidungsbäume sind deutlich effizienter als KNN, weswegen diese bei vergleichbarer Klassifizierungsgenauigkeit und Fehlertoleranz die preferierte Wahl sind.

\iffalse
Eigentlich eignen sich \textit{rekurrente neuronale Netze} (RNN) gut für diese Aufgabe, da serielle Daten verarbeitet werden
und zeitliche Abhängigkeiten für die Klassifizierung wichtig sind (TODO: Quelle).
Aber in dieser Arbeit wurde sich dagegen entschieden, da diese sehr rechenaufwändig sind und viel Speicher benötigen im Vergleich zu FFNN oder Entscheidungswälder,
weshalb diese suboptimal für kleine Mikrocontroller sind (TODO: Quelle).
\fi

\section{Standortkodierung}
\label{sec:model_location_encoding}
Als Standort wird ein einzigartiger diskreter Ort im Indoor-Szenario bezeichnet.
Bei der Klassifizierung können Standorte auf verschiedene Arten im ML-Modell kodiert werden.
Mian kodierte die Pfade zwischen Punkten, die von Interesse sind, als Standorte,
d. h. das Klassifizierungsergebnis ist der Pfad auf dem sich der Mikrocontroller befindet \cite{naveedThesis}.
\newline
\newline
In dieser Arbeit wird neben der Erkennung von Pfaden auch die Erkennung von Knoten, sowie die Erkennung von Pfaden und Knoten untersucht.
Pfade und Knoten liegen auf Routen, die als zyklischen Graphen betrachtet werden können, wie in Abbildung \ref{fig:location_encoding} illustriert.
Der Ansatz von Mian versucht die Kanten dieses Graphen zu klassifizieren und definiert diese als Standorte.
Daneben können auch nur die Knoten als Standorte kodiert werden und alle restlichen Datensätze als \textit{unbekannten Standort}.
Zuletzt können sowohl die Knoten und Kanten als Standorte definiert werden.
Die Knoten werden durch alle Datensätze dargestellt, die innerhalb eines Umkreises von einem Punkt sind, der von Interesse ist.
Folglich befinden sich alle restlichen Datensätze auf Kanten zwischen zwei Knoten oder gelten als \textit{unbekannt}.
\begin{figure}[h!]
    \centering
    \includegraphics[width=\linewidth]{images/location_encoding.png}
    \caption{Standortkodierung der Knoten und Pfade.}
    \label{fig:location_encoding}
\end{figure}
\newline
Daraus wird die Komplexität und Genauigkeit dieser Ansätze deutlich.
Der Kantenansatz ist ein Kompromiss zwischen Genauigkeit und Komplexität.
Dabei bestimmt die Anzahl der zu klassifizierenden Standorte die Komplexität.
Zum einen ist gerade bei langen Pfaden eine geringe Auflösung im Vergleich zur Realposition des Objektes zu erwarten,
d. h. es ist unklar, ob sich das Objekt am Anfang, Ende oder in dazwischen befindet.
Zum anderen werden in einem zyklischen Graphen mindestens so viele Standorte, wie beim Knotenansatz verwendet.
Der Knotenansatz benötigt am wenigsten Standorte zur Kodierung ist aber außerhalb der Standorte sehr ungenau.
Aus einer Historie von vorherigen Standorten kann aber ein möglicher Pfad inferiert werden,
allerdings können auch mehrere Pfade in Frage kommen, z. B. bei einer Gabelung.
Der kombinierte Ansatz kodiert soviele Standorte wie beide Ansätze zusammen,
wodurch dieser Ansatz am komplexisten ist und am schlechtesten für große Routen skaliert.
Dafür ist die Auflösung des kombinierten Ansatzes so gut, wie eine diskrete Kodierung es zulässt.
\newline
\newline
Ist in den aufgenommenen Trainingsdaten die Position des Objektes zum Zeitpunkt der Aufnahme der Sensorwerte bekannt,
so können die Standorte nach dem gewählten Kodierungsansatz beliebig genau beschriftet werden.
Dies kann in einem Weiterverarbeitungsschritt nach der Aufnahme der Daten mit einer Karte von den Interessepunkten geschehen.
\section{Entscheidungswald}
\label{sec:model_dt}
Entscheidungsbaum basierte Klassifizierer sind sehr effizient und können trotzdem hohe Klassifizierungsgenauigkeiten bei hohen Fehlertoleranzen erreichen \cite{dymelThesis}.
Entscheidungswälder erhöhen die Klassifizierungsgenauigkeit während die Varianz reduziert wird, dafür wird aber der Speicherbedarf mit jedem Baum linear erhöht.
Der Klassifizierer soll zukünftig auf einem Mikrocontroller ausgeführt werden, d. h. die Größe des Entscheidungswaldes ist durch den Programmspeicher des Mikrocontrollers limitiert.
Zu dem Zeitpunkt, wo diese Arbeit verfasst verfasst wird, sind die Limitierungen des Mikrocontrollers noch nicht bekannt.
Für gewöhnlich sind die Programmspeicher aber auf wenige Kilo-Byte beschränkt \cite{dymelThesis}.
\newline
\newline
Als Ensemble-Methode wird \textit{RandomForest} benutzt, da dieser Entscheidungsbäume auf Basis von zufälligen Teilmengen der Feature-Menge konstruiert.
Dadurch ist eine erhöhte Toleranz gegenüber Fehlern, wie fehlerhafte Sensorwerte oder anderen Features zu erwarten.
\newline
\newline
Um den Einfluss von verschiedenen Wald- und Baumgrößen auf die Fehlertoleranz und Klassifizierungsgenauigkeit hin zu untersuchen, werden verschiedene Größen
trainiert und auf den Testmengen evaluiert. Trotzdem wurden Dimensionen in einem Bereich gewählt, der für einen Mikrocontroller realistisch ist.
Es wurden jeweils Bäume und Wälder der Größe 8, 16, 32, und 64 untersucht.
\section{Feed Forward neuronales Netzwerk}
Für das KNN wird ein Feed Forward neuronales Netzwerk verwendet.
Das FFNN besteht aus drei bis sechs Schichten.
Alle Schichten, außer der letzten, verwenden ReLU als Aktivierungsfunktion.
Die letzte Schicht verwendet SoftMax.
\newline
\newline
Die Größe der Eingabeschicht ist abhängig von der Anzahl der verwendeten Features.
Die Größe der Ausgabeschicht ist die Anzahl der verschieden diskreten Standorte, die unterschieden werden.
Je nach Konfiguration gibt es 1, 2 oder 4 verdeckte Schichten, die jeweils 16, 32, 64 oder 128 Neuronen haben.
\newline
\newline
Die Standorte sind kategorische Daten, d. h. ihr Wert hate keine Aussage über die Beziehung der Standorte zueinander.
Aus diesem Grund werden sie kategorische Enkodiert, d. h. aus einem Wert $i$ aus $N$ möglichen Werten wird eine Liste der Größe $N$,
die überall 0 ist außer an der Stelle $i$, der 1 ist.
Folglich wird als Kostenfunktion \textit{kategorische Crossentropy} verwendet.
\newline
\newline
Als Lernalgorithmus wird Adam verwendet mit einer Batch-Größe von 50.
Trainiert wird über 75 Epochen.

\iffalse
\begin{itemize}
    \item Braucht man mehr Neuronen/Hidden Layer mit steigender Ort Anzahl?
\end{itemize}
\fi

\section{Feedback Kanten}
\begin{itemize}
    \item Was ist das?
    \item Wofür ist das Relevant?
    \item Wie funktioniert es?
    \item Wie wird das Problem umformuliert, für künstliche Rekurenz?
\end{itemize}

\section{Modell der Anomalieerkennung}
\begin{itemize}
    \item Wie funktioniert es?
    \item Welche Metriken werden verwendet?
\end{itemize}

\section{Training der Modelle}
\begin{itemize}
    \item Trainieren in Phasen => Erklären wie es genau funktioniert?
    \item Warum trainieren wir in Phasen? => Erkläre, dass wir Klassifizierungsfehler lernen wollen, damit wir wieder zurückfinden.
    \item Sag das wir die Zyklen der einzelnen Pfade nutzen
    \item Werden mehr Trainingsdaten benötigt mit steigender Ort Anzahl? Wenn ja wie viel?
    \item Doppeltes Trainieren nach Analyse der Feature Importance.
    \item Warum machen wir das so?
    \item Zufälliges Relabeling im Training? Warum und wie funktionierts?
    \item Anteil der Daten der gerelabelt wird steigt mit Anzahl der Zyklen
\end{itemize}
