\chapter{Machine Learning Modelle}
In dieser Arbeit wird die Device-Based Indoor-Lokalisation auf Basis von Sensorwerten untersucht.
Der Ansatz ist inspiriert von dem Orientierungssinn von Mensch und Tier.
Dabei werden diskrete Standorte unterschieden, sowie ob eine Anomalie entdeckt wurde,
d. h. ob das Modell sich an einem unbekannten Standort oder auf einem unbekannten Pfad befindet.
\newline
\newline
Abbildung \ref{fig:model_idea} zeigt die Architektur des verfolgten Ansatzes.
Zunächst werden aus den Sensorwerten Features extrahiert.
Die resultierende Feature-Menge wird dann vom ML-Modell genutzt, um den Standort zu klassifizieren.
Zuletzt wird auf Basis historischer Daten und dem Klassifizierungsergebnis von einem weiteren ML-Modell zur Anomalieerkennung bestimmt, ob eine Anomalie vorliegt.
\newline
\newline
Mian verwendete eine ähnliche Architektur \cite{naveedThesis}.
Sein FFNN zur Standorterkennung hatte ebenfalls eine Rückwärtskante, um die zuletzt erkannten Standorte als Features zu verwenden.
In dieser Arbeit wird zusätzlich versucht Anomalien zu erkennen.
Dafür wird ein weiteres ML-Modell verwendet, welches aus dem Klassifizierungsverhalten des ML-Modells zur Standorterkennung schließt, ob eine Anomalie vorliegt.
Dies motiviert die Erweiterung, um eine weitere Phase zur Feature-Extrahierung aus den zuletzt erkannten Standorten,
sowie einer Phase zur Evaluierung des ML-Modells zur Anomalieerkennung aus der resultierenden Feature-Menge.
Die Anomalieerkennung ist somit eine Information, die aus dem Standorterkennungsverhalten des vorangestellen ML-Modells resultiert,
weshalb sowohl der Standort als auch die Einschätzung als Anomalie als Ergebnis zurückgegeben wird.
\begin{figure}[h!]
    \centering
    \includegraphics[width=\linewidth]{images/model_idea.png}
    \caption{Architektur des verfolgten Ansatzes.}
    \label{fig:model_idea}
\end{figure}
\newline
In dieser Arbeit werden Entscheidungsbaum basierte Klassifizierer mit KNN verglichen, insbesondere den von Mian verwendeten Ansatz mit FFNN.
Entscheidungsbäume sind deutlich effizienter in der Ausführungszeit als KNN \cite{dymelThesis},
allerdings sind sie in ihrer Generalisierungsfähigkeit durch die berechneten Features begrenzt,
wohingegen KNN komplexe Features selbst erlernen können \cite{seide2011feature}.

\section{Standortkodierung}
\label{sec:model_location_encoding}
Als Standort wird ein einzigartiger diskreter Ort im Indoor-Szenario bezeichnet.
Bei der Klassifizierung können Standorte auf verschiedene Arten im ML-Modell kodiert werden.
Mian kodierte die Pfade zwischen Punkten, die von Interesse sind, als Standorte,
d. h. das Klassifizierungsergebnis ist der Pfad auf dem sich der Mikrocontroller befindet \cite{naveedThesis}.
\newline
\newline
In dieser Arbeit wird neben der Erkennung von Pfaden auch die Erkennung von Knoten, sowie die Erkennung von Pfaden und Knoten untersucht.
Pfade und Knoten liegen auf Routen, die als zyklischen Graphen betrachtet werden können, wie in Abbildung \ref{fig:location_encoding} illustriert.
Der Ansatz von Mian versucht die Kanten dieses Graphen zu klassifizieren und definiert diese als Standorte.
Daneben können auch nur die Knoten als Standorte kodiert werden und alle restlichen Datensätze als \textit{unbekannten Standort}.
Zuletzt können sowohl die Knoten und Kanten als Standorte definiert werden.
Die Knoten werden durch alle Datensätze dargestellt, die innerhalb eines Umkreises von einem Punkt sind, der von Interesse ist.
Folglich befinden sich alle restlichen Datensätze auf Kanten zwischen zwei Knoten oder gelten als \textit{unbekannt}.
\begin{figure}[h!]
    \centering
    \includegraphics[width=\linewidth]{images/location_encoding.png}
    \caption{Standortkodierung der Knoten und Pfade.}
    \label{fig:location_encoding}
\end{figure}
\newline
Daraus wird die Komplexität und Genauigkeit dieser Ansätze deutlich.
Der Kantenansatz ist ein Kompromiss zwischen Genauigkeit und Komplexität.
Dabei bestimmt die Anzahl der zu klassifizierenden Standorte die Komplexität.
Zum einen ist gerade bei langen Pfaden eine geringe Auflösung im Vergleich zur Realposition des Objektes zu erwarten,
d. h. es ist unklar, ob sich das Objekt am Anfang, Ende oder in dazwischen befindet.
Zum anderen werden in einem zyklischen Graphen mindestens so viele Standorte, wie beim Knotenansatz verwendet.
Der Knotenansatz benötigt am wenigsten Standorte zur Kodierung ist aber außerhalb der Standorte sehr ungenau.
Aus einer Historie von vorherigen Standorten kann aber ein möglicher Pfad inferiert werden,
allerdings können auch mehrere Pfade in Frage kommen, z. B. bei einer Gabelung.
Der kombinierte Ansatz kodiert soviele Standorte wie beide Ansätze zusammen,
wodurch dieser Ansatz am komplexisten ist und am schlechtesten für große Routen skaliert.
Dafür ist die Auflösung des kombinierten Ansatzes so gut, wie eine diskrete Kodierung es zulässt.
\newline
\newline
Ist in den aufgenommenen Trainingsdaten die Position des Objektes zum Zeitpunkt der Aufnahme der Sensorwerte bekannt,
so können die Standorte nach dem gewählten Kodierungsansatz beliebig genau beschriftet werden.
Dies kann in einem Weiterverarbeitungsschritt nach der Aufnahme der Daten mit einer Karte von den Interessepunkten geschehen.
\section{Entscheidungswald}
\label{sec:model_dt}
Entscheidungsbaum basierte Klassifizierer sind sehr effizient und können trotzdem hohe Klassifizierungsgenauigkeiten bei hohen Fehlertoleranzen erreichen \cite{dymelThesis}.
Entscheidungswälder erhöhen die Klassifizierungsgenauigkeit während die Varianz reduziert wird, dafür wird aber der Speicherbedarf mit jedem Baum linear erhöht.
Der Klassifizierer soll zukünftig auf einem Mikrocontroller ausgeführt werden, d. h. die Größe des Entscheidungswaldes ist durch den Programmspeicher des Mikrocontrollers limitiert.
Zu dem Zeitpunkt, wo diese Arbeit verfasst verfasst wird, sind die Limitierungen des Mikrocontrollers noch nicht bekannt.
Für gewöhnlich sind die Programmspeicher aber auf wenige Kilo-Byte beschränkt \cite{dymelThesis}.
\newline
\newline
Als Ensemble-Methode wird \textit{RandomForest} benutzt, da dieser Entscheidungsbäume auf Basis von zufälligen Teilmengen der Feature-Menge konstruiert.
Dadurch ist eine erhöhte Toleranz gegenüber Fehlern, wie fehlerhafte Sensorwerte oder anderen Features zu erwarten.
\newline
\newline
Um den Einfluss von verschiedenen Wald- und Baumgrößen auf die Fehlertoleranz und Klassifizierungsgenauigkeit hin zu untersuchen, werden verschiedene Größen
trainiert und auf den Testmengen evaluiert. Trotzdem wurden Dimensionen in einem Bereich gewählt, der für einen Mikrocontroller realistisch ist.
Es wurden jeweils Bäume und Wälder der Größe 8, 16, 32, und 64 untersucht.
\section{Feed Forward neuronales Netzwerk}
Für das KNN wird ein Feed Forward neuronales Netzwerk verwendet.
Das FFNN besteht aus drei bis sechs Schichten.
Alle Schichten, außer der letzten, verwenden ReLU als Aktivierungsfunktion.
Die letzte Schicht verwendet SoftMax.
\newline
\newline
Die Größe der Eingabeschicht ist abhängig von der Anzahl der verwendeten Features.
Die Größe der Ausgabeschicht ist die Anzahl der verschieden diskreten Standorte, die unterschieden werden.
Je nach Konfiguration gibt es 1, 2 oder 4 verdeckte Schichten, die jeweils 16, 32, 64 oder 128 Neuronen haben.
\newline
\newline
Die Standorte sind kategorische Daten, d. h. ihr Wert hate keine Aussage über die Beziehung der Standorte zueinander.
Aus diesem Grund werden sie kategorische Enkodiert, d. h. aus einem Wert $i$ aus $N$ möglichen Werten wird eine Liste der Größe $N$,
die überall 0 ist außer an der Stelle $i$, der 1 ist.
Folglich wird als Kostenfunktion \textit{kategorische Crossentropy} verwendet.
\newline
\newline
Als Lernalgorithmus wird Adam verwendet mit einer Batch-Größe von 50.
Trainiert wird über 75 Epochen.

\iffalse
\begin{itemize}
    \item Braucht man mehr Neuronen/Hidden Layer mit steigender Ort Anzahl?
\end{itemize}
\fi
\section{Training von KNN}
Um die Zielfunktion zu approximieren ist eine Kostenfunktion nötig, die den Abstand von der approximierten Funktion zu der Zielfunktion angibt.
Im Optimierungsprozess wird die Kostenfunktion minimiert. Oft wird die Kostenfunktion auch als Verlustfunktion bezeichnet.
Die Kostenfunktion kann aber zusätzlich noch Regulurisierungsterme besitzen.
Üblicherweise wird Kreuzentropie für die Verlustfunktion verwendet, da sich diese experimentell als Beste Verlustfunktion für flache KNN erwiesen hat.
\newline
\newline
Das KNN wird für eine endliche Anzahl an Trainingsdurläufen trainiert.
Ein Trainingsdurchlauf einer Trainingsmenge wird als Epoche bezeichnet.
Mit jeder Epoche werden die Kosten erhoben und der Optimierer wird ausgeführt.
Der Fehler wird dabei rückwerts durch das KNN propagiert, wodurch die Parameter $\boldsymbol\theta$ sich verändern.
Dieser Vorgang wird als \textit{Backpropagation} bezeichnet.
Zu den Parametern gehört die Struktur des neuronalen Netzwerks, Aktivierungsfunktionen, Gewichte und Bias.
Mit Fehler ist der Unterschied von dem des KNN ermittelten Wertes zu dem Zielwert gemeint.
\newline
\newline
Abhängig von der Strategie des Optimierers müssen die Gradienten rückwerts, rekursiv Schicht für Schicht zurück propagiert werden.
Ziel ist es den Fehler unter Berücksichtigung der Parameter zurück zu propagieren.
Dafür ist es nötig die Ableitungen von der Kostenfunktion zu den Parametern zu berechnen.
Gleichung \ref{formular:general_knn} zeigt, dass das Ergebnis einer Schicht die Aktivierung der gewichteten Summe des
Ergebnises der jeweiligen vorherigen Schicht ist.
Um die Ableitungen unter Berücksichtigung der Parameter zu berechnen ist die Anwendung der Kettenregel nötig.
\newline
\newline
Angefangen mit dem Gradienten der Aktivierungen der Ausgabeschicht, wird der Fehler rekursiv zurück propagiert.
Dabei wird in jeder Schicht die Korrektur auf die Parameter addiert.
Anschließend kann der Vorgang für nächste Epoche wiederholt werden.
\newline
\newline
TODO: Batchsize
\section{Klassifizierungsgenauigkeit der Anomalien}
\label{sec:eval_anomalieerkennung}
Bei der Anomalieerkennung werden Entscheidungswälder und FFNNs mit den besten ML-Modellen zur Standorterkennung trainiert und mit den drei Baseline-Modellen verglichen.
Tabelle \ref{tab:anomaly_detection_prediction_accuracy} zeigt die Klassifizierungsgenauigkeiten über die verschiedenen Standortkomplexitäten,
wobei die Klassifizierungsgenauigkeit $P(A)$ nochmal genauer aufgeschlüsselt ist in den Anteil der korrekten Klassifizierungen, wenn eine bzw. keine Anomalie vorlag.
Die trainierten FFNNs geben stets aus, dass keine Anomalie vorliegt.
Es ist unklar, warum die FFNNs sich so verhalten.
\begin{table}[h!]
    \hspace{-1cm}
    \begin{tabular}{ | l | c | c | c | c | c | c | c | c | }
        \hline
        Standorte & 9 & 16 & 17 & 25 & 32 & 48 & 52 & 102 \\\hline
        \multicolumn{9}{ | l |}{$P(A)$}\\\hline
        Entscheidungswald & 82,59\% & 81,19\% & 87,14\% & 84,91\% & 79,06\% & 83,47\% & 81,93\% & 76,00\% \\\hline
        FFNN & 77,88\% & 77,88\% & 77,88\% & 77,88\% & 77,88\% & 77,88\% & 77,88\% & 77,88\% \\\hline
        Topologie (DT) & 84,77\% & 30,57\% & 83,51\% & 79,76\% & 28,63\% & 24,97\% & 80,55\% & 29,47\% \\\hline
        Topologie (KNN) & 86,10\% & 52,17\% & 77,72\% & 79,30\% & 45,06\% & 41,92\% & 74,77\% & 43,55\% \\\hline
        \multicolumn{9}{ | l |}{Anteil korrekt klassifiziert, indem Anomalie vorlag}\\\hline
        Entscheidungswald & 34,86\% & 35,52\% & 52,58\% & 50,92\% & 32,21\% & 50,64\% & 23,21\% & 1,92\% \\\hline
        FFNN & 0,00\% & 0,00\% & 0,00\% & 0,00\% & 0,00\% & 0,00\% & 0,00\% & 0,00\% \\\hline
        \multicolumn{9}{ | l |}{Anteil korrekt klassifiziert, indem keine Anomalie vorlag}\\\hline
        Entscheidungswald & 96,14\% & 94,41\% & 97,05\% & 95,83\% & 92,48\% & 93,13\% & 98,96\% & 97,18\% \\\hline
        FFNN & 100,00\% & 100,00\% & 100,00\% & 100,00\% & 100,00\% & 100,00\% & 100,00\% & 100,00\% \\\hline
    \end{tabular}
    \caption{Metrik $P(A)$ über Standorte und verschiedenen Konfigurationen der Modelle zur Anomalieerkennung.}
    \label{tab:anomaly_detection_prediction_accuracy}
\end{table}
\newline
\newline
Die Entscheidungswälder hingegen eignen sich besser für den Anomalieerkennungszweck.
Es werden zwischen 1,92\% und 52,58\% der Anomalien erkannt und zwischen 1,04\% und 7,52\% falsch als Anomalien erkannt.
Die Klassifizierungsgenauigkeit des Entscheidungswaldes zur Anomalieerkennung ist abhängig von der Klassifizierungsgenauigkeit zur Standorterkennung
und von der Standortkomplexität.
Je besser das Standorterkennungsmodell und je höher die Standortkomplexität, desto höher ist die Anomalieerkennungsrate.
Aus diesem Grund ist die Klassifizierungsgenauigkeit bei den Standortkomplexitäten, die mit der Kodierungsmethode mit Kanten und Knoten zusammenhängen,
geringer, als bei der Kodierungsmethode, wo nur die Knoten kodiert werden.



\begin{itemize}
    \item Was ist das?
    \item Zusammenhang zu Enkodierungsansatz
    \item Schwierig zu trainieren, da man einerseits Robust sein will und andererseits nicht weiß was trainiert werden soll
    \item Post Processing
    \item Metriken: Location Change Frequency, Accumulated Confididence, Fraction Zero
    \item Sag Motivation, warum diese Metriken (Indikatoren)
    \item Beispiele
    \item Wie zuverlässig können Anomalien erkannt werden?
    \item Vergleich mit Coin-Toss, True und False
    \item Vergleiche Enkodierungsansätze, ob einer besser als der andere ist
\end{itemize}

