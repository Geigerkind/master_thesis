\section{Anomalieerkennung}
Als Anomalieerkennung wird in dieser Arbeit das Problem bezeichnet, zu erkennen, dass das Objekt sich an einem
unbekannten Standort oder auf einem unbekannten Pfad befindet.
Abbildung \ref{fig:model_idea} zeigt, dass die Anomalieerkennung ein eigener Schritt bei der Evaluierung der Sensordaten ist.
Die Eingabe sind Features, die auf historischen Daten und dem momentanen Standort basieren.
\newline
\newline
Es wird ein zweites ML-Modell trainiert, anstatt dem ML-Modell zur Standorterkennung einen \textit{Anomaliestandort} lernen zu lassen.
Dies ist begründet auf der Schwierigkeit Trainingsdaten für Anomalien basierend auf den Sensordaten zu entwickeln,
da es unendlich viele Szenarien geben könnte, die als Anomalie zu bezeichnen sind.
Stattdessen werden Features genutzt, die eine Abweichung von der Normalität ausdrücken,
d. h. das ML-Modell lernt nicht explizite Anomaliepfade, sondern das Verhalten des Standortklassifizierungsmodells einzuordnen.
\newline
\newline
Dafür werden analog zu Kapitel \ref{sec:model_dt} und \ref{sec:model_ffnn} Entscheidungswälder und FFNN trainiert.
Allerdings bedarf dieses Modell keine Rückwärtskante und das FFNN kann für binäre Klassifizierung vereinfacht werden.
Die letzte Schicht des FFNN hat ein Neuron und nutzt die Sigmoid-Funktion, anstatt der SoftMax-Funktion.
Außerdem wird für die Kostenfunktion \textit{binäre Crossentropy} verwendet, anstatt kategorische Crossentropy.
Binäre Crossentropy bedarf keine kategorische Enkodierung im Gegensatz zur kategorischen Crossentropy.
\newline
\newline
Die ML-Modelle zur Anomalieerkennung können seperat von den ML-Modellen zur Standorterkennung trainiert werden.
Sie werden im Gegensatz zu den ML-Modellen zur Standorterkennung nur einmalig trainiert mit vorbereiteten Trainingsdaten.
Die Trainingsdaten werden aus den Anomaliedatenmengen und Datenmengen für die verschieden Routen generiert.
Dafür werden die Klassifizierungsergebnisse auf diesen Datenmengen von den zuvor trainierten ML-Modellen zur Standorterkennung genutzt.
Daraus werden beschriftete Features extrahiert, die in Kapitel \ref{sec:data_anomalie} detailiert erläutert werden.