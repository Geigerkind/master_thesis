\chapter{Diskussion}
Ein Großteil dieser Arbeit ist die Generierung von Sensordaten.
Dazu wurde der Ansatz von Mian aufgegriffen, wodurch verschiedene Routen in CoppeliaSim simuliert wurden.
Der Nachteil dieses Ansatzes ist die schlechte Skalierbarkeit, wenn es um die Untersuchung von Pfaden und Standorten geht.
Um weitere Pfade und Standorte zu untersuchen, mussten entweder komplexere Routen in CoppeliaSim erstellt werden,
Routen synhetisch kombiniert werden oder Routen unabhängig von einander trainiert werden.
\newline
\newline
Der erste Ansatz ist sehr aufwändig und der zweite Ansatz unterscheidet sich nicht stark von dem dritten Ansatz,
da die Routen nur an einem Punkt kombiniert werden.
Ein besserer Ansatz wäre es gewesen, wenn man lediglich abgeschlossene Teilstücke simuliert hätte,
z.~B. eine Kurve oder eine Gerade nach links laufend.
Jedes dieser Teilstücke hätte eine feste Länge und könnte sogar transformiert werden in sowohl der Länge, als auch den aufgenommen Sensordaten.
Durch geschickte Kombination, könnten dann beliebig komplexe Routen mit beliebig vielen Pfaden und Standorten generiert werden,
was die Untersuchung erheblich erleichtert hätte.
