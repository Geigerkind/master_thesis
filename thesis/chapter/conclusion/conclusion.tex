\chapter{Schlussfolgerungen}
Diese Arbeit hat gezeigt, dass sich Klassifizierer mit Entscheidungsbäumen besser als FFNNs zur Standortbestimmung eignen.
Es können mit kleinen Entscheidungswäldern, Klassifizierungsgenauigkeiten von 98,62\% bei einer geringen Standortkomplexität von 9 Orten
und 87,35\% bei einer hohen Standortkomplexität von 102 Orten erzielt werden.
Anomalien von üblichen Wegen konnten nur bis zu 52,58\% bei einer Fehlerrate von 2,95\% korrekt klassifiziert werden.
\newline
\newline
Die untersuchten ML-Modelle sind klein genug, um auf Mikrocontrollern ausgeführt zu werden.
Es wurde gezeigt, dass ein Datenfenster von drei Einträgen ausreicht, um die Sensorwerte zu glätten und Features zu extrahieren.
Entscheidungswälder benötigen zwar mehr Programmspeicher, dafür aber weniger RAM und deutlich weniger Operationen als FFNNs, wodurch
ein geringerer Energiebedarf zu erwarten ist.
\newline
\newline
Der Kodierungsansatz, der nur Knoten kodiert, skaliert besser als der Kodierungsansatz, der Kanten und Knoten kodiert.
Resultierende ML-Modelle haben damit bessere Klassifizierungsergebnisse erzielt.
Entscheidungswälder skalieren mit steigender Standortkomplexität besser als FFNNs.
Mit jedem zusätzlichen Standort nimmt die Klassifizierungsgenauigkeit für Entscheidungswälder um 0,1 Prozentpunkte ab,
wohingegen sie für FFNNs um 0,24 Prozentpunkte abnimmt.
\newline
\newline
Die Wichtigkeit der Features ist vom Einsatzszenario abhängig.
Zur Einschätzung der Wichtigkeit haben sich die Permutationswichtigkeit \cite{breiman2001random} und ähnliche Modifizierungen der Testmengen erwiesen,
die die Wichtigkeit durch den Klassifizierungsfehler im Vergleich zur originalen Testmenge bestimmen, z. B. die Nullung einzelner Features.
\newpage
Die ML-Modelle wurden mit fehlerhaften Daten trainiert, um die Fehlertoleranz zu erhöhen.
Es hat sich gezeigt, dass dadurch sowohl Entscheidungswälder als auch FFNNs robust gegenüber Abweichungen der Sensorwerte,
Ausfall von einzelnen Sensoren und Permutation der Routen sind.
Der durchschnittliche Fehler in diesen Szenarien beträgt 2,91 Prozentpunkte für Entscheidungswälder und 6 Prozentpunkte für FFNNs.
Im Schnitt benötigt ein Entscheidungswald 1,9- bis 3,4 Klassifizierungen, um den korrekten Standort wieder zu bestimmen,
nachdem es einen falschen Standort bestimmt hat.
Ein FFNN benötigt im Schnitt 2,2- bis 4,4 Klassifizierungen.
\newline
\newline
Das Training mit einer Rückwärtskante, die erkannte Standorte für die nachfolgende Klassifizierung als Feature verfügbar macht, ist sehr aufwändig und rechenintensiv.
Ohne die Rückwärtskante konnten bessere Klassifizierungsergebnisse erzielt werden als mit Rückwärtskante.
Dabei ist das resultierende ML-Modell kleiner und unkomplizierter.
FFNNs konnten ohne Rückwärtskante deutlich bessere Klassifizierungsergebnisse erzielen.
\newline
\newline
Zukünftige Arbeiten sollten die untersuchten Ansätze mit Echtdaten untersuchen und ein Modell zur Abschätzung des Energiebedarfs erstellen.
Um die ML-Modelle mit Echtdaten zu untersuchen, muss ein Prototyp der Sensorenbox konstruiert werden.
Möglicherweise könnte ein Arduino Board mit handelsüblichen, vergleichbaren Sensoren, oder ein SmartPhone ausreichen.
In einem geeigneten Szenario müssen dann ausreichend Daten aufgenommen werden.
\newline
\newline
Das Energiemodell soll für verschiedene ML-Modelle, Batteriegrößen und Mikrocontroller erstellt werden.
Mit den resultierenden oberen Schranken für die Größen verschiedener ML-Modelle,
können dann gezielter ML-Ansätze untersucht werden.