\chapter{Schlussfolgerungen}
TODO

\begin{itemize}
    \item Kurze, knappe und gut formulierte Schlussfolgerung
    \begin{itemize}
        \item Was ist besser Entscheidungsbaum oder Entscheidungswald? Welche Vor- und Nachteile?
        \item Sollten Entscheidungswälder verwendet werden? => Vor und Nachteile Besprechen
        \item Sollten KNNs verwendet werden? => Vor und Nachteile Besprechen
        \item Is die Realzeit relevant für die Ortung(?)
        \item Wie Fehlertolerant ist die Ortung?
        \item Wie gut könenn Anomalien erkannt werden?
    \end{itemize}
    \item Kurze Zusammenfassung wichtigster Dinge
    \item Zukünftige Arbeit
    \item Future Work:
    \begin{itemize}
        \item Weitere Optimierung der Modelle damit Größe und Laufzeit verringert werden kann,
              e. g. greedy Verkleinerung guter Modelle oder GA für die Suche guter Parameter
        \item Bei Entscheidungsbäume kann man das Cherry-Pickung über verschiedene Ensemble Methoden erweitern
        \item Man könnte RNN untersuchen, diese sollten sehr gut für diese Aufgabe geeignet sein. Außerdem spielt Evaluierungszeit nicht wirklich eine Rolle, solange es nicht häufig evaluiert wird.
        Es wäre interessant zu sehen, wie gut man werden kann.
        \item FFNN wie Mian trainieren nur mit vielen sehr kleinen Hidden Layern. Anschließend Prunen und Layer entfernen (siehe Begriff von Mathe NN), die keine Features erlernten bzw. nicht relevant sind.
        Also Vorteile von Deep NN nutzen.
    \end{itemize}
\end{itemize}