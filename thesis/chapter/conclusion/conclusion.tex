\chapter{Schlussfolgerungen}
\begin{itemize}
    \item Kurze, knappe und gut formulierte Schlussfolgerung
    \begin{itemize}
        \item Was ist besser Entscheidungsbaum oder Entscheidungswald? Welche Vor- und Nachteile?
        \item Sollten Entscheidungswälder verwendet werden? => Vor und Nachteile Besprechen
        \item Sollten KNNs verwendet werden? => Vor und Nachteile Besprechen
        \item Is die Realzeit relevant für die Ortung(?)
        \item Wie Fehlertolerant ist die Ortung?
        \item Wie gut könenn Anomalien erkannt werden?
    \end{itemize}
    \item Kurze Zusammenfassung wichtigster Dinge
\end{itemize}
Zukünfitge Arbeiten sollten drei Bereiche weiter untersuchen.
Zunächst sollten diese ML-Modelle mit Echtdaten untersucht werden.
Dafür müsste ein Prototyp der Sensorenbox konstruiert werden und in
einem geeigneten Szenario ausreichend Daten aufgenommen werden.
Möglicherweise könnte ein Arduino Board mit handelsüblichen, vergleichbaren Sensoren ausreichen.
\newline
\newline
Weitere Optimierungen der ML-Modelle sollten untersucht werden.
Zunächst könnte die Anzahl der Features reduziert werden durch eine sorgfältige Auswahl mit Hilfe eines Optimierungsverfahrens.
Dort sollte die Klassifizierungsgenauigkeit und Ausführungszeit maximiert werden unter Einhaltung der Hardwarelimitierungen.
Zur Optimierung von KNN könnten die Ansatze von Giese aufgegriffen werden \cite{gieseThesis}.
\newline
\newline
Ein Energiemodell zur Einschätzung des Energieverbrauchs der verschiedenen ML-Modelle könnte erstellt werden.
Dieses könnte für verschiedene Batteriegrößen und Mikrocontroller, Einschätzungen für obere Schranken der Größe von
verschiedenen ML-Ansätzen angeben.
Damit können dann gezielter weitere Ansätze untersucht werden.