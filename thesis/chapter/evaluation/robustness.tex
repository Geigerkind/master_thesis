\section{Fehlertoleranz}
Bei der Fehlertoleranz wird die Fähigkeit der ML-Modelle untersucht, mit fehlerhaften Sensordaten Standorte zu erkennen.
Dafür wurden für jeden Sensor modifizierte Testmengen erstellt.
Die erste Testmenge fügt ein Rauschen von 5\% hinzu und die zweite Testmenge simuliert den Ausfall des Sensors, indem alle Sensorwerte genullt werden.
Zudem wurde untersucht, was passiert wenn die Sensorenbox nicht dem trainierten Pfad folgt, indem die Testmenge permutiert wurde.
Damit Entscheidungswald und FFNN fair verglichen werden können, werden die besten ML-Modelle der Standortkomplexität 9 verwendet.
\newline
\newline
Tabelle \ref{tab:robustness} zeigt die Differenz der Klassifizierungsgenauigkeiten $P(A)_{\text{cont}}$ und $P(A)$ von den modifizierten Testmengen zur originalen Testmenge.
Die Testmengen mit einem Rauschen von 5\% wurde ausgelassen, da es keine Auswirkung auf die Klassifizierungsgenauigkeit hatte.
Vermutlich ist 5\% Rauschen zu wenig, um einen Einfluss auszuüben.
\begin{table}[h!]
    \hspace{-1.25cm}
    \begin{tabular}{ | l | c | c | c | c | }
        \hline
        Testmenge & Entscheidungswald & FFNN & Entscheidungswald & FNNN \\\hline
        & \multicolumn{2}{ c }{mit Rückwärtskante} & \multicolumn{2}{| c |}{ohne Rückwärtskante} \\\hline
        Licht & 4.46\%-Pkt. & 4.65\%-Pkt. & 5.28\%-Pkt. & 6.93\%-Pkt \\\hline
        Geräusch & 3.20\%-Pkt. & 5.00\%-Pkt. & 1.63\%-Pkt. & 5.11\%-Pkt. \\\hline
        Temperatur & 15.15\%-Pkt. & 6.60\%-Pkt. & 8.10\%-Pkt. & 13.50\%-Pkt. \\\hline
        Ausrichtung zum Magnetfeld & 3.32\%-Pkt. & 19.94\%-Pkt. & 2.51\%-Pkt. & 2.78\%-Pkt. \\\hline
        WLAN-Zugangspunkte & 2.60\%-Pkt. & 22.65\%-Pkt. & 3.74\%-Pkt. & 14.13\%-Pkt. \\\hline
        Accelerometer & 1.41\%-Pkt. & 9.52\%-Pkt. & 0.62\%-Pkt. & 1.33\%-Pkt. \\\hline
        Gyroskop & 8.52\%-Pkt. & 4.58\%-Pkt. & 0.91\%-Pkt. & 3.30\%-Pkt. \\\hline
        Permutierte Testmenge & 2.27\%-Pkt. & -0.13\%-Pkt. & 0.47\%-Pkt. & 0.93\%-Pkt. \\\hline
        \textbf{Durchschnitt} & \textbf{5,8\%-Pkt.} & \textbf{9,1\%-Pkt.} & \textbf{2,91\%-Pkt.} & \textbf{6,00\%-Pkt.} \\\hline
    \end{tabular}
    \caption{Fehler der modifizierten Testmengen zur originalen Testmenge.}
    \label{tab:robustness}
\end{table}
\newline
\newline
Die ML-Modelle mit und ohne Rückwärtskante sind robust gegenüber der Nullung der Features und gegenüber der permutierten Testmenge.
Aus der Menge stechen die Fehler durch die Nullung der Features des Temperatur- und Magnetfeldsensors, sowie der WLAN-Zugangspunkte.
Außerdem ist der Fehler durch die Nullung der Features des Accelerometers und Gyroskops bei den ML-Modellen mit Rückwärtskanten
deutlich größer als bei den ML-Modellen ohne Rückwärtskante.
\newline
\newline
Die Permutationswichtigkeit hat den Features des Temperatursensors eine geringere Wichtigkeit zugeordnet, als die Nullung es tut.
Dies ist dadurch begründet, dass die Sensordaten des Temperatursensors mit wenigen Ausnahmen sehr homogen sind.
Für den Temperatursensor wird eine Umgebungstemperatur simuliert, die sich nur verändert, wenn die Sensorenbox einer Temperaturquelle näher kommt.
Für den größten Teil der Daten misst der Temperatursensor die Umgebungstemperatur, weswegen eine permutation keinen großen Fehler verursacht.
Die Nullung dieser Sensordaten hingegen deutet auf eine Temperaturquelle hin, die die Umgebungstemperatur verringert.
Dieses Ereignis ist im Vergleich zu einer Erhöhung der Umgebungstemperatur selten, weswegen die Nullung einen großen Fehler verursacht.
Würde dieses Ereignis häufiger vorkommen, wäre der Fehler vermutlich geringer.
\newline
\newline
Das FFNN mit Rückwärtskante hat im Vergleich zu den anderen ML-Modellen einen deutlich größeren Fehler, wenn die Features des Magnetfeldsensors genullt werden.
Diese Anomalie ist entgegen den Erwartungen der Permutationswichtigkeit, insbesondere da die anderen ML-Modelle
höhere Permutationswichtigkeiten für die Features des Magnetfeldsensors erzeugt haben.
Es ist unklar, warum dieses FFNN, im Vergleich zu den anderen ML-Modellen so anfällig dem gegenüber ist.
\newline
\newline
Die FFNNs erzeugen einen großen Fehler, wenn die WLAN-Zugangspunkte genullt werden.
Dies stimmt mit den Ergebnissen der Permutationswichtigkeit überein.
Insgesamt bilden die Features der WLAN-Zugangspunkte 14,7\% aller Features, wobei die Werte bei der Eingabeschicht binär sind.
Vermutlich ist aus diesem Grund der Einfluss dieser Features beim FFNN im Vergleich zu den Entscheidungswäldern so groß.
\newline
\newline
Die permutation der Testmenge hat nur einen geringen Fehler verursacht.
Dies war bei allen ML-Modellen zu erwarten, da das interne Datenfenster mit drei Einträgen sehr klein ist
und die Features der vorherigen Standorte bereits nach wenigen Klassifizierungen korrigiert werden.
Dementsprechend ist der Fehler bei den ML-Modellen ohne Rückwärtskante auch deutlich kleiner als bei den ML-Modellen mit Rückwärtskante.
\newline
\newline
Im Durchschnitt sind die ML-Modelle ohne Rückwärtskante robuster als die ML-Modelle mit Rückwärtskante.
Die Entscheidungswälder sind robuster als die FFNNs.
Dieser Fehler korreliert aber mit der Klassifizierungsgenauigkeit der Modelle (Abbildung \ref{fig:best_dt_vs_knn_fb_vs_no_fb}).
Der Entscheidungswald mit Rückwärtskante, der marginal bessere Klassifizierungsgenauigkeiten erzielt hat, als das FFNN ohne Rückwärtskante,
hat einen marginal geringeren Fehler erzeugt.
\newline
\newline
Tabellen \ref{tab:predictions_by_acc_pic_cont} und \ref{tab:predictions_by_acc_pic_wo_fb} geben die Klassifizierungsgenauigkeit $P(C)_{\text{cont}}$ bzw. $P(C)$ an.
Diese geben die Wahrscheinlichkeit an, dass ein Standort korrekt klassifiziert wird, wenn der Standort zuvor falsch klassifiziert wurde.
Wie zu erwarten ist die Wahrscheinlichkeit bei den ML-Modellen ohne Rückwärtskante deutlich größer.
Bei einer Standortkomplexität von 9 Orten benötigt ein Entscheidungswald mit Rückwärtskante ca. 5,3 Klassifizierungen,
ein FFNN mit Rückwärtskante ca. 5,8 Klassifizierungen, ein Entscheidungswald ohne Rückwärtskante ca. 1,9 Klassifizierungen
und ein FFNN ohne Rückwärtskante ca. 2,2 Klassifizierungen.
Bei einer Standortkomplexität von 102 Orten werden 8,1-, 25-, 3,4- und 4,4 Klassifizierungen benötigt.
Dies bestätigt, dass die ML-Modelle ohne Rückwärtskante deutlich robuster sind als mit Rückwärtskante.
\newline
\newline
Von Entscheidungswäldern ist zu erwarten, dass sie mit steigender Waldgröße robuster werden, da sich die Feature-Mengen der einzelnen Entscheidungsbäume unterscheiden.
Dies wird von den Klassifizierungsergebnissen teilweise gestützt, allerdings ist unklar, ob dies nicht nur mit der damit steigenden Klassifizierungsgenauigkeit zusammenhängt.
\newline
\newline
Im Vergleich zu Mian sind die ML-Modelle dieser Arbeit deutlich robuster.
Der Ausfall des Lichtsensors bei Mians Ansätzen hat einen Fehler von 88,15 Prozentpunkten verursacht \cite{naveedThesis},
wohingegen der Fehler des schlechtesten ML-Modells in dieser Arbeit gegenüber dem Ausfall des Lichtsensors nur 6,93 Prozentpunkten ist.
\newline
\newline
In dieser Arbeit haben sich Entscheidungswälder als robuster gegenüber Fehler herausgestellt.
Allerdings erzielen die Entscheidungswälder in dieser Arbeit auch insgesamt bessere Klassifizierungsergebnisse, weswegen dies zu erwarten ist.
Der durchschnittliche Fehler beider ML-Modelle ist vergleichbar, bei vergleichbarer Klassifizierungsgenauigkeit.
Zusätzlich hat sich die Untersuchung von modifizierten Testmengen mit gezielten Veränderungen als Ergänzung zur Permutationswichtigkeit bewiesen,
um die Signifikanz von einzelnen Features einzuschätzen.