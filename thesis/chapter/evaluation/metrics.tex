\newpage
\section{Metriken}
In dieser Arbeit werden verschiedene Metriken zur Ermittlung der Klassifizierungsgenauigkeit verwendet.
Zunächst die übliche Klassifizierungsgenauigkeit (\ref{formular:simple_accuracy}), in der die Anzahl der korrekt klassifizierten Standorte mit der Gesamtanzahl verglichen werden.
Diese Metrik wird auch von Mian verwendet und ist für den direkten Vergleich erforderlich.
\begin{align}
    \label{formular:simple_accuracy}
    P(A) := \frac{\text{Anzahl korrekter Klassifizierungen}}{\text{Gesamtanzahl}}
\end{align}
Die zweite Metrik (\ref{formular:accuracy_metrik2}) betrachtet die Klassifizierungsgenauigkeit unter Tolerierung, dass ein Standort
kontinuierlich zu früh oder zu spät verlassen wurde,
d.~h. Fehlklassifizierungen werden vernachlässigt, wenn kontinuierlich der letzte korrekte Standort bzw. der nächste korrekte
Standort klassifiziert wird mit einer Gesamttoleranz von $\epsilon$ Klassifizierungen.
Diese Metrik ist besonders gut für den Vergleich geeignet, da entgegen der Beschriftung in der Testmenge,
harte Übergänge zwischen zwei benachbarten Standorten nicht der Wirklichkeit entsprechen.
Im Übergang können Sensorwerte möglicherweise mehrdeutig sein und dies sollte bei der Standortbestimmung, solange sie konsequent ist, beachtet werden.
Dadurch wird, im Vergleich, das Rauschen in der Klassifizierungsgenauigkeit so reduziert, dass durch diese Mehrdeutigkeit im Übergang entsteht,
wodurch die ML-Modelle besser vergleichbar sind.
\begin{flalign}
    \label{formular:accuracy_metrik2}
    &L := \text{Menge der von dem ML-Modell klassifizierten Standorte.} \nonumber\\
    &K := \text{Menge von den wirklichen Standorten.} \nonumber\\
    &\Phi(i) := \text{Index vom nächsten Standort.} \nonumber\\
    &\Psi(i) := \text{Index vom vorherigen Standort.} \nonumber\\
    &\Omega(i) := \Phi(i)-i\leq\epsilon\wedge\hspace{-0.3cm} \bigwedge\limits_{i\leq q \leq \min(\#K, \Phi(i))}\hspace{-0.3cm} L_q=K_{\Phi(i)} \nonumber\\
    &\Theta(i) := i-\Psi(i)\leq\epsilon\wedge\hspace{-0.3cm} \bigwedge\limits_{\max(0, \Psi(i))\leq q \leq i}\hspace{-0.3cm} L_q=K_{\Psi(i)} \nonumber\\
    &P(B\leq\epsilon) := \frac{\#\{L_i | L_i=K_i \vee \Omega(i) \vee \Theta(i)\text{ für } i\in\{0, 1, ..., \#L - 1\}\}}{\#K}
\end{flalign}
Zuletzt wird die Klassifizierungsgenauigkeit betrachtet, unter der Bedingung, dass der vorherige Standort inkorrekt war.
Diese Metrik ist besonders relevant bei der Beurteilung der Stabilität der ML-Modelle.
\begin{align}
    \label{formular:accuracy_previous_was_in_correct}
    P(C) := \frac{\text{Anzahl korrekter Klassifizierungen, wenn vorheriger Standort inkorrekt war}}{\text{Alle Klassifizierungen, wenn vorheriger Standort inkorrekt war}}
\end{align}