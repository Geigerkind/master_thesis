\section{Metriken}
In dieser Arbeit werden verschiedene Metriken zur Ermittlung der Klassifizierungsgenauigkeit ermittelt.
Zunächst die übliche Klassifizierungsgenauigkeit (\ref{formular:simple_accuracy}), in der die Anzahl der korrekt klassifizierten Standorte mit der Gesamtanzahl verglichen werden.
\begin{align}
    \label{formular:simple_accuracy}
    P(A) := \frac{\text{Anzahl korrekter Klassifizierungen}}{\text{Gesamtanzahl}}
\end{align}
Die zweite Metrik (\ref{formular:accuracy_metrik2}) betrachtet die Klassifizierungsgenauigkeit unter Tolerierung, dass ein Standort
fünf bzw. zehn Klassifizierungen kontinuierlich zu früh oder zu spät verlassen wurde,
d.~h. Fehlklassifizierungen werden vernachlässigt, wenn kontinuierlich der letzte korrekte Standort bzw. der nächste korrekte
Standort klassifiziert wird mit einer Gesamttoleranz von fünf bzw. zehn Klassifizierungen.
\begin{flalign}
    \label{formular:accuracy_metrik2}
    &\epsilon \in \{5, 10\} \nonumber\\
    &L := \text{Menge von dem ML-Modell klassifizierten Standorte.} \nonumber\\
    &K := \text{Menge von den wirklichen Standorten.} \nonumber\\
    &\Phi(i) := \text{Index von dem nächsten Standort.} \nonumber\\
    &\Psi(i) := \text{Index von dem vorherigen Standort.} \nonumber\\
    &\Omega(i) := \Phi(i)-i\leq\epsilon\wedge\hspace{-0.3cm} \bigwedge\limits_{i\leq q \leq \min(\#K, \Phi(i))}\hspace{-0.3cm} L_q=K_{\Phi(i)} \nonumber\\
    &\Theta(i) := i-\Psi(i)\leq\epsilon\wedge\hspace{-0.3cm} \bigwedge\limits_{\max(0, \Psi(i))\leq q \leq i}\hspace{-0.3cm} L_q=K_{\Psi(i)} \nonumber\\
    &P(B) := \frac{\#\{L_i | L_i=K_i \vee \Omega(i) \vee \Theta(i)\text{ für } i\in\{0, 1, ..., \#L - 1\}\}}{\#K}
\end{flalign}
Zuletzt zwei Metriken bei denen die Klassifizierungsgenauigkeit bestimmt wird, unter der Bedingung, dass der vorherige
Standort korrekt (\ref{formular:accuracy_previous_was_correct}) bzw. falsch (\ref{formular:accuracy_previous_was_wrong}) war.
\begin{align}
    \label{formular:accuracy_previous_was_correct}
    P(C) := \frac{\text{Anzahl korrekter Klassifizierungen, wenn vorheriger Standort korrekt war}}{\text{Alle Klassifizierungen, wenn vorheriger Standort korrekt war}}
\end{align}
\begin{align}
    \label{formular:accuracy_previous_was_wrong}
    P(D) := \frac{\text{Anzahl korrekter Klassifizierungen, wenn vorheriger Standort falsch war}}{\text{Alle Klassifizierungen, wenn vorheriger Standort falsch war}}
\end{align}