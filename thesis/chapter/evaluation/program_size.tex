\section{Programmspeicher}
Der Großteil des Programmspeichers wird für das ML-Modell benötigt.
Aus diesem Grund wird der Anteil des Programmspeichers $X$ in der Evaluation vernachlässigt,
der für die restlichen Funktionen und für die Feature-Extrahierung benötigt wird.
Zudem ist der benötigte Programmspeicher dieses Anteils konstant und skaliert nicht mit der Größe, wie die ML-Modelle.
\newline
\newline
Zur Estimierung des Programmspeichers der Entscheidungsbäume wird der hybride Ansatz mit einer Toleranz von $\epsilon=0$ angenommen,
d. h. es werden für eindeutige Ergebnisse diskrete Rückgaben zurückgegeben, anstatt der Wahrscheinlichkeitsverteilung.
Als Datentyp für die Vergleiche und allen Features wird angenommen, dass ein vier Byte Datentyp verwendet wird.
Für einen Vergleich werden fünf Instruktionen benötigt \cite{dymelThesis}.
Für eine Rückgabe werden zwischen zwei und $2(N+1)$ Instruktionen und zwischen 0 und $2N$ Parameter benötigt,
wobei $N$ die Anzahl der möglichen Standorte ist.
Die Größe einer Instruktion ist 4 Byte, da eine 32-Bit CPU angenommen wird.
Die Größe des zusammenfassenden Klassifizierers $Z$ wird vernachlässigt.
\newline
\newline
Für die FFNNs wird ebenfalls ein vier Byte Datentyp für die Biase und die Gewichte angenommen.
Die Größe des Algorithmus zur Ausführung des KNN ist unbekannt und wird als konstanter Wert $Y$ angenommen,
liegt aber, den Zahlen in Gieses Arbeit nach zu urteilen, zwischen 6 und 7 KB \cite{gieseThesis}.
\newline
\newline
Tabelle \ref{tab:predictions_by_loc_size} zeigt Estimierungen des benötigten Programmspeichers der verschiedenen Konfigurationen der ML-Modelle,
wobei die konstanten Anteile vernachlässigt werden.
Potentielle Optimierungen, z. B. durch den Compiler, wurden dabei nicht betrachtet,
sowie potentielle Optimierungen des FFNN, wie Giese sie vorgeschlagen hat \cite{gieseThesis}.
Giese hat mit dem CSC-MA-Bit Format die Programmgröße um 39\% reduzieren können.
Kompilierung mit der Optimierungsstufe \textit{O2} konnte experimentell generierten C-Code
eines Entscheidungswaldes um bis zu 21,3\% reduzieren.
\newline
\newline
Der benötigte Programmspeicher beider ML-Modelle skaliert mit der Anzahl der zu klassifizierenden Standorte, der Anzahl der Schichten bzw. Bäume
und der Anzahl der Neuronen pro Schicht bzw. der maximalen Baumhöhe.
Der von den Entscheidungswäldern benötigte Programmspeicher ist für fast alle Fälle zu viel
für die Limitierungen eines kleinen eingebetteten Systems.
Die FFNNs hingegen benötigen deutlich weniger Programmspeicher und könnten innerhalb der Limitierungen der kleinen eingebetten Systeme passen.
\begin{table}[h!]
    \hspace{-1.25cm}
    \begin{tabular}{ | c | c | c | c | c | c | c | c | c | c | }
        \hline
        \multicolumn{2}{ | l |}{Programmgröße in KB über Standorte} & 9 & 16 & 17 & 25 & 32 & 48 & 52 & 102 \\\hline
        \multicolumn{10}{| l |}{\textbf{Entscheidungswälder}}\\\hline
        Waldgröße & Max. Baumgröße & \multicolumn{8}{ c |}{}\\\hline
        16 & 8 & 75.9 & 71.2 & 116.1 & 171.2 & 132.8 & 198.7 & 266.1 & 316.3 \\\hline
        16 & 16 & 190.0 & 114.5 & 279.7 & 517.6 & 297.0 & 750.5 & 909.2 & 1063.2 \\\hline
        16 & 32 & 179.9 & 132.5 & 302.1 & 568.3 & 293.4 & 838.9 & 981.4 & 1147.8 \\\hline
        16 & 64 & 179.9 & 132.5 & 302.1 & 568.3 & 293.4 & 838.9 & 981.4 & 1147.8 \\\hline
        8 & 32 & 89.2 & 68.4 & 155.8 & 284.7 & 144.4 & 388.5 & 485.1 & 578.4 \\\hline
        16 & 32 & 179.9 & 132.5 & 302.1 & 568.3 & 293.4 & 838.9 & 981.4 & 1147.8 \\\hline
        32 & 32 & 374.7 & 250.7 & 578.8 & 1086.3 & 576.1 & 1659.7 & 2070.8 & 2294.5 \\\hline
        64 & 32 & 738.6 & 522.8 & 1181.0 & 2130.3 & 1141.4 & 3234.6 & 4127.5 & 4673.7 \\\hline
        32 & 64 & 374.7 & 250.7 & 578.8 & 1086.3 & 576.1 & 1659.7 & 2070.8 & 2294.5 \\\hline
        \multicolumn{10}{| l |}{\textbf{Feed Forward neuronale Netzwerke}}\\\hline
        \#Schichten & \#Neuronen & \multicolumn{8}{ c |}{}\\\hline
        1 & 16 & 2.8 & 3.2 & 3.3 & 3.8 & 4.2 & 5.2 & 5.5 & 8.7 \\\hline
        1 & 32 & 5.5 & 6.4 & 6.5 & 7.6 & 8.4 & 10.5 & 11.0 & 17.4 \\\hline
        1 & 64 & 11.0 & 12.8 & 13.1 & 15.1 & 16.9 & 21.0 & 22.0 & 34.8 \\\hline
        1 & 128 & 22.0 & 25.6 & 26.1 & 30.2 & 33.8 & 42.0 & 44.0 & 69.6 \\\hline
        2 & 32 & 9.6 & 10.5 & 10.6 & 11.6 & 12.5 & 14.6 & 15.1 & 21.5 \\\hline
        4 & 32 & 17.8 & 18.7 & 18.8 & 19.8 & 20.7 & 22.8 & 23.3 & 29.7 \\\hline
        8 & 32 & 34.2 & 35.1 & 35.2 & 36.2 & 37.1 & 39.2 & 39.7 & 46.1 \\\hline
        4 & 64 & 60.2 & 62.0 & 62.2 & 64.3 & 66.0 & 70.1 & 71.2 & 84.0 \\\hline
    \end{tabular}
    \caption{Größe in KB über Standorte und verschiedenen Konfigurationen der ML-Modelle zur Standorterkennung.}
    \label{tab:predictions_by_loc_size}
\end{table}