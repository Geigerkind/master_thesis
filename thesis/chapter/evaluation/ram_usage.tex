\newpage
\section{RAM}
Für den benötigten RAM muss neben dem Anteil der ML-Modelle, die Historie der Sensorwerte und die berechneten Features betrachtet werden.
Wichtig ist dabei der meiste RAM, der zu einem Zeitpunkt benötigt werden kann.
\newline
\newline
Für jeden Sensorwert, bis auf der Detektion von WLAN-Zugangspunkten, wird ein vier Byte Datentyp angenommen.
Für die Detektion der Zugangspunkte wird ein ein Byte Datentyp angenommen.
Insgesamt beträgt der benötigte RAM für einen Vektor von Sensorwerten damit 61 Byte.
\newline
\newline
Dies setzt sich zusammen aus dem Zeitstempel, der xyz-Komponente vom Accelerometer und Gyroskop, dem Lichtsensor,
dem Temperatursensor, dem Magnetfeldsensor, dem Geräuschsensor und fünf möglichen WLAN-Zugangspunkten.
Bei einem Datenfenster von drei Einträgen wird damit 183 Byte für Sensorwerte benötigt.
\newline
\newline
Der Anteil der Features ist abhängig von den Features, die für ein bestimmtes Szenario eingesetzt werden.
Insgesamt werden 38 Features verwendet, die vereinfacht alle als vier Byte Datentyp angenommen werden.
Zur Evaluierung des ML-Modells wird nur die aktuelle Feature-Menge benötigt.
Damit wird für die Feature-Menge insgesamt 152 Byte benötigt, wenn alle Features verwendet werden.
\newline
\newline
Zur Ausführung eines Entscheidungswaldes wird für die Rückgabe der Wahrscheinlichkeitsverteilung für jeden Standort vier Byte benötigt.
Je nach Implementierung würde dieser Vektor mehrmals benötigt werden, z. B. bei der parallelen Evaluierung der Entscheidungsbäume skaliert dies mit der Anzahl der Prozessoren.
In diesem Fall wird keine Nebenläufigkeit angenommen.
In dieser Arbeit wurden zwischen 9 Standorte und 102 Standorte untersucht, d. h. es wurden zwischen 36 Byte und 408 Byte benötigt.
Die Standortkomplexität ist jedoch abhängig vom Einsatzszenario.
Die anschließende Evaluierung eines Entscheidungswaldes zur Anomalieerkennung kann vernachlässigt werden,
da dieser ein diskretes Ergebnis zurückgeben kann und die benötigte Feature-Menge deutlich kleiner ist.
Damit wird für $N$ Standorte und $K$ Features mit einem Entscheidungswald als ML-Modell zu einem Zeitpunkt
ca. $183 + 4(N + K)$ Byte benötigt, d. h. bei 102 Standorten und 38 Features ca. 743 Byte.
\newline
\newline
Zur Ausführung eines FFNNs werden nur wenige Byte benötigt, um die nötigen Multiplikationen eines Neurons durchzuführen.
Dies würde die Ausführungszeit, und den Energiebedarf, aber signifikant erhöhen, da die benötigten Gewichte ständig aus dem Programmspeicher geladen werden müssten.
Das heißt, es müssen mindestens die Zwischenergebnisse einer Schicht im RAM gehalten werden, sowie ein Gewicht und ein Bias.
\newline
\newline
Damit benötigt ein FFNN, dessen größte Schicht $M$ Neuronen hat, mindestens $4(M+2)$ Byte.
Maximal wird $4M$ Byte, zuzüglich der Größe aller Gewichte und Biase benötigt.
Der maximale RAM, der zu einem Zeitpunkt, mit einem FFNN, benötigt wird, beträgt damit mindestens $183 + 4(M + K)$ Byte.