\section{Ausführungszeit und benötigte Energie}
Es ist sehr problemetisch eine sinvolle Estimierung für die benötigte Ausführungszeit und Energie anzugeben.
Ausführungszeit und die benötigte Energie sind abhängig von dem verwendeten Mikrocontroller.
Vergleichbare 32-Bit Mikrocontroller mit FPU (Floating Point Unit), zu den Microcontrollern die Dymel verwendet hat \cite{dymelThesis}, sind aus der AVR C Serie \cite{avr32BitDatasheet}.
Leider ist deren Datenblätter keine Information über die Ausführungszeit von Gleitkommazahlinstruktionen und kein Energiemodell zu entnehmen.
Es ist aber anzunehmen, dass deutlich weniger Cyclen benötigt werden für hardwareunterstützte Gleitkommazahloperationen, als Software basierte Alternativen.
Aus diesem Grund wird die Ausführungszeit in Gleitkommazahl- Vergleichen, Multiplikationen, Division, Additionen und Wurzel angegeben,
da diese die integralen Bestandteile der Feature-Extrahierung und Evaluation der ML-Modelle sind.
\newline
\newline
Hier werden die Konfigurationen (TODO) betrachtet, da diese die beste Klassifizierungsgenauigkeit erzielt in Relation zu ihrer Größe erzielt haben.
Die in dieser Arbeit vorgeschlagene Architektur (\ref{fig:model_idea}) hat fünf Bestandteile, die jeweils zur Gesamtausführungszeit beitragen.
Die Aufnahme der Sensorwerte wird als konstanter Energieverbrauch angenommen und in dieser Rechnung vernachlässigt.
In der ersten Feature-Extrahierung werden 34 Features aus dem Datenfenster extrahiert.
Tabelle \ref{tab:feature_operation_complexity} zeigt die estimierte Anzahl der Operationen, die pro Art des Features benötigt werden.
\begin{table}[h!]
    \centering
    \begin{tabular}{ | l | c | c | c | c | c | }
        \hline
        Art des Features & Vergleich & Multiplikation & Division & Addition & Wurzel \\\hline
        Standardabweichung (\textbf{7}) & 0 & 3 & 2 & 7 & 1 \\\hline
        Minimum (\textbf{6}) / Maximum (\textbf{6}) & 2 & 0 & 0 & 0 & 0 \\\hline
        Durchschnitt (\textbf{6}) & 0 & 0 & 1 & 2 & 0 \\\hline
        Wert (\textbf{9}) & 0 & 0 & 0 & 0 & 0 \\\hline
    \end{tabular}
    \caption{Estimierte Anzahl der Operationen pro Art des Features bei einer Datenfenstergröße von 3. Fettgedruckte Zahl zeigt die Verwendungsanazahl in der Feature-Menge an.}
    \label{tab:feature_operation_complexity}
\end{table}
\newline
\newline
Tabelle (TODO) zeigt die estimierte Anzahl der Operationen, die jeweils für den besten Entscheidungswäldern und FFNNs zur Standorterkennung benötigt werden.
Die Anzahl der Vergleichsoperationen und Multiplikationen hängen nicht mit der Anzahl der Standorte zusammen,
jedoch haben sich verschieden komplexe Strukturen für diese Fälle als Beste erwiesen.
\newline
\newline
Bei der Feature-Extrahierung für das ML-Modell zur Anomalieerkennung werden nur vier Features extrahiert.
Tabelle \ref{tab:anomaly_feature_operation_complexity} zeigt die estimierte Anzahl der Operationen, die für die einzelnen Features benötigt werden.
Der Entscheidungswald zur Anomalieerkennung benötigt (TODO) Vergleiche und das FFNN zur Anomalieerkennung benötigt (TODO) Multiplikationen.
\begin{table}[h!]
    \centering
    \begin{tabular}{ | p{4.5cm} | c | c | c | c | c | }
        \hline
        Feature & Vergleich & Multiplikation & Division & Addition & Wurzel \\\hline
        Abweichung zum ØStandortänderungen & 4 & 0 & 2 & 5 & 0 \\\hline
        Abweichung zum ØKlassifizierungswahrscheinlichkeit & 4 & 0 & 2 & 5 & 0 \\\hline
        Topologieverletzung & 5 & 1 & 0 & 1 & 0 \\\hline
        Standardabweichung Top 5 Klassifizierungen & 0 & 5 & 2 & 13 & 1 \\\hline
    \end{tabular}
    \caption{Estimierte Anzahl der Operationen pro Feature der Anomalieerkennung.}
    \label{tab:anomaly_feature_operation_complexity}
\end{table}
\newline
\newline
Tabelle (TODO) fasst die Anzahl der Operationen für eine Konfiguration mit ausschließlich Entscheidungswäldern und FFNNs zusammen.
Es ist zu erwarten, dass die Entscheidungswälder weniger Ausführungszeit benötigen als die FFNNs, da deutlich weniger Operationen benötigt werden,
wodurch sich ein Entscheidungsbaum basierter Klasifizierer in Hinsicht auf die benötigte Energie besser eignet.
Bei konstanter Bewegung wurde in den Testszenarien der Mikrocontroller alle (TODO)s aufgeweckt.
Dies ist aber unrealistisch, da die Sensorenbox auch für lange Zeit an einem Ort verbleiben kann, was sich positiv auf den Energiebedarf auswirkt.