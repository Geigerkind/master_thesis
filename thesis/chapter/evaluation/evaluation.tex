\chapter{Evaluation}
TODO

\section{Klassifizierungsgenauigkeit}
\begin{itemize}
    \item Metriken
    \item Wie groß ist die Wahrscheinlichkeit, dass die Orte korrekt erkannt werden?
    \item Entscheidungsbaum vs KNN
    \item Skalierung mit Anzahl der Orte
    \item Signifikanz der Features
    \item Einfluss von einzelnen Features für die Klassifizierungsgenauigkeit(?)
    \item Ist es sinnvoll für jeden Ort ein Feature zu haben, dass auf eins gesetzt wird, wenn der Ort erkannt wurde und ansonsten exponentiell abfällt. Wie schnell sollte es fallen, wenn ja?
\end{itemize}

\section{Fehlertoleranz}
\begin{itemize}
    \item Metriken => Insbesondere continued\_predict hier erwähnen
    \item Wenn falscher Ort erkannt wurde, wie lange dauert es um wieder den korrekten Ort zu finden?
    \item Was passiert wenn Sensoren ausfallen?
    \item Transportbox nicht dem trainierten Pfad folgt?
    \item Änderungen der Fabrik, e.g. Licht, Wärme, Magnet, Schnelligkeit der Fließbänder
    \item Einfluss von einzelnen Features für die Fehlertoleranz(?)
\end{itemize}

\section{Ressourcennutzung}
\begin{itemize}
    \item Metriken(?)
    \item Entscheidungsbaum Ausführung
    \item FFNN Ausführung
    \item Feature extrahierung
    \begin{itemize}
        \item Verhältnis von Kosten zu Nutzen
    \end{itemize}
    \item Daten Sammlung => Interrupts (Wakeups)
    \item Einfluss von einzelnen Features für den Ressourcenverbrauch(?)
\end{itemize}
