\chapter{Evaluation}
In der Evaluation werden drei Aspekte betrachtet: Klassifizierungsgenauigkeit, Robustheit und Resourcennutzung.
Bei der Klassifizierungsgenauigkeit wird einerseits die Standortbestimmung und andererseits die Anomalieerkennung evaluiert.
Dabei wird sowohl auf verschiedene Größen von FFNN und Entscheidungswälder eingegangen,
als auch auf verschieden viele zu unterscheidende Standorte.
\newline
\newline
Bei der Robustheit wird auf den Fehler des besten FFNN und Entscheidungswald bei verschiedene Fehlerszenarien eingegangen.
Diese bestehen aus fehlerhaften Sensordaten durch Rauschen oder ausgefallenen Sensoren und Routen mit permutierten Teilstücken.
\newline
\newline
Bei der Resourcennutzung wird auf die Programmgröße und die Ausführungszeit der besten ML-Modelle eingegangen.
Außerdem wird der Energieverbrauch für verschiedene Szenarien eingeschätzt.
\newline
\newline
Unterschieden werden zwei Varianten der Testmengen.
Diese unterscheiden sich in der Art, wie die Features für den vorherigen Standort bestimmt werden.
In der ersten Variante sind alle vorherigen Standorte korrekt.
In der zweiten Variante ist der erste vorherige Standort als \textit{unbekannt} beschriftet und alle folgenden vorherigen Standorte werden iterativ durch das ML-Modell bestimmt,
d.~h. in dieser Variante wird der propagierte Fehler durch die Rückwärtskante des ML-Modells betrachtet.
Metriken, die auf die zweite Variante der Testmenge angewendet werden sind mit \texttt{cont} markiert.
\newline
\newline
Die ML-Modelle wurden mit Trainingsdaten von den vier aufgenommen Routen trainiert.
Sie teilen ein relatives Koordinatensystem, in dem die künstlichen Interferenzquellen für die ergänzten Sensoren liegen.
Das heißt, ein Training mit mehreren Routen kann als eine sehr komplexe Route mit überlappenden Pfaden betrachtet werden,
da diese Routen technisch gesehen übereinander liegen.

\section{Metriken}
In dieser Arbeit werden verschiedene Metriken zur Ermittlung der Klassifizierungsgenauigkeit ermittelt.
Zunächst die übliche Klassifizierungsgenauigkeit (\ref{formular:simple_accuracy}), in der die Anzahl der korrekt klassifizierten Standorte mit der Gesamtanzahl verglichen werden.
\begin{align}
    \label{formular:simple_accuracy}
    P(A) := \frac{\text{Anzahl korrekter Klassifizierungen}}{\text{Gesamtanzahl}}
\end{align}
Die zweite Metrik (\ref{formular:accuracy_metrik2}) betrachtet die Klassifizierungsgenauigkeit unter Tolerierung, dass ein Standort
fünf bzw. zehn Klassifizierungen kontinuierlich zu früh oder zu spät verlassen wurde,
d.~h. Fehlklassifizierungen werden vernachlässigt, wenn kontinuierlich der letzte korrekte Standort bzw. der nächste korrekte
Standort klassifiziert wird mit einer Gesamttoleranz von fünf bzw. zehn Klassifizierungen.
\begin{flalign}
    \label{formular:accuracy_metrik2}
    &\epsilon \in \{5, 10\} \nonumber\\
    &L := \text{Menge von dem ML-Modell klassifizierten Standorte.} \nonumber\\
    &K := \text{Menge von den wirklichen Standorten.} \nonumber\\
    &\Phi(i) := \text{Index von dem nächsten Standort.} \nonumber\\
    &\Psi(i) := \text{Index von dem vorherigen Standort.} \nonumber\\
    &\Omega(i) := \Phi(i)-i\leq\epsilon\wedge\hspace{-0.3cm} \bigwedge\limits_{i\leq q \leq \min(\#K, \Phi(i))}\hspace{-0.3cm} L_q=K_{\Phi(i)} \nonumber\\
    &\Theta(i) := i-\Psi(i)\leq\epsilon\wedge\hspace{-0.3cm} \bigwedge\limits_{\max(0, \Psi(i))\leq q \leq i}\hspace{-0.3cm} L_q=K_{\Psi(i)} \nonumber\\
    &P(B) := \frac{\#\{L_i | L_i=K_i \vee \Omega(i) \vee \Theta(i)\text{ für } i\in\{0, 1, ..., \#L - 1\}\}}{\#K}
\end{flalign}
Die Testmenge existiert in zwei Varianten, die sich durch die Bestimmung des vorherigen Standortes in den beinhalteten Feature-Mengen unterscheidet.
In der ersten Variante sind alle vorherigen Standorte korrekt.
In der zweiten Variante ist der erste vorherige Standort 0 und alle folgenden vorherigen Standorte werden iterativ durch das ML-Modell bestimmt,
d.~h. in dieser Variante wird der propagierte Fehler durch die Rückwärtskante des ML-Modells betrachtet.
\newline
\newline
Für die zweite Variante sind zwei weitere Metriken relevant.
Bei diesen Metriken wird die Klassifizierungsgenauigkeit bestimmt, unter der Bedingung, dass der vorherige
Standort korrekt (\ref{formular:accuracy_previous_was_correct}) bzw. falsch (\ref{formular:accuracy_previous_was_wrong}) war.
\begin{align}
    \label{formular:accuracy_previous_was_correct}
    P(C) := \frac{\text{Anzahl korrekter Klassifizierungen, wenn vorheriger Standort korrekt war}}{\text{Alle Klassifizierungen, wenn vorheriger Standort korrekt war}}
\end{align}
\begin{align}
    \label{formular:accuracy_previous_was_wrong}
    P(D) := \frac{\text{Anzahl korrekter Klassifizierungen, wenn vorheriger Standort falsch war}}{\text{Alle Klassifizierungen, wenn vorheriger Standort falsch war}}
\end{align}
\section{Klassifizierungsgenauigkeit}
\begin{itemize}
    \item Metriken
    \item Wie groß ist die Wahrscheinlichkeit, dass die Orte korrekt erkannt werden?
    \item Entscheidungsbaum vs KNN
    \item Skalierung mit Anzahl der Orte
    \item Signifikanz der Features
    \item Einfluss von einzelnen Features für die Klassifizierungsgenauigkeit(?) => Last Distinct location möglicherweise schlecht, da die eigentliche Last Distinct Location oft durch fehlende Interrupts übersprungen wird
    \item Welche Feature werden genutzt? => Abhängig von Feature Importance und wie günstig zu berechnen
    \item Ist es sinnvoll für jeden Ort ein Feature zu haben, dass auf eins gesetzt wird, wenn der Ort erkannt wurde und ansonsten exponentiell abfällt. Wie schnell sollte es fallen, wenn ja?
    \item Werden mehr Trainingsdaten benötigt mit steigender Ort Anzahl? Wenn ja wie viel? (Hier oder bei ML-Modell Training)
    \item Irgendwo drüber reden wie wir evaluieren, also continued predict vs predict und die ganzen Testsets
    \item Wie viele Trainingsdaten werden benötigt?
    \begin{itemize}
        \item Um KNN zu trainieren?
        \item Um Entscheidungsbaum zu trainieren?
        \item Ggf. Unterschiede klären
        \item (Gehört das schon in eine Evaluation, oder ist das hier okay?)
    \end{itemize}
\end{itemize}
\section{Klassifizierungsgenauigkeit der Anomalien}
\label{sec:eval_anomalieerkennung}
Bei der Anomalieerkennung werden Entscheidungswälder und FFNNs mit den besten ML-Modellen zur Standorterkennung trainiert und mit den drei Baseline-Modellen verglichen.
Tabelle \ref{tab:anomaly_detection_prediction_accuracy} zeigt die Klassifizierungsgenauigkeiten über die verschiedenen Standortkomplexitäten,
wobei die Klassifizierungsgenauigkeit $P(A)$ nochmal genauer aufgeschlüsselt ist in den Anteil der korrekten Klassifizierungen, wenn eine bzw. keine Anomalie vorlag.
Die trainierten FFNNs geben stets aus, dass keine Anomalie vorliegt.
Es ist unklar, warum die FFNNs sich so verhalten.
\begin{table}[h!]
    \hspace{-1cm}
    \begin{tabular}{ | l | c | c | c | c | c | c | c | c | }
        \hline
        Standorte & 9 & 16 & 17 & 25 & 32 & 48 & 52 & 102 \\\hline
        \multicolumn{9}{ | l |}{$P(A)$}\\\hline
        Entscheidungswald & 82,59\% & 81,19\% & 87,14\% & 84,91\% & 79,06\% & 83,47\% & 81,93\% & 76,00\% \\\hline
        FFNN & 77,88\% & 77,88\% & 77,88\% & 77,88\% & 77,88\% & 77,88\% & 77,88\% & 77,88\% \\\hline
        Topologie (DT) & 84,77\% & 30,57\% & 83,51\% & 79,76\% & 28,63\% & 24,97\% & 80,55\% & 29,47\% \\\hline
        Topologie (KNN) & 86,10\% & 52,17\% & 77,72\% & 79,30\% & 45,06\% & 41,92\% & 74,77\% & 43,55\% \\\hline
        \multicolumn{9}{ | l |}{Anteil korrekt klassifiziert, indem Anomalie vorlag}\\\hline
        Entscheidungswald & 34,86\% & 35,52\% & 52,58\% & 50,92\% & 32,21\% & 50,64\% & 23,21\% & 1,92\% \\\hline
        FFNN & 0,00\% & 0,00\% & 0,00\% & 0,00\% & 0,00\% & 0,00\% & 0,00\% & 0,00\% \\\hline
        \multicolumn{9}{ | l |}{Anteil korrekt klassifiziert, indem keine Anomalie vorlag}\\\hline
        Entscheidungswald & 96,14\% & 94,41\% & 97,05\% & 95,83\% & 92,48\% & 93,13\% & 98,96\% & 97,18\% \\\hline
        FFNN & 100,00\% & 100,00\% & 100,00\% & 100,00\% & 100,00\% & 100,00\% & 100,00\% & 100,00\% \\\hline
    \end{tabular}
    \caption{Metrik $P(A)$ über Standorte und verschiedenen Konfigurationen der Modelle zur Anomalieerkennung.}
    \label{tab:anomaly_detection_prediction_accuracy}
\end{table}
\newline
\newline
Die Entscheidungswälder hingegen eignen sich besser für den Anomalieerkennungszweck.
Es werden zwischen 1,92\% und 52,58\% der Anomalien erkannt und zwischen 1,04\% und 7,52\% falsch als Anomalien erkannt.
Die Klassifizierungsgenauigkeit des Entscheidungswaldes zur Anomalieerkennung ist abhängig von der Klassifizierungsgenauigkeit zur Standorterkennung
und von der Standortkomplexität.
Je besser das Standorterkennungsmodell und je höher die Standortkomplexität, desto höher ist die Anomalieerkennungsrate.
Aus diesem Grund ist die Klassifizierungsgenauigkeit bei den Standortkomplexitäten, die mit der Kodierungsmethode mit Kanten und Knoten zusammenhängen,
geringer, als bei der Kodierungsmethode, wo nur die Knoten kodiert werden.



\begin{itemize}
    \item Was ist das?
    \item Zusammenhang zu Enkodierungsansatz
    \item Schwierig zu trainieren, da man einerseits Robust sein will und andererseits nicht weiß was trainiert werden soll
    \item Post Processing
    \item Metriken: Location Change Frequency, Accumulated Confididence, Fraction Zero
    \item Sag Motivation, warum diese Metriken (Indikatoren)
    \item Beispiele
    \item Wie zuverlässig können Anomalien erkannt werden?
    \item Vergleich mit Coin-Toss, True und False
    \item Vergleiche Enkodierungsansätze, ob einer besser als der andere ist
\end{itemize}
\newpage
\section{Signifikanz der Features}
Die Signifikanz der Features wird über die Permutationswichtigkeit bestimmt.
Die Permutationswichtigkeit ist der Fehler, der durch das permutieren eines Features in der Testmenge entsteht, im Vergleich zu der ursprünglichen Testmenge.
Je größer der Fehler, desto wichtiger das Feature.
\newline
\newline
Abbildungen \ref{fig:feature_significance_dt} und \ref{fig:feature_significance_knn} zeigen die Permutationswichtigkeit eines Entscheidungswaldes und FFNN.
Die Permutationswichtigkeit von einzelnen Entscheidungswäldern bzw. FFNNs unterscheidet sich nicht stark.
Beide ML-Modelle weisen den vorherigen Standorten eine hohe Signifikanz zu.
Die Klassifizierunggenauigkeiten in Tabelle \ref{tab:predictions_by_acc_pic_cont} bestätigen diese Abhängigkeit.
Demnach ist die Wahrscheinlichkeit sehr gering, dass der Standort korrekt klassifiziert wird, wenn der vorherige Standort inkorrekt war.
\begin{figure}[h!]
    \centering
    \includegraphics[width=\linewidth]{images/evaluation_feature_importance_dt_pi.png}
    \caption{Permutationswichtigkeit der Features eines Entscheidungswaldes.}
    \label{fig:feature_significance_dt}
\end{figure}
\begin{figure}[h!]
    \centering
    \includegraphics[width=\linewidth]{images/evaluation_feature_importance_knn_pi.png}
    \caption{Permutationswichtigkeit der Features eines FFNN.}
    \label{fig:feature_significance_knn}
\end{figure}
\newline
\newline
Für die Entscheidungswälder sind alle Features, bis auf die vorherige Position und die Standardabweichung von der Ausrichtung zum Magnetfeld unwichtig.
Die FFNNs hingegen bedienen sich außerdem Features von dem Temperatursensor, dem Lichtsensor und vor allem der Detektierung von WLAN-Zugangspunkten.
Allerdings sind FFNNs deutlich abhängiger von dem vorherigen Standort als Entscheidungswälder.
\newline
\newline
Durch diese Abhängigkeit ist keine hohe Robustheit der ML-Modelle zu erwarten.
Diese Abhängigkeit ließe sich Eliminieren, indem ohne die Rückwärtskante trainiert wird.
Dies simplifiziert den Trainingsprozess, wodurch die ML-Modelle schneller zu trainieren sind.
Tabelle \ref{tab:predictions_wo_feedback_edge_by_acc} zeigt, dass diese ML-Modelle vergleichbare Klassifizierungsgenauigkeiten erzielen.
Insbesondere FFNNs erzielen deutlich bessere Klassfizierungsgenauigkeiten, erzielen aber dennoch schlechtere Ergebnisse als die Entscheidungswälder.
Hier ist die Metrik $P(A)$ mit der Metrik $P(A)_{\text{cont}}$ vergleichbar, da der propagierte Fehler durch die Eliminierung der Rückwärtskante nicht mehr existiert.
\newline
\newline
Abbildungen \ref{fig:feature_significance_dt_wo_fe} und \ref{fig:feature_significance_knn_wo_fe} zeigen die Permutationswichtigkeit der ML-Modelle ohne Rückwärtskante.
Beide ML-Modelle gewichten Features aus anderen Sensorwerten deutlich mehr, im Vergleich zu den ML-Modellen mit Rückwärtskante.
Die Entscheidungswälder gewichten dennoch nur die Standardabweichung der Ausrichtung zum Magnetfeld, die Detektierung der WLAN-Zugangspunkte und das Minimum des Lichtsensors stark.
Die anderen Features sind im Vergleich deutlich unwichtiger.
Das FFNN hingegen nutzt alle Features, wobei Features aus dem Accelerometer und Gyroskop unwichtiger sind.
\begin{figure}[h!]
    \centering
    \includegraphics[width=\linewidth]{images/fi_wo_fe_dt.png}
    \caption{Permutationswichtigkeit der Features eines Entscheidungswaldes ohne Rückwärtskante.}
    \label{fig:feature_significance_dt_wo_fe}
\end{figure}
\begin{figure}[h!]
    \centering
    \includegraphics[width=\linewidth]{images/fi_wo_fe_knn.png}
    \caption{Permutationswichtigkeit der Features eines FFNN ohne Rückwärtskante.}
    \label{fig:feature_significance_knn_wo_fe}
\end{figure}
\newline
\newline
Abbildung \ref{fig:feature_significance_dt_anomaly} zeigt die Permutationswichtigkeit von einem Entscheidungswald zur Anomalieerkennung.
Am wichtigsten sind die Features, die die Wahrscheinlichkeitsverteilung des Klassifizierungsergebnisses von dem ML-Modell zur Standorterkennung ausnutzen.
Die Abweichung zu den durchschnittlichen Standortänderungen hat kaum Einfluss.
Dies deutet darauf hin, dass die ML-Modelle zur Standorterkennung nicht so instabil in Anomalien sind, wie angenommen.
Aus dem selben Grund hat das Feature der Topologieverletzung auch wenig Einfluss, da es bei Anomalien wenig ausschlägt.
\begin{figure}[h!]
    \centering
    \includegraphics[width=0.65\linewidth]{images/fi_anomaly_dt.png}
    \caption{Permutationswichtigkeit der Features eines Entscheidungswaldes zur Anomalieerkennung.}
    \label{fig:feature_significance_dt_anomaly}
\end{figure}
\newpage
Die Wichitgkeit der Features variiert mit dem Szenario, indem die ML-Modelle eingesetzt werden.
Aus diesem Grund sollte die Analyse der Wichtigkeit der Features individuell für jedes Einsatzszenario durchgeführt werden,
womit durch eine sorgfältige Feature-Selektion die Klassifizierungsgenauigkeit und der Ressourcenverbrauch maximiert werden kann.
\section{Fehlertoleranz}
\begin{itemize}
    \item Irgendwas über Robustheit
    \item Metriken => Insbesondere continued\_predict hier erwähnen
    \item Wenn falscher Ort erkannt wurde, wie lange dauert es um wieder den korrekten Ort zu finden? Was ist die Wahrscheinlichkeit dafür? Wovon hängt es ab, dass wieder der richtige Ort erkannt wurde?
    \item Was passiert wenn Sensoren ausfallen? Wie robust ist es dagegen?
    \item Transportbox nicht dem trainierten Pfad folgt? Wie Robust ist sie dagegen? => Sezenarien enumiereren, wo sowas passieren kann, dann Testszenario erklären.
    \item Änderungen der Fabrik, e.g. Licht, Wärme, Magnet, Schnelligkeit der Fließbänder
    \item Einfluss von einzelnen Features für die Fehlertoleranz(?)
    \item In welchen Fällen ist die Ortung unzuverlässig? => Welche Fälle sind schwierig?
    \item Schlussfolgerung: Wie Fehlertolerant ist die Ortung? => Zusammenfassende Metriken über alle Testsets?
\end{itemize}
\section{Benötigte Anzahl der Trainingsdaten}
\begin{itemize}
    \item Werden mehr Trainingsdaten benötigt mit steigender Ort Anzahl? Wenn ja wie viel? (Hier oder bei ML-Modell Training)
    \item Wie viele Trainingsdaten werden benötigt?
    \begin{itemize}
        \item Um KNN zu trainieren?
        \item Um Entscheidungsbaum zu trainieren?
        \item Ggf. Unterschiede klären
        \item (Gehört das schon in eine Evaluation, oder ist das hier okay?)
    \end{itemize}
\end{itemize}
\section{Programmspeicher}
Der Großteil des Programmspeichers wird für das ML-Modell benötigt.
Aus diesem Grund wird der Anteil des Programmspeichers in der Evaluation vernachlässigt,
der für die restlichen Funktionen und für die Feature-Extrahierung benötigt wird.
Zudem ist der benötigte Programmspeicher dieses Anteils konstant und skaliert nicht mit der Größe, wie die ML-Modelle.
\newpage
Zur Einschätzung des Programmspeichers der Entscheidungsbäume wird der hybride Ansatz mit einer Toleranz von $\epsilon=0$ angenommen,
d. h. es werden für eindeutige Ergebnisse diskrete Rückgaben zurückgegeben, anstatt der Wahrscheinlichkeitsverteilung.
Als Datentyp für die Vergleiche und allen Features wird angenommen, dass ein vier Byte Datentyp verwendet wird.
Für einen Vergleich werden fünf Instruktionen benötigt \cite{dymelThesis}.
Für eine Rückgabe werden zwischen zwei und $2(N+1)$ Instruktionen und zwischen 0 und $2N$ Parameter benötigt,
wobei $N$ die Anzahl der möglichen Standorte ist.
Die Größe einer Instruktion ist vier Byte, da eine 32-Bit CPU angenommen wird.
Die Größe des Algorithmus des zusammenfassenden Klassifizierers wird vernachlässigt.
\newline
\newline
Für die FFNNs wird ebenfalls ein vier Byte Datentyp für die Biase und die Gewichte angenommen.
Die Größe des Algorithmus zur Ausführung des FFNNs ist unbekannt und wird als konstanter Wert angenommen,
liegt aber, den Zahlen in Gieses Arbeit nach zu urteilen, zwischen 6 KB und 7 KB \cite{gieseThesis}.
Diese Messungen beziehen sich aber auf einen 8-Bit Mikrocontroller, weshalb dieser Anteil für ein 32-Bit Mikrocontroller möglicherweise größer ist.
\newline
\newline
Tabelle \ref{tab:predictions_by_loc_size} zeigt Einschätzungen des benötigten Programmspeichers der verschiedenen Konfigurationen der ML-Modelle,
wobei die konstanten Anteile vernachlässigt werden.
Potentielle Optimierungen der Entscheidungswälder, z. B. durch den Compiler,
sowie potentielle Optimierungen des FFNNs, wie Giese in \cite{gieseThesis} vorgeschlagen hat, wurden dabei nicht betrachtet.
Giese hat mit dem CSC-MA-Bit Format die Programmgröße um 39\% reduzieren können.
Kompilierung mit der Optimierungsstufe \textit{O2} konnte experimentell generierten C-Code
eines Entscheidungswaldes um bis zu 21,3\% reduzieren.
\begin{table}[h!]
    \hspace{-2cm}
    \begin{tabular}{ | c | c | c | c | c | c | c | c | c | c | }
        \hline
        \multicolumn{2}{ | l |}{Programmgröße in KB über Standorte} & 9 & 16 & 17 & 25 & 32 & 48 & 52 & 102 \\\hline
        \multicolumn{10}{| l |}{\textbf{Entscheidungswälder}}\\\hline
        Waldgröße & Max. Baumgröße & \multicolumn{8}{ c |}{}\\\hline
        16 & 8 & 72.2 & 71.2 & 119.8 & 157.5 & 132.8 & 184.3 & 237.9 & 316.3 \\\hline
        16 & 16 & 158.2 & 114.5 & 264.9 & 465.0 & 297.0 & 573.7 & 724.7 & 1063.2 \\\hline
        16 & 32 & 158.8 & 132.5 & 277.8 & 472.2 & 293.4 & 625.9 & 771.2 & 1147.8 \\\hline
        16 & 64 & 158.8 & 132.5 & 277.8 & 472.2 & 293.4 & 625.9 & 771.2 & 1147.8 \\\hline
        8 & 32 & 72.2 & 68.4 & 135.4 & 242.1 & 144.4 & 317.2 & 415.2 & 578.4 \\\hline
        32 & 32 & 325.9 & 250.7 & 550.8 & 951.2 & 576.1 & 1151.9 & 1676.2 & 2294.5 \\\hline
        64 & 32 & 669.2 & 522.8 & 1107.6 & 1906.5 & 1141.4 & 2440.0 & 3244.9 & 4673.7 \\\hline
        32 & 64 & 325.9 & 250.7 & 550.8 & 951.2 & 576.1 & 1151.9 & 1676.2 & 2294.5 \\\hline
        \multicolumn{10}{| l |}{\textbf{Feed Forward neuronale Netzwerke}}\\\hline
        \#Schichten & \#Neuronen & \multicolumn{8}{ c |}{}\\\hline
        1 & 16 & 2.8 & 3.2 & 3.3 & 3.8 & 4.2 & 5.2 & 5.5 & 8.7 \\\hline
        1 & 32 & 5.5 & 6.4 & 6.5 & 7.6 & 8.4 & 10.5 & 11.0 & 17.4 \\\hline
        1 & 64 & 11.0 & 12.8 & 13.1 & 15.1 & 16.9 & 21.0 & 22.0 & 34.8 \\\hline
        1 & 128 & 22.0 & 25.6 & 26.1 & 30.2 & 33.8 & 42.0 & 44.0 & 69.6 \\\hline
        2 & 32 & 9.6 & 10.5 & 10.6 & 11.6 & 12.5 & 14.6 & 15.1 & 21.5 \\\hline
        4 & 32 & 17.8 & 18.7 & 18.8 & 19.8 & 20.7 & 22.8 & 23.3 & 29.7 \\\hline
        8 & 32 & 34.2 & 35.1 & 35.2 & 36.2 & 37.1 & 39.2 & 39.7 & 46.1 \\\hline
        4 & 64 & 60.2 & 62.0 & 62.2 & 64.3 & 66.0 & 70.1 & 71.2 & 84.0 \\\hline
    \end{tabular}
    \caption{Programmgröße in KB über Standorte und Konfigurationen der ML-Modelle.}
    \label{tab:predictions_by_loc_size}
\end{table}
\newline
\newline
Der benötigte Programmspeicher beider ML-Modelle skaliert mit der Standortkomplexität, der Anzahl der verdeckten Schichten bzw. der Waldgröße
und der Anzahl der Neuronen pro verdeckte Schicht bzw. der maximalen Baumhöhe.
Für Entscheidungswälder werden maximal ca. 4,7 MB benötigt.
Der beste Entscheidungswald bei einer Standortkomplexität von 102 benötigt aber nur ca. 1,1 MB.
Die FFNNs hingegen benötigen deutlich weniger Programmspeicher.
Für beide ML-Modelle gibt es Mikrocontroller, die ausreichend Programmspeicher anbieten.
Im Vergleich zu dem WFFNN und WFBNN von Mian sind die FFNNs dieser Arbeit zwischen 54\% und 97,6\% kleiner bei ähnlicher Standortkomplexität.
Die Entscheidungswälder können bis zu 47\% kleiner sein, aber auch bis zu 720\% größer sein.
\section{RAM}
Für den benötigten RAM muss neben dem Anteil der ML-Modelle, die Historie der Sensorwerte und die berechneten Features betrachtet werden.
Wichtig ist dabei der größte akkumulierte benötigte RAM, der zu einem Zeitpunkt benötigt werden kann.
\newline
\newline
Für jeden Sensorwert, bis auf der Detektion von WLAN-Zugangspunkten, wird ein vier Byte Datentyp angenommen.
Für die Detektion der Zugangspunkte wird ein ein Byte Datentyp angenommen.
Insgesamt beträgt der benötigte RAM für einen Vektor von Sensorwerten damit 61 Byte.
Dies setzt sich zusammen aus dem Zeitstempel, der xyz-Komponente von Accelerometer und Gyroskop, dem Lichtsensor,
der Temperatursensor, der Magnetfeldsensor, der Geräuschsensor und fünf möglichen WLAN-Zugangspunkten.
Bei einem Datenfenster von drei Einträgen wird damit 183 Byte für Sensorwerte benötigt.
\newline
\newline
Der Anteil der Features ist abhängig von den Features die für ein bestimmtes Szenario eingesetzt werden.
Insgesamt werden aber 34 Features verwendet, die vereinfacht alle als 4 Byte Datentyp angenommen werden.
Zur Evaluierung des ML-Modells wird nur die aktuelle Feature-Menge benötigt.
Damit wird für die Feature-Menge insgesamt 136 Byte benötigt, wenn alle Features verwendet werden.
\newline
\newline
Zur Ausführung eines Entscheidungswaldes wird für die Rückgabe der Wahrscheinlichkeitsverteilung für jeden Standort vier Byte benötigt.
Je nach Implementierung würde dieser Vektor mehrmals benötigt werden, z. B. bei der parallelen Evaluierung der Entscheidungsbäume skaliert dies mit der Anzahl der Prozessor.
In diesem Fall wird keine Nebenläufigkeit angenommen.
In dieser Arbeit wurden zwischen 9 und 102 Standorte untersucht, d. h. es wurden zwischen 36 und 408 Byte benötigt.
Die Anzahl der Standorte ist aber abhängig von dem Einsatzszenario.
Die anschließende Evaluierung eines Entscheidungswaldes zur Anomalieerkennung kann vernachlässigt werden,
da dieser ein diskretes Ergebnis zurückgeben kann und die benötigte Feature-Menge deutlich kleiner ist.
Damit wird für $N$ Standorte und $K$ Features mit einem Entscheidungswald als ML-Modell zu einem Zeitpunkt
ca. $183 + 4(N + K)$ Byte benötigt, d. h. bei 102 Standorten und 34 Features ca. 727 Byte.
\newline
\newline
Zur Ausführung eines KNN können nur wenige Byte verwendet werden, um die nötigen Multiplikationen eines Neuronen durchzuführen.
Dies würde die Ausführungszeit, und den Energiebedarf, aber signifikant erhöhen, da die benötigten Gewichte ständig aus dem Programmspeicher geladen werden müssen.
Das heißt, es müssen mindestens die Zwischenergebnisse einer Schicht im RAM gehalten werden, sowie ein Gewicht und und ein Bias.
Damit benötigt ein FFNN, dessen größte Schicht $M$ Neuronen hat, mindestens $4(M+2)$ Byte.
Maximal wird $4M$ Byte, zuzüglich der Größe aller Gewichte und Biase benötigt.
Der maximale RAM, der zu einem Zeitpunkt benötigt wird mit einem FFNN, beträgt damit mindestestens $183 + 4(M + K)$,
wobei $M$ die Schicht mit den meisten Neuronen von entweder dem ML-Modell zur Standort- oder Anomalieerkennung ist.
\section{Ausführungszeit und benötigte Energie}
Es ist problemetisch eine sinvolle Estimierung für die benötigte Ausführungszeit und Energie anzugeben, da
die Ausführungszeit und die benötigte Energie abhängig von dem verwendeten Mikrocontroller sind.
Vergleichbare 32-Bit Mikrocontroller mit FPU (Floating Point Unit), zu den Microcontrollern die Dymel verwendet hat \cite{dymelThesis}, sind aus der AVR C-Serie \cite{avr32BitDatasheet}.
Leider ist aus deren Datenblätter keine Information über die Ausführungszeit von Gleitkommazahlinstruktionen und kein Energiemodell zu entnehmen.
Es ist aber anzunehmen, dass deutlich weniger Cyclen benötigt werden für hardwareunterstützte Gleitkommazahloperationen, als Software basierte Alternativen.
Aus diesem Grund wird die Ausführungszeit in Gleitkommazahl- Vergleichen, Multiplikationen, Division, Additionen und Wurzel angegeben,
da diese die integralen Bestandteile der Feature-Extrahierung und Evaluation der ML-Modelle sind.
\newline
\newline
Die in dieser Arbeit vorgeschlagene Architektur (\ref{fig:model_idea}) hat fünf Bestandteile, die jeweils zur Gesamtausführungszeit beitragen.
Die Aufnahme der Sensorwerte wird als konstanter Energieverbrauch angenommen und in dieser Rechnung vernachlässigt.
In der ersten Feature-Extrahierung werden 34 Features aus dem Datenfenster extrahiert.
Tabelle \ref{tab:feature_operation_complexity} zeigt die estimierte Anzahl der Operationen, die pro Art des Features benötigt werden.
\begin{table}[h!]
    \centering
    \begin{tabular}{ | l | c | c | c | c | c | }
        \hline
        Art des Features & Vergleich & Multiplikation & Division & Addition & Wurzel \\\hline
        Standardabweichung (\textbf{7}) & 0 & 3 & 2 & 7 & 1 \\\hline
        Minimum (\textbf{6}) / Maximum (\textbf{6}) & 2 & 0 & 0 & 0 & 0 \\\hline
        Durchschnitt (\textbf{6}) & 0 & 0 & 1 & 2 & 0 \\\hline
        Wert (\textbf{9}) & 0 & 0 & 0 & 0 & 0 \\\hline
    \end{tabular}
    \caption{Estimierte Anzahl der Operationen pro Art des Features bei einer Datenfenstergröße von 3. Fettgedruckte Zahl zeigt die Verwendungsanazahl in der Feature-Menge an.}
    \label{tab:feature_operation_complexity}
\end{table}
\newline
\newline
Zur Evaluierung eines Entscheidungsbaumes werden höchstens $Q\leq\text{Waldgröße}\ \cdot\ \text{Max. Baumhöhe}$ Vergleiche benötigt,
sowie $R\leq\text{Waldgröße}\ \cdot\ \text{\#Standorte}$ Additionen und \text{\#Standorte} zusätzliche Vergleiche, um die einzelnen Entscheidungsbäume zusammenzufassen \cite{dymelThesis}.
Für einen Entscheidungswald mit 8 Bäumen mit einer maximalen Höhe von 16 und 102 Standorten werden damit 230 Vergleiche und 816 Additionen benötigt.
\newline
\newline
Für die Evaluierung eines FFNNs mit der in Kapitel \ref{sec:model_ffnn} beschriebenen Struktur setzen sich die benötigten Operationen folgendermaßen zusammen.
Die Größe der ersten Schicht des FFNNs ist $n_1:=\text{\#Features}$.
Die Größe der letzten Schicht ist $n_m:=\text{\#Standorte}$.
Dazwischen liegen $m-2$ Schichten, die jeweils $K$ Neuronen haben.
Bei jeder Schicht $i$ werden für jedes Neuron in $n_{i+1}$, $n_i$ Multiplikationen und $n_i$ Additionen, sowie ein Vergleich für die Aktivierungsfunktion ReLU verwendet.
Für die SoftMax-Funktion in der letzten Schicht müssen $n_m$ Divisionen durchgeführt werden, $n_m$ Additionen, sowie $n_m$ die $\exp$-Funktion ausgeführt werden.
Insgesamt werden für die Ausführung eines FFNNs mit einer verdeckten Schicht mit 32 Neuronen,
bei 102 Standorten, 4352 Multiplikationen, 4454 Additionen, 134 Vergleiche, 102 Divisionen und 102 $\exp$-Funktionen benötigt.
\newline
\newline
Bei der Feature-Extrahierung für das ML-Modell zur Anomalieerkennung werden nur vier Features extrahiert.
Tabelle \ref{tab:anomaly_feature_operation_complexity} zeigt die estimierte Anzahl der Operationen, die für die einzelnen Features benötigt werden.
Der Entscheidungswald zur Anomalieerkennung besteht aus 4 Entscheidungsbäumen mit einer maximalen Baumhöhe von 8.
Das FFNN zur Anomalieerkennung hat eine verdeckte Schicht mit 16 Neuronen und die Ausgabeschicht hat nur ein Neuron.
Der Entscheidungswald benötigt damit 34 Vergleiche und 8 Additionen.
Das FFNN benötigt 80 Multiplikationen, 81 Additionen, 17 Vergleiche, eine Division und eine $\exp$-Funktion.
\begin{table}[h!]
    \centering
    \begin{tabular}{ | p{4.5cm} | c | c | c | c | c | }
        \hline
        Feature & Vergleich & Multiplikation & Division & Addition & Wurzel \\\hline
        Abweichung zum ØStandortänderungen & 4 & 0 & 2 & 5 & 0 \\\hline
        Abweichung zum ØKlassifizierungswahrscheinlichkeit & 4 & 0 & 2 & 5 & 0 \\\hline
        Topologieverletzung & 5 & 1 & 0 & 1 & 0 \\\hline
        Standardabweichung Top 5 Klassifizierungen & 0 & 5 & 2 & 13 & 1 \\\hline
    \end{tabular}
    \caption{Estimierte Anzahl der Operationen pro Feature der Anomalieerkennung.}
    \label{tab:anomaly_feature_operation_complexity}
\end{table}
\newline
\newline
Tabelle \ref{tab:complexity_summary} fasst die Anzahl der Operationen für eine Konfiguration mit ausschließlich Entscheidungswäldern und FFNNs zusammen.
Es ist zu erwarten, dass die Entscheidungswälder weniger Ausführungszeit benötigen als die FFNNs, da deutlich weniger Operationen benötigt werden,
wodurch sich ein Entscheidungsbaum basierter Klassifizierer in Hinsicht auf die benötigte Energie besser eignet.
Bei konstanter Bewegung wurde in den Testszenarien der Mikrocontroller alle 166 ms aufgeweckt.
Dies ist aber unrealistisch, da die Sensorenbox auch für lange Zeit an einem Ort verbleiben kann, was sich positiv auf den Energiebedarf auswirkt.
\begin{table}[h!]
    \centering
    \begin{tabular}{ | l | c | c | }
        \hline
        Operation & Entscheidungswald & FFNN \\\hline
        Vergleich & 289 & 176 \\\hline
        Multiplikation & 27 & 4459 \\\hline
        Division & 26 & 129 \\\hline
        Addition & 909 & 5444 \\\hline
        Wurzel & 8 & 8 \\\hline
        $\exp$-Funktion & 0 & 103 \\\hline
    \end{tabular}
    \caption{Estimierte Anzahl der Operationen für die gesamte Ausführung mit den im Besipiel genannten Größen für Entscheidungswald und FFNN.}
    \label{tab:complexity_summary}
\end{table}
