\chapter{Evaluation}
In der Evaluation werden drei Aspekte betrachtet: Klassifizierungsgenauigkeit, Robustheit und Resourcennutzung.
Bei der Klassifizierungsgenauigkeit wird einerseits die Standorterkennung und andererseits die Anomalieerkennung evaluiert.
Dabei wird sowohl auf verschiedene Größen von FFNN und Entscheidungswälder eingegangen,
als auch auf verschieden viele zu unterscheidenden Standorte.
\newline
\newline
Bei der Robustheit wird auf den Fehler des besten FFNN und Entscheidungswald bei verschiedene Fehlerszenarien eingegangen.
Diese bestehen aus fehlerhafte Sensordaten durch Rauschen oder ausgefallenen Sensoren,
Routen mit permutierten Teilstücken und Routen, bei denen simuliert wird, dass der letzte Standort übersprungen wurde.
\newline
\newline
Bei der Resourcennutzung wird auf die Programmgröße und die Ausführungszeit der besten ML-Modelle eingegangen.
Außderdem wird der Energieverbrauch für verschiedene Szenarien eingeschätzt.
\newline
\newline
Unterschieden werden zwei Varianten der Testmengen.
Diese unterscheiden sich in der Art, wie der die Features für den vorherigen Standort bestimmt werden.
In der ersten Variante sind alle vorherigen Standorte korrekt.
In der zweiten Variante ist der erste vorherige Standort 0 und alle folgenden vorherigen Standorte werden iterativ durch das ML-Modell bestimmt,
d.~h. in dieser Variante wird der propagierte Fehler durch die Rückwärtskante des ML-Modells betrachtet.

\section{Metriken}
In dieser Arbeit werden verschiedene Metriken zur Ermittlung der Klassifizierungsgenauigkeit ermittelt.
Zunächst die übliche Klassifizierungsgenauigkeit (\ref{formular:simple_accuracy}), in der die Anzahl der korrekt klassifizierten Standorte mit der Gesamtanzahl verglichen werden.
\begin{align}
    \label{formular:simple_accuracy}
    P(A) := \frac{\text{Anzahl korrekter Klassifizierungen}}{\text{Gesamtanzahl}}
\end{align}
Die zweite Metrik (\ref{formular:accuracy_metrik2}) betrachtet die Klassifizierungsgenauigkeit unter Tolerierung, dass ein Standort
fünf bzw. zehn Klassifizierungen kontinuierlich zu früh oder zu spät verlassen wurde,
d.~h. Fehlklassifizierungen werden vernachlässigt, wenn kontinuierlich der letzte korrekte Standort bzw. der nächste korrekte
Standort klassifiziert wird mit einer Gesamttoleranz von fünf bzw. zehn Klassifizierungen.
\begin{flalign}
    \label{formular:accuracy_metrik2}
    &\epsilon \in \{5, 10\} \nonumber\\
    &L := \text{Menge von dem ML-Modell klassifizierten Standorte.} \nonumber\\
    &K := \text{Menge von den wirklichen Standorten.} \nonumber\\
    &\Phi(i) := \text{Index von dem nächsten Standort.} \nonumber\\
    &\Psi(i) := \text{Index von dem vorherigen Standort.} \nonumber\\
    &\Omega(i) := \Phi(i)-i\leq\epsilon\wedge\hspace{-0.3cm} \bigwedge\limits_{i\leq q \leq \min(\#K, \Phi(i))}\hspace{-0.3cm} L_q=K_{\Phi(i)} \nonumber\\
    &\Theta(i) := i-\Psi(i)\leq\epsilon\wedge\hspace{-0.3cm} \bigwedge\limits_{\max(0, \Psi(i))\leq q \leq i}\hspace{-0.3cm} L_q=K_{\Psi(i)} \nonumber\\
    &P(B) := \frac{\#\{L_i | L_i=K_i \vee \Omega(i) \vee \Theta(i)\text{ für } i\in\{0, 1, ..., \#L - 1\}\}}{\#K}
\end{flalign}
Die Testmenge existiert in zwei Varianten, die sich durch die Bestimmung des vorherigen Standortes in den beinhalteten Feature-Mengen unterscheidet.
In der ersten Variante sind alle vorherigen Standorte korrekt.
In der zweiten Variante ist der erste vorherige Standort 0 und alle folgenden vorherigen Standorte werden iterativ durch das ML-Modell bestimmt,
d.~h. in dieser Variante wird der propagierte Fehler durch die Rückwärtskante des ML-Modells betrachtet.
\newline
\newline
Für die zweite Variante sind zwei weitere Metriken relevant.
Bei diesen Metriken wird die Klassifizierungsgenauigkeit bestimmt, unter der Bedingung, dass der vorherige
Standort korrekt (\ref{formular:accuracy_previous_was_correct}) bzw. falsch (\ref{formular:accuracy_previous_was_wrong}) war.
\begin{align}
    \label{formular:accuracy_previous_was_correct}
    P(C) := \frac{\text{Anzahl korrekter Klassifizierungen, wenn vorheriger Standort korrekt war}}{\text{Alle Klassifizierungen, wenn vorheriger Standort korrekt war}}
\end{align}
\begin{align}
    \label{formular:accuracy_previous_was_wrong}
    P(D) := \frac{\text{Anzahl korrekter Klassifizierungen, wenn vorheriger Standort falsch war}}{\text{Alle Klassifizierungen, wenn vorheriger Standort falsch war}}
\end{align}
\section{Klassifizierungsgenauigkeit}
\begin{itemize}
    \item Metriken
    \item Wie groß ist die Wahrscheinlichkeit, dass die Orte korrekt erkannt werden?
    \item Entscheidungsbaum vs KNN
    \item Skalierung mit Anzahl der Orte
    \item Signifikanz der Features
    \item Einfluss von einzelnen Features für die Klassifizierungsgenauigkeit(?) => Last Distinct location möglicherweise schlecht, da die eigentliche Last Distinct Location oft durch fehlende Interrupts übersprungen wird
    \item Welche Feature werden genutzt? => Abhängig von Feature Importance und wie günstig zu berechnen
    \item Ist es sinnvoll für jeden Ort ein Feature zu haben, dass auf eins gesetzt wird, wenn der Ort erkannt wurde und ansonsten exponentiell abfällt. Wie schnell sollte es fallen, wenn ja?
    \item Werden mehr Trainingsdaten benötigt mit steigender Ort Anzahl? Wenn ja wie viel? (Hier oder bei ML-Modell Training)
    \item Irgendwo drüber reden wie wir evaluieren, also continued predict vs predict und die ganzen Testsets
    \item Wie viele Trainingsdaten werden benötigt?
    \begin{itemize}
        \item Um KNN zu trainieren?
        \item Um Entscheidungsbaum zu trainieren?
        \item Ggf. Unterschiede klären
        \item (Gehört das schon in eine Evaluation, oder ist das hier okay?)
    \end{itemize}
\end{itemize}
\section{Klassifizierungsgenauigkeit der Anomalien}
\label{sec:eval_anomalieerkennung}
Bei der Anomalieerkennung werden Entscheidungswälder und FFNNs mit den besten ML-Modellen zur Standorterkennung trainiert und mit den drei Baseline-Modellen verglichen.
Tabelle \ref{tab:anomaly_detection_prediction_accuracy} zeigt die Klassifizierungsgenauigkeiten über die verschiedenen Standortkomplexitäten,
wobei die Klassifizierungsgenauigkeit $P(A)$ nochmal genauer aufgeschlüsselt ist in den Anteil der korrekten Klassifizierungen, wenn eine bzw. keine Anomalie vorlag.
Die trainierten FFNNs geben stets aus, dass keine Anomalie vorliegt.
Es ist unklar, warum die FFNNs sich so verhalten.
\begin{table}[h!]
    \hspace{-1cm}
    \begin{tabular}{ | l | c | c | c | c | c | c | c | c | }
        \hline
        Standorte & 9 & 16 & 17 & 25 & 32 & 48 & 52 & 102 \\\hline
        \multicolumn{9}{ | l |}{$P(A)$}\\\hline
        Entscheidungswald & 82,59\% & 81,19\% & 87,14\% & 84,91\% & 79,06\% & 83,47\% & 81,93\% & 76,00\% \\\hline
        FFNN & 77,88\% & 77,88\% & 77,88\% & 77,88\% & 77,88\% & 77,88\% & 77,88\% & 77,88\% \\\hline
        Topologie (DT) & 84,77\% & 30,57\% & 83,51\% & 79,76\% & 28,63\% & 24,97\% & 80,55\% & 29,47\% \\\hline
        Topologie (KNN) & 86,10\% & 52,17\% & 77,72\% & 79,30\% & 45,06\% & 41,92\% & 74,77\% & 43,55\% \\\hline
        \multicolumn{9}{ | l |}{Anteil korrekt klassifiziert, indem Anomalie vorlag}\\\hline
        Entscheidungswald & 34,86\% & 35,52\% & 52,58\% & 50,92\% & 32,21\% & 50,64\% & 23,21\% & 1,92\% \\\hline
        FFNN & 0,00\% & 0,00\% & 0,00\% & 0,00\% & 0,00\% & 0,00\% & 0,00\% & 0,00\% \\\hline
        \multicolumn{9}{ | l |}{Anteil korrekt klassifiziert, indem keine Anomalie vorlag}\\\hline
        Entscheidungswald & 96,14\% & 94,41\% & 97,05\% & 95,83\% & 92,48\% & 93,13\% & 98,96\% & 97,18\% \\\hline
        FFNN & 100,00\% & 100,00\% & 100,00\% & 100,00\% & 100,00\% & 100,00\% & 100,00\% & 100,00\% \\\hline
    \end{tabular}
    \caption{Metrik $P(A)$ über Standorte und verschiedenen Konfigurationen der Modelle zur Anomalieerkennung.}
    \label{tab:anomaly_detection_prediction_accuracy}
\end{table}
\newline
\newline
Die Entscheidungswälder hingegen eignen sich besser für den Anomalieerkennungszweck.
Es werden zwischen 1,92\% und 52,58\% der Anomalien erkannt und zwischen 1,04\% und 7,52\% falsch als Anomalien erkannt.
Die Klassifizierungsgenauigkeit des Entscheidungswaldes zur Anomalieerkennung ist abhängig von der Klassifizierungsgenauigkeit zur Standorterkennung
und von der Standortkomplexität.
Je besser das Standorterkennungsmodell und je höher die Standortkomplexität, desto höher ist die Anomalieerkennungsrate.
Aus diesem Grund ist die Klassifizierungsgenauigkeit bei den Standortkomplexitäten, die mit der Kodierungsmethode mit Kanten und Knoten zusammenhängen,
geringer, als bei der Kodierungsmethode, wo nur die Knoten kodiert werden.



\begin{itemize}
    \item Was ist das?
    \item Zusammenhang zu Enkodierungsansatz
    \item Schwierig zu trainieren, da man einerseits Robust sein will und andererseits nicht weiß was trainiert werden soll
    \item Post Processing
    \item Metriken: Location Change Frequency, Accumulated Confididence, Fraction Zero
    \item Sag Motivation, warum diese Metriken (Indikatoren)
    \item Beispiele
    \item Wie zuverlässig können Anomalien erkannt werden?
    \item Vergleich mit Coin-Toss, True und False
    \item Vergleiche Enkodierungsansätze, ob einer besser als der andere ist
\end{itemize}

\section{Signifikanz der Features}
\begin{itemize}
    \item Signifikanz der Features
    \item Einfluss von einzelnen Features für die Klassifizierungsgenauigkeit(?) => Last Distinct location möglicherweise schlecht, da die eigentliche Last Distinct Location oft durch fehlende Interrupts übersprungen wird
    \item Welche Feature werden genutzt? => Abhängig von Feature Importance und wie günstig zu berechnen
    \item Ist es sinnvoll für jeden Ort ein Feature zu haben, dass auf eins gesetzt wird, wenn der Ort erkannt wurde und ansonsten exponentiell abfällt. Wie schnell sollte es fallen, wenn ja?
\end{itemize}

\section{Benötigte Anzahl der Trainingsdaten}
\begin{itemize}
    \item Werden mehr Trainingsdaten benötigt mit steigender Ort Anzahl? Wenn ja wie viel? (Hier oder bei ML-Modell Training)
    \item Wie viele Trainingsdaten werden benötigt?
    \begin{itemize}
        \item Um KNN zu trainieren?
        \item Um Entscheidungsbaum zu trainieren?
        \item Ggf. Unterschiede klären
        \item (Gehört das schon in eine Evaluation, oder ist das hier okay?)
    \end{itemize}
\end{itemize}

\section{Fehlertoleranz}
\begin{itemize}
    \item Irgendwas über Robustheit
    \item Metriken => Insbesondere continued\_predict hier erwähnen
    \item Wenn falscher Ort erkannt wurde, wie lange dauert es um wieder den korrekten Ort zu finden? Was ist die Wahrscheinlichkeit dafür? Wovon hängt es ab, dass wieder der richtige Ort erkannt wurde?
    \item Was passiert wenn Sensoren ausfallen? Wie robust ist es dagegen?
    \item Transportbox nicht dem trainierten Pfad folgt? Wie Robust ist sie dagegen? => Sezenarien enumiereren, wo sowas passieren kann, dann Testszenario erklären.
    \item Änderungen der Fabrik, e.g. Licht, Wärme, Magnet, Schnelligkeit der Fließbänder
    \item Einfluss von einzelnen Features für die Fehlertoleranz(?)
    \item In welchen Fällen ist die Ortung unzuverlässig? => Welche Fälle sind schwierig?
    \item Schlussfolgerung: Wie Fehlertolerant ist die Ortung? => Zusammenfassende Metriken über alle Testsets?
\end{itemize}
\section{Ressourcenbedarf auf dem Mirkocontroller}
\begin{itemize}
    \item Energieverbrauch
    \item Speicherverbrauch
    \item RAM verbrauch
    \item Siehe Kubik und Co.
\end{itemize}

\begin{table}[h!]
    \hspace{-1.5cm}
    \begin{tabular}{ | c | c | c | c | c | c | c | c | c | c | }
        \hline
        \multicolumn{2}{ | l |}{acc / Standorte } & 9 & 16 & 17 & 25 & 32 & 48 & 52 & 102 \\\hline
        \multicolumn{10}{| l |}{\textbf{Entscheidungswälder}}\\\hline
        Waldgröße & Max. Baumgröße & \multicolumn{8}{ c |}{}\\\hline
        16 & 8 & 98.65\% & 97.33\% & 97.96\% & 97.21\% & 95.85\% & 91.37\% & 94.31\% & 86.13\% \\\hline
        16 & 16 & 98.77\% & 98.78\% & 98.54\% & 98.23\% & 98.04\% & 96.98\% & 98.00\% & 96.17\% \\\hline
        16 & 32 & 98.72\% & 98.63\% & 98.46\% & 98.29\% & 97.57\% & 97.32\% & 98.07\% & 96.26\% \\\hline
        16 & 64 & 98.72\% & 98.63\% & 98.46\% & 98.31\% & 97.57\% & 97.32\% & 97.98\% & 96.26\% \\\hline
        8 & 32 & 98.58\% & 98.19\% & 98.40\% & 98.30\% & 97.35\% & 96.78\% & 98.00\% & 95.90\% \\\hline
        16 & 32 & 98.72\% & 98.63\% & 98.46\% & 98.29\% & 97.57\% & 97.32\% & 98.07\% & 96.26\% \\\hline
        32 & 32 & 98.82\% & 98.91\% & 98.60\% & 98.43\% & 98.32\% & 97.51\% & 98.11\% & 96.50\% \\\hline
        64 & 32 & 98.70\% & 98.91\% & 98.56\% & 98.46\% & 98.17\% & 97.78\% & 98.23\% & 96.61\% \\\hline
        32 & 64 & 98.82\% & 98.91\% & 98.60\% & 98.43\% & 98.32\% & 97.51\% & 98.11\% & 96.42\% \\\hline
        \multicolumn{10}{| l |}{\textbf{Feed Forward neuronale Netzwerke}}\\\hline
        \#Schichten & \#Neuronen & \multicolumn{8}{ c |}{}\\\hline
        1 & 16 & 97.86\% & 97.05\% & 96.47\% & 97.27\% & 92.99\% & 90.83\% & 95.66\% & 87.51\% \\\hline
        1 & 32 & 98.21\% & 98.07\% & 97.15\% & 97.24\% & 94.21\% & 93.26\% & 96.92\% & 89.69\% \\\hline
        1 & 64 & 98.12\% & 98.05\% & 97.72\% & 97.85\% & 95.09\% & 94.39\% & 96.38\% & 90.57\% \\\hline
        1 & 128 & 98.22\% & 98.79\% & 97.77\% & 97.78\% & 96.71\% & 94.81\% & 97.52\% & 92.22\% \\\hline
        2 & 32 & 97.78\% & 97.81\% & 97.53\% & 98.06\% & 96.17\% & 94.24\% & 97.08\% & 91.70\% \\\hline
        4 & 32 & 98.08\% & 97.70\% & 97.54\% & 98.10\% & 95.51\% & 95.26\% & 97.02\% & 91.84\% \\\hline
        8 & 32 & 98.14\% & 95.97\% & 97.48\% & 97.99\% & 95.65\% & 94.47\% & 97.73\% & 94.41\% \\\hline
        4 & 64 & 98.33\% & 97.51\% & 97.78\% & 97.96\% & 94.18\% & 96.55\% & 97.54\% & 91.60\% \\\hline
    \end{tabular}
    \caption{TODO: Caption}
    \label{tab:TODO_LABEL}
\end{table}
