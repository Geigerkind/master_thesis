\section{Simulation von Interrupts}
Die Simulationsdaten, die mit CoppeliaSim aufgenommen wurden, enthalten alle 50 ms Einträge für die aufgenommen Sensoren.
Unter realen Bedingungen wäre solch eine Abtastrate aber nicht mit den Limitierungen der Batterielaufzeit zu vereinbaren.
Aus diesem Grund führen in solchen Systemen Sensoren \textit{Interrupts} aus, wenn eine signifikante Änderung festgestellt oder ein Schwellenwert überschritten wurde.
Interrupts sind Benachrichtigungen an die CPU, dass der Sensor ausgelesen werden, wodurch die CPU in der Zwischenzeit schlafen und somit Energie preservieren kann.
\newline
\newline
Um dieses Verhalten nachzustellen werden diese Interrupts simuliert, wodurch die Datenmenge gefiltert wird.
In dieser Arbeit wird die Änderung zu dem letzten Interrupt eines Sensors modelliert,
d. h. jeder Sensor merkt sich seinen Sensorwert, wenn es einen Interrupt auslöst
und führt das nächste mal nur einen Interrupt aus, wenn sich der Sensorwert um einen bestimmten Prozentsatz zu dem gemerkten Sensorwert unterscheidet.
\newline
\newline
Dadurch wird einerseits ein realistischeres Szenario dargestellt,
denn von einer Sensorenbox die still steht würde auch keine Aktivität erwartet werden.
Andererseits verringert sich die Datenmenge, wodurch die Trainingszeit verringert wird.

\iffalse
% Möglicherweise schlecht, wegen dem Datenfenster
\newline
\newline
Allerdings hat sich gezeigt, dass manche Standorte deutlich unterrepräsentiert sind in der Trainingsmenge.
Aus diesem Grund wurde zusätzlich für die Trainingsmenge eine Mindestabtastrate von 4 Hertz eingeführt,
falls in diesem Zeitraum kein Interrupt für diesen Standort stattgefunden hat.
\fi