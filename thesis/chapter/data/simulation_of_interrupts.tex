\section{Simulation von Interrupts}
Die Simulationsdaten, die mit CoppeliaSim aufgenommen wurden, enthalten alle 50ms Einträge für die aufgenommen Sensoren.
Unter realen Bedingungen wäre so eine Abtastrate aber nicht mit den Limitierungen der Batterielaufzeit zu vereinbaren.
Aus diesem Grund sollten Sensoren \textit{Interrupts} auslösen, wenn eine signifikante Änderung festgestellt oder ein Schwellenwert überschritten wurde.
Interrupts sind Benachrichtigungen an die CPU, dass der Sensor ausgelesen werden sollte, wodurch die CPU in der Zwischenzeit schlafen und somit Energie gespart werden kann.
\newline
\newline
Um dieses Verhalten nachzustellen werden diese Interrupts simuliert, wodurch die Datenmenge gefiltert wird.
In dieser Arbeit wird die Änderung zum letzten Interrupt eines Sensors modelliert,
d. h. jeder Sensor merkt sich seinen Sensorwert, wenn er einen Interrupt auslöst
und führt das nächste Mal nur einen Interrupt aus, wenn sich der Sensorwert um einen bestimmten Prozentsatz zum gemerkten Sensorwert unterscheidet.
\newpage
Dadurch wird einerseits ein realistischeres Szenario dargestellt,
denn von einer Sensorenbox, die still steht, würde auch keine Aktivität erwartet werden.
Andererseits verringert sich die Datenmenge, wodurch die Trainingszeit verringert wird.
\newline
\newline
Unterschieden werden drei Ansätze, um Interrupts zu erzeugen.
Der erste Ansatz vergleicht den Betrag der Differenz mit einem festen Schwellenwert.
Der zweite Ansatz erfordert, dass die Änderung einen Prozentsatz des vorherigen Wertes ausmacht.
Der dritte Ansatz erfordert nur eine Änderung.
\newline
\newline
Die verwendeten Ansätze und Schwellenwerte sind Tabelle \ref{tab:interrupt_values} zu entnehmen.
Für Sensorwerte des Accelerometers und Gyroskops, sowie des Magnetfeldsensors wurde der erste Ansatz verwendet,
da signifikante Änderungswerte nicht relativ zum vorherigen Sensorwert sind, sondern lediglich eine absolute Änderung erzeugen.
Für Sensorwerte des Temperatur-, Licht- und Geräuschsensors wurde der zweite Ansatz verwendet,
da diese ein allgegenwärtiges Rauschen aufnehmen, das je nach Umgebung varrieren kann.
Für die WLAN-Zugangspunkte wird der dritte Ansatz verwendet, da die Sensorwerte binär sind.
Die Schwellenwerte wurden so gewählt, dass jeder Standort Interrupts auslöst und nicht mehr als 50\% der Ursprungsdaten genutzt werden.
Je nach Route werden zwischen 15\% und 50\% verwendet.
\begin{table}[h!]
    \centering
    \begin{tabular}{ | l | c | c | c | }
        \hline
        Sensordaten & Fester Schwellenwert & Variabler Schwellenwert & Änderung \\\hline
        Accelerometer & 0.1 & - & - \\\hline
        Gyroskop & 0.1 & - & - \\\hline
        Magnetfeld & 8 & - & - \\\hline
        Temperatur & - & 0.12 & - \\\hline
        Licht & - & 0.12 & - \\\hline
        Geräusch & - & 0.16 & - \\\hline
        WLAN-Zugangspunkte & - & - & 1 \\\hline
    \end{tabular}
    \caption{Schwellenwerte der simulierten Interrupts.}
    \label{tab:interrupt_values}
\end{table}

