\chapter{Trainings- und Validationsdaten}
Zu dem Zeitpunkt, zu dem diese Arbeit verfasst wurde, existierte noch keine Möglichkeit Echtdaten in einem
realistischen Szenario mit einem Mikrocontroller aufzunehmen, der über alle nötigen Sensoren verfügt.
Aus diesem Grund ist es nötig Daten simulativ zu erfassen.
\newline
\newline
Zur Datenerfassung wird der allzweck Robotersimulator CoppeliaSim verwendet \cite{coppeliaSim}.
Mit diesem Simulator werden verschiedene Fabrikszenarien simuliert und dabei verschiedene Sensorwerte erfasst.
Diese werden dann in einem Vorverarbeitungsschritt gefiltert und mit Sensordaten ergänzt, die in CoppeliaSim nicht verfügbar sind.
Zuletzt werden Features extrahiert und die resultierende Datenmenge in Trainings- und Validationsdaten unterteilt.

\section{Sensordaten aus Simulator CoppeliaSim}
Insgesamt wurden vier Routen über 20 Zyklen jeweils zweimal erfasst.
Das geschah einmal für Trainingsdaten und einmal für Testdaten, wobei die letzten fünf Zyklen der
Trainingsdaten als Validationsmenge genutzt werden.
Dabei wurden alle 50ms die xyz-Koordinaten, Accelerometerdaten und Gyroskopdaten erfasst, sowie Lichtintensität und Metadaten.
Zu den Metadaten gehören Zeitstempel, Beschriftung des Routenabschnitts und Beschriftung des derzeitigen Zykluses.
Ein Zyklus ist der vollständige Umlauf einer Route.
\newline
\newline
Abbildung \ref{fig:simple_square_labeled} zeigt eine der vier Routen \glqq simple\_square\grqq.
Jede Route ist mit Markierungen für Zyklen und Standorte ausgestattet.
Die Zyklusmarkierung wird genutzt, um die Datensätze mit dem derzeitigen Zyklus zu beschriften.
Jedes Mal, wenn die Sensorbox diese Markierung überschreitet, wird der Zähler für den Zyklus inkrementiert.
\newpage
Die Standortmarkierung wird genutzt, um die Datensätze mit dem derzeitigen Routenabschnitt zu beschriften.
Jedes Mal, wenn die Sensorbox diese Markierung überschreitet, wird der derzeitige Wert für den Routenabschnitt auf den Wert der Markierung gesetzt.
Dabei werden alle aufgenommenen Datensätze immer mit dem derzeitigen Wert für den Routenabschnitt annotiert.
\begin{figure}[h!]
    \centering
    \includegraphics[width=0.75\linewidth]{images/simple_square_labeled.png}
    \caption{Modell der Route \glqq simple\_square\grqq\ in CoppeliaSim mit Beschriftungen.}
    \label{fig:simple_square_labeled}
\end{figure}
\newline
\newline
Neben \glqq simple\_square \grqq\ gibt es noch drei weitere Routen (siehe Abbildungen \ref{fig:long_rectangle}, \ref{fig:rectangle_with_ramp} und \ref{fig:many_corners}).
Die Route \glqq long\_rectangle\grqq\ weist lange Pfade mit wenig Änderungen auf.
Die Route \glqq rectangle\_with\_ramp\grqq\ besitzt zusätzlich zwei Rampen, wodurch Höhenunterschiede simuliert werden.
Die Route \glqq many\_corners\grqq\ ist sehr komplex und hat viele verschiedene Standorte.
Die Förderbänder können verschiedene Geschwindigkeiten haben mit sowohl abrupten Übergängen als auch fließenden Übergängen zueinander.
\newline
\newline
Je nach Kodierungsart (siehe Kapitel \ref{sec:model_location_encoding}) müssen die Knoten und Kanten des zyklischen Graphen als Standorte kodiert werden.
Mian hatte in seiner Arbeit die Kanten während der simulation markiert \cite{naveedThesis}.
In dieser Arbeit wurde der Programm-Code seiner Arbeit für die Simulation übernommen, weswegen die aufgenommen Daten initial genau so beschriftet sind.
Aus diesem Grund müssen die aufgenommen Datensätze transformiert werden, sodass nicht die Kanten, sondern entweder Knoten oder Knoten und Kanten kodiert werden.
Zu einem Knoten gehören die Datensätze, die sich in einem Umkreis vom ersten Datensatz befindet, der eine einzigartige Beschriftung in der Simulation erhalten hatte.
Bei dem Kodierungsansatz, der nur die Knoten kodiert, werden die restlichen Datensätze als \textit{unbekannt} beschriftet.
Bei dem Kodierungsansatz, der Knoten und Kanten kodiert, werden die restlichen Datensätze mit einer neuen einzigartigen Beschriftung markiert, die abhängig von der
topologischen Position der zuvor bestimmten Knoten ist.

\section{Künstlichen Sensordaten}
\begin{itemize}
    \item Motivation: Warum ist das nötig?
\end{itemize}

\subsection{Magnetfeld}
\begin{itemize}
    \item Welchen Sensor spiegelt das wieder?
    \item Wie funktioniert das Modell?
    \item Was und Wie wurden Daten ergänzt?
\end{itemize}

\subsection{Temperatur}
\begin{itemize}
    \item Welchen Sensor spiegelt das wieder?
    \item Wie funktioniert das Modell?
    \item Was und Wie wurden Daten ergänzt?
\end{itemize}

\subsection{Lautstärke}
\begin{itemize}
    \item Welchen Sensor spiegelt das wieder?
    \item Wie funktioniert das Modell?
    \item Was und Wie wurden Daten ergänzt?
\end{itemize}

\subsection{WLAN Zugangspunkte}
\begin{itemize}
    \item Welchen Sensor spiegelt das wieder?
    \item Wie funktioniert das Modell?
    \item Was und Wie wurden Daten ergänzt?
\end{itemize}
\section{Simulation von Interrupts}
\begin{itemize}
    \item Sag irgendwie, dass es mehrere Arten gibt wie man das machen kann und sag welche du ausgewählt hast.
    \item Motivation => Energieverbrauch, Spiegelung der echten Datenaufnahme, Reduzierung der Trainingsdaten
    \item Wie funktioniert ist?
    \item Wie und Wann bei der Datenverarbeitung wird es gemacht?
    \item So wird bei einer Idle Box auch kein Interrupt erzeugt. (Sneaky Beispiel)
    \item Vorteile bei der Trainingszeit, da weniger Trainingsdaten
    \item Wie gut repräsentiert ist jeder Standort danach? => Graph vorher vs. nachher mit training sampling rate
    \item => Problem: Locations können einfach verpasst werden => Sampling rate?
    \item => Damit alle locations ausreichend in den Trainingsdaten repräsentiert sind, wird eine training sampling rate eingeführt
\end{itemize}
\section{Feature-Extrahierung}
Die Feature-Extrahierung ist der Prozess, in dem Feature aus den Rohdaten der Datenmenge extrahiert werden.
In dieser Arbeit findet dieser Schritt nach der Filterung durch künstliche Interrupts statt.
Feature sind berechnete Attribute und Eigenschaften von einem oder mehreren Sensorwerten der Rohdaten.
\newline
\newline
Durch die Feature-Extrahierung müssen die ML-Modelle diese Features nicht selbständig lernen, sondern lediglich darauf abstrahieren.
Dies erleichtert das Training, kann aber gerade für Deep NN die Generalisierungsfähigkeit der Modelle einschränken,
wenn die Features nicht manuell konstruiert wurden \cite{seide2011feature}.
Einerseits benötigt aber der Entscheidungsbaum ML-Modell aber einen solchen Prozess,
da durch viel Rauschen eine Partition von Rohdaten deutlich komplexer ist als Features.
Andererseits kann das FFNN durch die Hardware Limitierungen möglicherweise nicht groß genug sein, um diese Features zu erlernen.
\newline
\newline
Mian hat in seiner Arbeit ein Datenfenster verwendet, um die Sensorwerte zu glätten \cite{naveedThesis}.
Als Datenfenster werden die letzten $N$ Einträge der Sensordaten bezeichnet, wobei $N$ die Fenstergröße ist.
Allerdings hat Mian auch eine hohe Abtastrate verwendet, wodurch die Unterschiede zu
hintereinander liegenden Datensätzen gering ist und Rauschen eine große Auswirkung hat.
Durch die künstlichen Interrupts werden nur Datensätze verwendet, die signifikante Änderungen enthalten.
In der Praxis hat sich dennoch ein Datenfenster als sinnvoll erwiesen, da dadurch mehr Features konstruiert werden können,
die eine bessere Generalisierung ermöglichen.
Allerdings hat sich gezeigt, dass ein zu großes Datenfenster durch die im Mittel geringe Abtastrate, die Klassifizierungsgenauigkeit verringert.
Dies kann durch die im Mittel geringe Abtastrate durch die künstlichen Interrupts begründet werden, da in einem großen Datenfenster dann
noch Datensätze enthalten sind, die zur Erkennung eines Standorts nicht relevant sind.
\newline
\newline
Tabelle \ref{tab:all_features} listet alle verwendeten Features auf.
Zunächst werden aus jeden Sensor die gleiche Menge von Features extrahiert, sofern der Sensor dies erlaubt.
Aus den Datenfenstern der Sensordaten werden die Standardabweichung, Minimum, Maximum und der Durchschnitt für jeden Sensor berechnet.
Zusätzlich der momentane Wert jedes Sensors als Feature verwendet.
Da die Werte des Accelerometers und Gyroskops abhängig von der Ausrichtung der Sensorenbox ist,
wird von denen nur der Betrag der Summe der x-, y- und z-Komponenten verwendet.
Für die Detektionsdaten der WLAN-Zugangspunkte erschien die Features aus Standardabweichung, Minimum, Maximum und Durchschnitt nicht sinnvoll, da diese nur binäre Werte annehmen.
\begin{table}[h!]
    \hspace{-0.65cm}
    \begin{tabular}{ | l | c | c | c | c | c | }
        \hline
        Sensordaten & Standardabweichung & Minimum & Maximum & Durchschnitt & Wert \\\hline
        Accelerometer & X & X & X & X & X \\\hline
        Gyroskop & X & X & X & X & X \\\hline
        Magnetfeld & X & X & X & X & X \\\hline
        Temperatur & X & X & X & X & X \\\hline
        Licht & X & X & X & X & X \\\hline
        Geräusch & X & X & X & X & X \\\hline
        WLAN-Zugangspunkte & - & - & - & - & X \\\hline
        Letzter Standort & - & - & - & - & X \\\hline
        Zeit & X & - & - & - & - \\\hline
    \end{tabular}
    \caption{Extrahierte Features aus verfügbaren Sensordaten.}
    \label{tab:all_features}
\end{table}
\newline
\newline
Daneben wurden noch drei weitere Features aus den Metadaten extrahiert.
Das erste Feature ist der zuletzt bestimmte Standort des ML-Modells.
Das zweite Feature ist der zuletzt bestimmte Standort des ML-Modells, dass nicht als unbekannter Standort gilt und nicht der aktuelle Standort ist.
Mian hat ein Feature für jeden zu klassifizierenden Standort als Eingabe benutzt, dass auf eins gesetzt wird, wenn es erkannt wurde und ansonsten exponentiell abfällt \cite{naveedThesis}.
In dieser Arbeit wurde sich dagegen entschieden, da dies schlecht skaliert mit steigender Anzahl von zu klassifizierenden Standorten
und eine Abhängigkeit von dem zuletzt geratenen Standort für die Robustheit vermieden werden sollte.
Das dritte Feature ist die Standardweichung über die Zeit des Datenfensters.
Alle anderen Features sind Zeitunabhängig, da sie auf Zeitunabhängige Features im Datenfenster beziehen.
\newline
\newline
Die Signifikanz der einzelnen Features ist abhängig von der Einsatzumgebung, weshalb in Kapitel \ref{sec:model_training}
ein Feature-Auswahl Schritt in dem Trainingsprozess vorgeschlagen wurde, um die Anzahl der Features zu verringern.
Dabei wird die Signifikanz oder Wichtigkeit über die Permutationswichtigkeit bestimmt.
Aus diesem Grund muss individuell für jedes Einsatzgebiet abgewogen werden, welche Sensoren und Features am meisten Nutzen bringen
im Vergleich zu deren Kosten und Energieverbrauch.

\section{Aufteilung der Daten}
\begin{itemize}
    \item Erkläre irgendwo das Relabeling der Daten, wenn mehrer Routen geladen werden
    \item Kurz und knapp wie und warum werden die Daten aufgeteilt. => Zyklen
    \item Sollten Trainingsdaten um synthetische Daten ergänzt werden?
    \begin{itemize}
        \item Fault Daten, um das Modell Robuster zu machen
        \item Synthetische Routen => Was ist das? Wie werden sie erzeugt?
    \end{itemize}
    \item Wie viele Trainingsdaten werden benötigt?
    \begin{itemize}
        \item Um KNN zu trainieren?
        \item Um Entscheidungsbaum zu trainieren?
        \item Ggf. Unterschiede klären
        \item (Gehört das schon in eine Evaluation, oder ist das hier okay?)
    \end{itemize}
\end{itemize}