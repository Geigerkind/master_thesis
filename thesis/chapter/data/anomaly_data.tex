\section{Anomaliedaten}
\label{sec:data_anomalie}
Für Anomaliedaten werden die bekannten Routen modifiziert, indem Umleitungen eingebaut werden, die teilweise Standorte überspringen
und nicht in den Trainingsdaten vorhanden sind.
Dadurch wird ein Szenario simuliert, in der das Klassifizierungsverhalten von dem ML-Modell zur Standorterkennung auf unbekannten Wegen beobachtet werden kann.
Aus den Klassifizierungsergebnissen des ML-Modells zur Standorterkennung werden schließlich Features extrahiert,
die von dem ML-Modell zur Anomalieerkennung verwendet werden.
\newline
\newline
Wenn eine Anomalie vorliegt, wird erwartet, dass das ML-Modell unsicherer wird und stärker fluktuiert,
wodurch zwei Features motiviert sind.
Das erste Feature ist abgeleitet aus der Anzahl der Standortänderungen.
Zum einen wird die durchschnittliche Anzahl der Standortänderungen in einem Datenfenster bestimmt.
Zum anderen wird die durchschnittliche Anzahl der Standortänderungen bestimmt, wenn keine Anomalie vorliegt.
Das Feature ist der Betrag der Differenz von diesen beiden Komponenten.
\newline
\newline
Das zweite Feature ist analog zu dem ersten Feature konstruiert.
Dieses nutzt ansatt der Anzahl der Standortänderungen die summierte Wahrscheinlichkeit der erkannten Standorte.
Das ML-Modell zur Standorterkennung hat keine diskrete Ausgabe, sondern gibt einen Vektor von Wahrscheinlichkeiten aus,
der für jeden Standort die Klassifizierungswahrscheinlichkeit angibt.
Dabei gilt der klassifizierte Standort als der Eintrag im Vektor mit der höchsten Wahrscheinlichkeit.
Diese Wahrscheinlichkeit wird dann analog zu den Standortänderungen summiert.
\newline
\newline
Zusätzlich wird das Ergebnis des Modells auf Basis der Topologie als Feature verwendet.
Dieses Feature indiziert, dass das ML-Modell zur Standorterkennung ein Fehler gemacht haben muss
oder dass tatsächlich eine Anomalie vorliegt, da die Topologie verletzt wurde.
