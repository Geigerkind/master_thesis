\section{Anomaliedaten}
\label{sec:data_anomalie}
Zur Anomalieerkennung wird ein separates ML-Modell trainiert.
Eine Anomalie ist ein unbekannter Standort oder Pfad, der nicht in den Trainingsdaten enthalten ist.
Das Problem bei der Generierung von Trainingsdaten ist, dass es sehr viele unterschiedliche Arten von Anomalien geben kann,
die unmöglich durch Testdaten repräsentativ dargestellt werden können.
Aus diesem Grund wird das ML-Modell zur Anomalieerkennung nicht auf Sensordaten trainiert, sondern auf den
Klassifizierungsdaten des ML-Modell zur Standorterkennung.
Ziel ist es anhand des Klassifizierungsverhalten des ML-Modells zur Standorerkennung einschätzen zu können, ob eine Anomalie vorliegt.
\newline
\newline
Für Anomaliedaten werden bekannte und unbekannte Routen verwendet, sowie Kombinationen aus diesen Routen.
Diese Daten werden dann von dem ML-Modell zur Standorterkennung klassifiziert.
Aus den Klassifizierungsergebnissen werden schließlich Features extrahiert, die von dem ML-Modell zur Anomalieerkennung verwendet werden.
\newline
\newline
Wenn eine Anomalie vorliegt wird erwartet, dass das ML-Modell unsicherer wird und stärker fluktuiert,
wodurch zwei Features motiviert sind.
Das erste Feature ist abgeleitet aus der Anzahl der Standortänderungen.
Zum einen wird die durchschnittliche Anzahl der Standortänderungen in einem Datenfenster bestimmt.
Zum anderen wird die durchschnittliche Anzahl der Standortänderungen bestimmt, wenn keine Anomalie vorliegt.
Das Feature ist der Betrag der Differenz von diesen beiden Komponenten.
\newline
\newline
Das zweite Feature ist analog zu dem ersten Feature konstruiert.
Dieses nutzt ansatt der Anzahl der Standortänderungen die summierte Wahrscheinlichkeit der erkannten Standorte.
Das ML-Modell zur Standorterkennung hat keine diskrete Ausgabe, sondern gibt einen Vektor von Wahrscheinlichkeiten aus,
der für jeden Standort die Klassifizierungswahrscheinlichkeit angibt.
Davon wird jeweils die höchste Wahrscheinlichkeit addiert, d. h. die Wahrscheinlichkeit des erkannten Standortes.