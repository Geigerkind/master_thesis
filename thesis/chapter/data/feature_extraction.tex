\section{Feature-Extrahierung}
Die Feature-Extrahierung ist der Prozess, in dem Feature aus den Rohdaten der Datenmenge extrahiert werden.
In dieser Arbeit findet dieser Schritt nach der Filterung durch künstliche Interrupts statt.
Features sind berechnete Attribute und Eigenschaften von einem oder mehreren Sensorwerten der Rohdaten.
\newpage
Durch die Feature-Extrahierung müssen die ML-Modelle diese Features nicht selbständig lernen, sondern lediglich darauf abstrahieren.
Dies erleichtert das Training, kann aber gerade für tiefe NN die Generalisierungsfähigkeit einschränken,
wenn die Features nicht manuell konstruiert wurden \cite{seide2011feature}.
Einerseits benötigt das Entscheidungsbaum ML-Modell einen solchen Prozess,
da Features das Rauschen der Rohdaten verringern und dadurch eine Partitionierung vereinfachen.
Andererseits kann das FFNN durch die Limitierungen der Hardware möglicherweise nicht groß genug sein, um diese Features zu erlernen.
\newline
\newline
Mian hat in seiner Arbeit ein Datenfenster verwendet, um die Sensorwerte zu glätten \cite{naveedThesis}.
Als Datenfenster werden die letzten $N$ Einträge der Sensordaten bezeichnet, wobei $N$ die Fenstergröße ist.
Mian hat eine hohe Abtastrate verwendet, wodurch die Unterschiede zu hintereinander liegenden Datensätzen gering ist und Rauschen eine große Auswirkung hat.
Durch die künstlichen Interrupts werden nur Datensätze verwendet, die signifikante Änderungen enthalten,
wodurch große Datenfenster in dieser Arbeit nicht benötigt werden.
Es wird dennoch ein Datenfenster verwendet, da dadurch mehr Features konstruiert werden können, die eine bessere Generalisierung ermöglichen.
Das Datenfenster ist aber signifikant kleiner, da durch die künstlichen Interrupts eine im Mittel geringe Abtastrate zu erwarten ist.
\newline
\newline
Tabelle \ref{tab:all_features} listet alle verwendeten Features auf.
Zunächst werden aus jedem Sensor, sofern der Sensor dies erlaubt, die gleiche Menge von Features extrahiert.
Aus den Datenfenstern der Sensordaten werden die Standardabweichung, Minimum, Maximum und der Durchschnitt für jeden Sensor berechnet.
Zusätzlich wird der momentane Wert jedes Sensors als Feature verwendet.
Da die Werte des Accelerometers und Gyroskops abhängig von der Ausrichtung der Sensorenbox ist,
wird von diesen nur der Betrag der Summe der x-, y- und z-Komponenten verwendet.
Für die Detektionsdaten der WLAN-Zugangspunkte erschien die Standardabweichung, Minimum, Maximum und Durchschnitt nicht sinnvoll, da die Daten nur binäre Werte annehmen.
\begin{table}[h!]
    \hspace{-0.5cm}
    \begin{tabular}{ | p{3.2cm} | c | c | c | c | c | }
        \hline
        Sensordaten & Standardabweichung & Minimum & Maximum & Durchschnitt & Wert \\\hline
        Accelerometer & X & X & X & X & X \\\hline
        Gyroskop & X & X & X & X & X \\\hline
        Magnetfeld & X & X & X & X & X \\\hline
        Temperatur & X & X & X & X & X \\\hline
        Licht & X & X & X & X & X \\\hline
        Geräusch & X & X & X & X & X \\\hline
        WLAN-Zugangspunkte & - & - & - & - & X \\\hline
        Letzter Standort & - & - & - & - & X \\\hline
        Letzter unterschiedlicher Standort & - & - & - & - & X \\\hline
        Zeit & X & - & - & - & - \\\hline
    \end{tabular}
    \caption{Extrahierte Features aus verfügbaren Sensordaten.}
    \label{tab:all_features}
\end{table}
\newline
\newline
Daneben wurden noch drei weitere Features aus den Metadaten extrahiert.
Das erste Feature ist der zuletzt bestimmte Standort des ML-Modells.
Das zweite Feature ist der zuletzt bestimmte Standort des ML-Modells, der nicht als unbekannter Standort gilt und nicht der aktuelle Standort ist.
Mian hat ein Feature für jeden zu klassifizierenden Standort als Eingabe benutzt, dass auf eins gesetzt wird, wenn es erkannt wurde und ansonsten exponentiell abfällt \cite{naveedThesis}.
In dieser Arbeit wurde sich dagegen entschieden, da dies schlecht mit der steigenden Anzahl von zu klassifizierenden Standorten skaliert
und eine Abhängigkeit zu dem zuletzt bestimmten Standort für die Robustheit vermieden werden sollte.
Das dritte Feature ist die Standardweichung über die Zeit des Datenfensters.
Alle anderen Features sind Zeitunabhängig, da sie sich auf Zeitunabhängige Sensorwerte im Datenfenster beziehen.
\newline
\newline
Die Signifikanz der einzelnen Features ist abhängig von der Einsatzumgebung, weshalb in Kapitel \ref{sec:model_training}
ein Feature-Auswahl Schritt in dem Trainingsprozess vorgeschlagen wurde, um die Anzahl der Features zu verringern.
Dabei wird die Signifikanz, oder Wichtigkeit, über die Permutationswichtigkeit bestimmt.
Aus diesem Grund muss individuell für jedes Einsatzgebiet abgewogen werden, welche Sensoren und Features am meisten Nutzen,
im Vergleich zu deren Kosten und Energieverbrauch, bringen.
