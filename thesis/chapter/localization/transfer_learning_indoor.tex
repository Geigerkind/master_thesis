\section{WiFi basierte Indoor-Lokalisierung mit Transfer Lernen}
Pan et al. untersuchten Indoor-Lokaliserung basierend auf WiFi RSS Daten \cite{pan2008transfer}.
\textit{Received Signal Strength} (RSS) wird von dem Empfänger von mehreren Sendern gemessen.
Dadurch steht ein Vektor von Signalstärken zur Verfügung aus denen die Position approximiert werden kann.
\newline
\newline
Sie bemerken, dass bei ML Ansätzen oft zwei Annahmen getroffen werden.
Zum einen wird eine \textit{Offline}-Phase vorausgesetzt mit ausreichend beschrifteten Daten,
d. h. eine Trainingsphase bevor das Modell eingesetzt wird.
Zum anderen wird angenommen, dass das gelernte Modell statisch über Zeit, Raum und Geräte ist.
\newline
\newline
Beispielsweise können sich Daten über die Zeit ändern, wenn Mitarbeiter Mittags zur Kantine gehen.
Es kann schwierig sein ausreichend Daten für große Gebäude zu sammeln.
Verschiedene Geräte können verschiedene Sensorwerte erfassen.
Dies führt zu erheblichen Kalibrierungsaufwand der ML Modelle.
\newline
\newline
Die Autoren schlagen Transfer Learning vor, um dieses Problem zu lösen.
Transfer Learning befasst sich mit dem Problem, wenn Trainings- und Testdaten verschiedener Verteilungen folgen oder in verschiedenen Feature-Räumen repräsentiert sind.
Die gewonnene Erfahrung beim Training soll dabei auf das neue, ähnliche Problem übertragen werden.
Dafür müssen Beziehungen gefunden werden, die den Erfahrungstransfer ermöglichen.
\newline
\newline
Sie trainierten ein \textit{Hidden Markov Model} (HMM), wobei die Ortserkennung als Klassifizierungsproblem diskreter Orte modelliert wurde.
Auf ihren Testdaten waren sie signifikant besser als Ansätze ohne Transfer Learning.
