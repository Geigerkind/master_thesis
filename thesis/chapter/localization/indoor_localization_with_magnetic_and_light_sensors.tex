\section{Indoor-Lokalisierung mit Magnet- und Lichtsensoren}
Wang et al. haben einen Indoor-Lokalisierungsansatz untersucht, der Magnetfelddaten und
Lichtsensordaten eines Smartphones nutzt, um die Position des Gerätes zu bestimmen \cite{wang2018deepml}.
In einem Vorverarbeitungsschritt kombinieren die Autoren Magnetfelddaten und Lichsensordaten zu einer bimodalen Abbildung.
Damit wird ein Deep LSTM (\textbf{L}ong  \textbf{S}hort \textbf{T}erm \textbf{M}emory NN) trainiert.
\newline
\newline
Die Autoren merken an, dass viele Ansätze RSS oder CSI (\textbf{C}hannel \textbf{S}tate \textbf{I}nformation) nutzen, um Indoor-Lokalisierungsmodelle zu generieren.
Diese sind aber unzerverlässig, wenn die Signalstärke schlecht ist oder nicht verfügbar, z. B. in einem Parkhaus.
Dahingegen ist das Magnetfeld und Licht omnipresent.
Die Magnetfelddaten weisen eine geringe Varianz auf.
Diskrete Orte können aber durch Anomalien unterschieden werden, die durch Interferenz von Gebäuden und Geräten verursacht wird.
Licht ist ebenfalls meistens vorhanden und weist unterschiede durch Intensität und Form der Lampe, sowie Schatten und Reflektion auf.
Durch die Kombination dieser Daten können eine vielzahl von diskreten Orten unterschieden werden.
\newline
\newline
Die Autoren verglichen zwei Szenarien.
Das erste Szenario ist ein Labor, welches viele Tische, Stühle und Computer enthält.
Das zweite Senario ist ein langer Korridor.
In ihren Ergebnissen ist in beiden Szenarien der Fehler zum größten Teil unterhalb 0,5 m.
Der größte Fehler betrug im Labor 3,7 m und im Korridor 6,5 m.
Im Vergleich zu einem Modell, dass lediglich die Magnetfelddaten nutzt, erwies sich das Modell, das die Kombination nutzte, als deutlich besser.