\section{Orientierungssinn von Mensch und Tier}
Lokalisierung ist aber nicht nur beschränkt auf Geräte.
Menschen und Tiere haben einen Orientierungssinn, der die Navigation anhand von Orientierungspunkten ermöglicht \cite{krukar2017landmark, menzel1996knowledge}.
Beispielsweise hängt die Navigation von Honigbienen stark von den Orientierungspunkten ab \cite{menzel1996knowledge}.
Anstatt die direkte Route zu wählen, fliegen Honigbienen die Orientierungspunkte ab, um zu ihrem Ziel zu gelangen.
Dabei ist die Navigation robust gegenüber leichte Veränderungen der Orientierungspunkte.
\newline
\newline
Neben Honigbienen zeigen Vögel, Insekten oder Meereslebewesen verschiedene Arten von Navigation durch Ausnutzen ihrer Sinne \cite{aakesson2014animal}.
So gibt es Vögel und Insekten, die anhand der Position der Sonne und ihrer inneren Uhr navigieren.
Nachtaktive Singvögel hingegen scheinen sich anhand der Sterne zu orientieren.
Viele Tiere haben auch einen Sinn, um das Magnetfeld der Erde wahrzunehmen und navigieren basierend auf diesem magnetischen Kompass.