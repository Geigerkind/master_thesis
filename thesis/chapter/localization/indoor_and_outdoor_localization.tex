\section{Indoor- und Outdoor-Lokalisierung}
GPS ist eine weit verbreites Standortbestimmungssystem im Outdoor-Kontext  und kann für eine Vielzahl von Anwendungen eingesetzt werden,
z. B. Tracking, Navigation oder Rettungsaktionen \cite{kaplan2005understanding}.
Es skaliert zu einer arbiträren Anzahl von Nutzern, da es einen device-based Ansatz verwendet.
Die Geräte berechnen aus den empfangenden Signalen von mehreren Satelliten ihre Position.
Dabei ist die Positionsbestimmung bis zu 5 m genau \cite{sadowski2018rssi}.
Das GPS Modul ist aber sehr teuer und bedarf viel Energie.
Außerdem kann es im Indoor-Kontext nicht eingesetzt werden, da die Signalstärke zu den Satelliten stark beeinträchtigt ist.
Aus diesem Grund ist es für Mikrocontroller im Indoor-Bereich ungeeignet.
\newline
\newline
Indoor-Lokalisierung bedarf meist einer hohen Auflösung, muss sicherheitskritische Vorgaben einhalten,
energieeffizient sein, skalierbar sein und geringe Kosten haben \cite{xiao2016survey}.
Es gibt verschiedene Ansätze, die entweder device-based oder device-free sind.
\newline
\newline
Device-based Ansätze nutzen die Sensoren des Geräts, um die Position zu bestimmen.
Beispielsweise können visuelle Features mit der Kamera extrahiert werden,
RSS Messungen des WiFi Netzwerkes durchgeführt werden, oder Bewegungs- und Lichtsensoren verwendet werden.
Der Vorteil sind die geringen Infrastrukturkosten.
\newline
\newline
Device-free Ansätze bedürfen einer Infrastruktur, um Objekte im Interessebereich wahrzunehmen.
Dafür können zum einen TOA (\textbf{T}ime \textbf{O}f \textbf{A}rrival) basierte System verwendet werden, d.~h. die Position wird über
die Reflektion von Radio- oder Schallwellen ermittelt.
Zum anderen kommen auch \textit{Tag}-Systeme zum Einsatz, in denen die RFID-Tags des Objektes an vorinstallierten Orten gelesen wird.
Dadurch kann die Position approximiert werden.
Dabei ist die Auflösung abhängig von der Verteilung und Anzahl der lesenden Geräte im Interessebereich.