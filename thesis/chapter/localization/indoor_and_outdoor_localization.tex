\section{Indoor- und Outdoor-Lokalisierung}
GPS ist eine weit verbreites Standortbestimmungssystem im Outdoor-Kontext  und kann für eine Vielzahl von Anwendungen eingesetzt werden,
z. B. Tracking, Navigation oder Rettungsaktionen \cite{kaplan2005understanding}.
Es skaliert zu einer arbiträren Anzahl von Nutzern, da es einen device-based Ansatz verwendet.
Die Geräte berechnen aus den empfangenden Signalen von mehreren Satelliten ihre Position.
Dabei ist die Standortbestimmung bis zu 5 m genau \cite{sadowski2018rssi},
wobei es Varianten gibt, die noch bessere Auflösungen erzielen können \cite{parkinson1996differential}.
Der Energieverbrauch ist sehr hoch \cite{jurdak2013energy}, wodurch es ungeeignet für kleine batteriegestützte Systeme ist.
In einem Indoor-Kontext kann GPS aber meistens nicht eingesetzt werden,
da das Gebäude die benötigte Signalstärke zu den Satteliten beeinträchtigen kann \cite{xiao2016survey, jin2006indoor}.
Aus diesem Grund ist es für batteriegestützte Mikrocontroller im Indoor-Bereich suboptimal.
\newline
\newline
Indoor-Lokalisierung bedarf meist einer hohen Auflösung, muss sicherheitskritische Vorgaben einhalten,
energieeffizient sein, skalierbar sein und geringe Kosten haben \cite{xiao2016survey}.
Es gibt verschiedene Ansätze, die entweder device-based oder device-free sind.
\newline
\newline
Device-based Ansätze nutzen die Sensoren des Geräts, um die Position zu bestimmen \cite{xiao2016survey}.
Beispielsweise können visuelle Features mit der Kamera extrahiert werden \cite{poulose2019hybrid, cunha2011using},
RSS Messungen des WiFi Netzwerkes durchgeführt werden \cite{pan2008transfer},
Bewegungs- oder Lichtsensoren verwendet werden \cite{poulose2019hybrid, wang2018deepml, xiao2016survey}.
Der Vorteil sind die geringen Infrastrukturkosten \cite{xiao2016survey}.
\newline
\newline
Device-free Ansätze bedürfen einer Infrastruktur, um Objekte im Interessebereich wahrzunehmen \cite{xiao2016survey}.
Diese werden zum Beispiel für Überwachsungsszenarien eingesetzt \cite{qian2018widar2}.
Ansätze nutzen beispielsweise die bestehende Kamerainfrastruktur aus \cite{kim2019info},
WiFi basierte Ansätze \cite{qian2018widar2}, Infrarot basierte Ansätze \cite{kemper2010passive}
oder RFID basierte Ansätze \cite{yang2015see}.