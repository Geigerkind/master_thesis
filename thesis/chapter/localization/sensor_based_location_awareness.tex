\section{Sensorbasierter Orientierungssinn mit FFNN}
Dieser Arbeit ging die Arbeit von Mian voran, der sich zum gleichen Thema mit FFNN auseinander gesetzt hat \cite{naveedThesis}.
Mian nutzte den Simulator CoppeliaSim, um Daten von verschieden komplexen Routen zu generieren.
Die Routen unterschieden sich dabei in der Anzahl verschiedener Orte und Pfade die für einen Zyklus einer Route verwendet werden können.
Die aufgenommenen Daten enthalten Sensorwerte für Beschleunigung, Gyroskop, Licht und Beschriftungen für die Standorte.
Dabei werden als Standorte die Teilstücke der Routen bezeichnet aus denen die Route zusammengesetzt ist.
\newline
\newline
Mian entschied sich die aufgenommen Sensordaten vorzuverarbeiten.
Zunächst werden die Sensoren für fünf Stichproben über den Median geglättet.
Aus den resultierenden Sensorwerten wird die Veränderung zum vorherigen Sensorwert für jeden Sensor ermittelt.
Für jeden Sensor wird als Feature der Betrag dieser Differenz verwendet.
Um Muster aus einer Folge von Feature-Mengen zu inferieren hat Mian ein Datenfenster eingeführt, über das
hintereinander liegende Feature-Mengen zu einer Feature-Menge konkatiniert werden.
Zudem werden die zuletzt besuchten Standorte in Form einer exponentiell fallenenden Funktion über die Zeit als weitere Features hinzugefügt.
\newline
\newline
Mit dieser Eingabe trainierte Mian ein FFNN mit einer Rückwärtskante von der Ausgabe- zur Eingabeschicht,
um die zuletzt bestimmten Standorte als Features nutzen zu können.
Die Rückwärtskante wurde im Training simuliert, indem die Trainingsdaten in zwei Teilmengen partitioniert wurden.
Mit der ersten Teilmenge wurde das FFNN mit korrekt beschrifteten Trainingsdaten trainiert.
Das trainierte FFNN wurde dann genutzt, um die Standorte der zweiten Teilmenge zu bestimmen.
Daraufhin wurde das FFNN mit der zweiten Teilmenge trainiert, bevor es auf einer Testmenge validiert wurde.
\newline
\newline
Mian validierte sein Modell auf einer Testmenge die aus sechs distinkten Standorten besteht über fünf Zyklen.
Mit einem Datenfenster ab 25 konnten Klassifizierungsgenauigkeiten von bis zu 96\% erreicht werden.