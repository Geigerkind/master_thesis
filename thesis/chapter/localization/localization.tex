\chapter{Standortbestimmung}
Als Standortbestimmung, oder \textit{Lokalisierung}, wird der Prozess bezeichnet die Position von einem Gerät oder Nutzer in einem Koordinatensystem zu bestimmen \cite{bulusu2000gps}.
Unterschieden wird dabei zwischen \textit{Indoor}- und \textit{Outdoor}-Lokalisierung \cite{zafari2019survey, bulusu2000gps}.
Bei Indoor-Lokalisierung wird ein Szenario innerhalb von Gebäuden betrachtet und bei Outdoor-Lokalisierung ein Szenario unter dem freien Himmel.
\newline
\newline
Weiterhin wird unterscheiden zwischen \textit{Device-Based}- und \textit{Device-Free}-Lokalisierung \cite{xiao2016survey}.
Bei Device-Based-Lokalisierung bestimmt das Gerät selbst die Position, wohingegen bei der Device-Free-Lokalisierung
die Position von der Infrastruktur bestimmt wird.
\newline
\newline
Lokalisierung ist aber nicht nur beschränkt für Geräte.
Menschen und Tiere haben einen Orientierungssinn, der die Navigation anhand von Orientierungspunkten ermöglicht \cite{menzel1996knowledge}.
Beispielsweise die Navigation von Honigbienen hängt stark von den Orientierungspunkten ab.
Anstatt die direkte Route zu wählen, fliegen Honigbienen die Orientierungspunkte ab, um zu ihrem Ziel zu gelangen.
Dabei ist die Navigation robust gegenüber leichte Veränderungen der Orientierungspunkte.

\section{Indoor- und Outdoor-Lokalisierung}
GPS ist eine weit verbreites Standortbestimmungssystem im Outdoor-Kontext  und kann für eine Vielzahl von Anwendungen eingesetzt werden,
z. B. Tracking, Navigation oder Rettungsaktionen \cite{kaplan2005understanding}.
Es skaliert zu einer arbiträren Anzahl von Nutzern, da es einen device-based Ansatz verwendet.
Die Geräte berechnen aus den empfangenden Signalen von mehreren Satelliten ihre Position.
Dabei ist die Standortbestimmung bis zu 5 m genau \cite{sadowski2018rssi},
wobei es Varianten gibt, die noch bessere Auflösungen erzielen können \cite{parkinson1996differential}.
Der Energieverbrauch ist sehr hoch \cite{jurdak2013energy}, wodurch es ungeeignet für kleine batteriegestützte Systeme ist.
In einem Indoor-Kontext kann GPS aber meistens nicht eingesetzt werden,
da das Gebäude die benötigte Signalstärke zu den Satteliten beeinträchtigen kann \cite{xiao2016survey, jin2006indoor}.
Aus diesem Grund ist es für batteriegestützte Mikrocontroller im Indoor-Bereich suboptimal.
\newline
\newline
Indoor-Lokalisierung bedarf meist einer hohen Auflösung, muss sicherheitskritische Vorgaben einhalten,
energieeffizient sein, skalierbar sein und geringe Kosten haben \cite{xiao2016survey}.
Es gibt verschiedene Ansätze, die entweder device-based oder device-free sind.
\newline
\newline
Device-based Ansätze nutzen die Sensoren des Geräts, um die Position zu bestimmen \cite{xiao2016survey}.
Beispielsweise können visuelle Features mit der Kamera extrahiert werden \cite{poulose2019hybrid, cunha2011using},
RSS Messungen des WiFi Netzwerkes durchgeführt werden \cite{pan2008transfer},
Bewegungs- oder Lichtsensoren verwendet werden \cite{poulose2019hybrid, wang2018deepml, xiao2016survey}.
Der Vorteil sind die geringen Infrastrukturkosten \cite{xiao2016survey}.
\newline
\newline
Device-free Ansätze bedürfen einer Infrastruktur, um Objekte im Interessebereich wahrzunehmen \cite{xiao2016survey}.
Diese werden zum Beispiel für Überwachsungsszenarien eingesetzt \cite{qian2018widar2}.
Ansätze nutzen beispielsweise die bestehende Kamerainfrastruktur aus \cite{kim2019info},
WiFi basierte Ansätze \cite{qian2018widar2}, Infrarot basierte Ansätze \cite{kemper2010passive}
oder RFID basierte Ansätze \cite{yang2015see}.
\section{Standortbestimmung mit maschinellen Lernen}
TODO

\subsection{Sensorbasierter Orientierungssinn mit FFNN}
Dieser Arbeit ging die Arbeit von Mian voran, der sich zum gleichen Thema mit FFNN auseinander gesetzt hat \cite{naveedThesis}.
Mian nutzte den Simulator CoppeliaSim, um Daten von verschieden komplexen Routen zu generieren.
Die Routen unterschieden sich dabei in der Anzahl verschiedener Orte und Pfade die für einen Zyklus verwendet werden können.
Die aufgenommenen Daten enthalten Sensorwerte für Beschleunigung, Gyroskop, Licht und Beschriftungen für die Standorte.
Dabei werden als Standorte die Teilstücke der Routen bezeichnet aus denen die Route zusammengesetzt ist.
\newline
\newline
Mian entschied sich die aufgenommen Sensordaten vorzuverarbeiten, aus denen er folgende Features extrahierte:
\begin{itemize}
    \item Median von allen Sensoren
    \item Veränderung der Sensordaten zum vorherigen Wert
    \item Wert der Sensorwertvektoren des Beschleunigungssensors und Gyroskops
    \item Zuletzt besuchte Standorte, die über Zeit vergessen werden
\end{itemize}
Um Muster aus einer Folge von Feature-Mengen zu inferieren hat Mian ein Datenfenster eingeführt, über das
hintereinander liegende Feature-Mengen zu einer Feature-Menge konkatiniert werden.
\newline
\newline
Mit dieser Eingabe trainierte Mian ein FFNN mit einer Rückwärtskante, um den zuletzt bestimmten Standort als Feature nutzen zu können.
Die Rückwärtskante wurde im Training simuliert, indem die Trainingsdaten in zwei Teilmengen partitioniert wurden.
Mit der ersten Teilmenge wurde das FFNN mit korrekt beschrifteten Trainingsdaten trainiert.
Das trainierte FFNN wurde dann genutzt, um die Standorte der zweiten Teilmenge zu bestimmen.
Daraufhin wurde das FFNN mit der zweiten Teilmenge trainiert, bevor es auf einer Testmenge validiert wurde.
\newline
\newline
Mian validierte sein Modell auf einer Testmenge die aus 6 distinkten Standorten besteht über fünf Zyklen.
Mit einem Datenfenster ab 25 konnten Klassifizierungsgenauigkeiten von bis zu 96\% erreicht werden.




\iffalse
\begin{itemize}
    \item Vision Based
\end{itemize}
\fi
