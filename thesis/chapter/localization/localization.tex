\chapter{Standortbestimmung}
Als Standortbestimmung, oder \textit{Lokalisierung}, wird der Prozess bezeichnet die Position von einem Gerät oder Nutzer in einem Koordinatensystem zu bestimmen \cite{bulusu2000gps}.
Unterschieden wird dabei zwischen \textit{Indoor}- und \textit{Outdoor}-Lokalisierung \cite{zafari2019survey, bulusu2000gps}.
Bei Indoor-Lokalisierung wird ein Szenario innerhalb von Gebäuden betrachtet und bei Outdoor-Lokalisierung ein Szenario unter dem freien Himmel.
\newline
\newline
Weiterhin wird unterscheiden zwischen \textit{Device-Based}- und \textit{Device-Free}-Lokalisierung \cite{xiao2016survey}.
Bei Device-Based-Lokalisierung bestimmt das Gerät selbst die Position, wohingegen bei der Device-Free-Lokalisierung
die Position von der Infrastruktur bestimmt wird.
\newline
\newline
Lokalisierung ist aber nicht nur beschränkt für Geräte.
Menschen und Tiere haben einen Orientierungssinn, der die Navigation anhand von Orientierungspunkten ermöglicht \cite{menzel1996knowledge}.
Beispielsweise die Navigation von Honigbienen hängt stark von den Orientierungspunkten ab.
Anstatt die direkte Route zu wählen, fliegen Honigbienen die Orientierungspunkte ab, um zu ihrem Ziel zu gelangen.
Dabei ist die Navigation robust gegenüber leichte Veränderungen der Orientierungspunkte.

\section{Indoor- und Outdoor-Lokalisierung}
GPS ist eine weit verbreites Standortbestimmungssystem im Outdoor-Kontext  und kann für eine Vielzahl von Anwendungen eingesetzt werden,
z. B. Tracking, Navigation oder Rettungsaktionen \cite{kaplan2005understanding}.
Es skaliert zu einer arbiträren Anzahl von Nutzern, da es einen device-based Ansatz verwendet.
Die Geräte berechnen aus den empfangenden Signalen von mehreren Satelliten ihre Position.
Dabei ist die Standortbestimmung bis zu 5 m genau \cite{sadowski2018rssi},
wobei es Varianten gibt, die noch bessere Auflösungen erzielen können \cite{parkinson1996differential}.
Der Energieverbrauch ist sehr hoch \cite{jurdak2013energy}, wodurch es ungeeignet für kleine batteriegestützte Systeme ist.
In einem Indoor-Kontext kann GPS aber meistens nicht eingesetzt werden,
da das Gebäude die benötigte Signalstärke zu den Satteliten beeinträchtigen kann \cite{xiao2016survey, jin2006indoor}.
Aus diesem Grund ist es für batteriegestützte Mikrocontroller im Indoor-Bereich suboptimal.
\newline
\newline
Indoor-Lokalisierung bedarf meist einer hohen Auflösung, muss sicherheitskritische Vorgaben einhalten,
energieeffizient sein, skalierbar sein und geringe Kosten haben \cite{xiao2016survey}.
Es gibt verschiedene Ansätze, die entweder device-based oder device-free sind.
\newline
\newline
Device-based Ansätze nutzen die Sensoren des Geräts, um die Position zu bestimmen \cite{xiao2016survey}.
Beispielsweise können visuelle Features mit der Kamera extrahiert werden \cite{poulose2019hybrid, cunha2011using},
RSS Messungen des WiFi Netzwerkes durchgeführt werden \cite{pan2008transfer},
Bewegungs- oder Lichtsensoren verwendet werden \cite{poulose2019hybrid, wang2018deepml, xiao2016survey}.
Der Vorteil sind die geringen Infrastrukturkosten \cite{xiao2016survey}.
\newline
\newline
Device-free Ansätze bedürfen einer Infrastruktur, um Objekte im Interessebereich wahrzunehmen \cite{xiao2016survey}.
Diese werden zum Beispiel für Überwachsungsszenarien eingesetzt \cite{qian2018widar2}.
Ansätze nutzen beispielsweise die bestehende Kamerainfrastruktur aus \cite{kim2019info},
WiFi basierte Ansätze \cite{qian2018widar2}, Infrarot basierte Ansätze \cite{kemper2010passive}
oder RFID basierte Ansätze \cite{yang2015see}.
\newpage
\section{WiFi basierte Indoor-Lokalisierung mit Transfer Lernen}
Pan et al. untersuchten Indoor-Lokaliserung basierend auf WiFi RSS Daten \cite{pan2008transfer}.
Der Empfänger misst die \textit{Received Signal Strength} (RSS) mehrerer Sender.
Dadurch steht ein Vektor von Signalstärken zur Verfügung aus dem die Position approximiert werden kann.
\newline
\newline
Pan et al. stellen fest, dass bei ML Ansätzen oft zwei Annahmen getroffen werden.
Zum einen wird eine \textit{Offline}-Phase vorausgesetzt mit ausreichend beschrifteten Daten,
d. h. eine Trainingsphase bevor das Modell eingesetzt wird.
Zum anderen wird angenommen, dass das gelernte Modell statisch über Zeit, Raum und Geräte ist.
\newline
\newline
In der Praxis können sich Modelle jedoch über die Zeit ändern, z.~B. wenn Mitarbeiter Mittags zur Kantine gehen.
Es kann schwierig sein ausreichend Daten für große Gebäude zu sammeln.
Verschiedene Geräte können verschiedene Sensorwerte erfassen.
Dies führt zu einen erheblichen Kalibrierungsaufwand der ML Modelle.
\newline
\newline
Pan et al. schlagen Transfer Learning vor, um dieses Problem zu lösen.
Transfer Learning befasst sich mit dem Problem, wenn Trainings- und Testdaten verschiedener Verteilungen folgen oder in verschiedenen Feature-Räumen repräsentiert sind.
Die gewonnene Erfahrung beim Training soll dabei auf das neue, ähnliche Problem übertragen werden.
Dafür müssen Beziehungen gefunden werden, die den Erfahrungstransfer ermöglichen.
\newline
\newline
Sie trainierten ein \textit{Hidden Markov Model} (HMM), wobei die Ortserkennung als Klassifizierungsproblem diskreter Orte modelliert wurde.
Auf ihren Testdaten waren sie signifikant besser als Ansätze ohne Transfer Learning.

\section{Sensorbasierter Orientierungssinn mit FFNN}
Dieser Arbeit ging die Arbeit von Mian voran, der sich zum gleichen Thema mit FFNN auseinander gesetzt hat \cite{naveedThesis}.
Mian nutzte den Simulator CoppeliaSim, um Daten von verschieden komplexen Routen zu generieren.
Die Routen unterschieden sich dabei in der Anzahl verschiedener Orte und Pfade die für einen Zyklus einer Route verwendet werden können.
Die aufgenommenen Daten enthalten Sensorwerte für Beschleunigung, Gyroskop, Licht und Beschriftungen für die Standorte.
Dabei werden als Standorte die Teilstücke der Routen bezeichnet aus denen die Route zusammengesetzt ist.
\newline
\newline
Mian entschied sich die aufgenommen Sensordaten vorzuverarbeiten.
Zunächst werden die Sensoren für fünf Stichproben über den Median geglättet.
Aus den resultierenden Sensorwerten wird die Veränderung zum vorherigen Sensorwert für jeden Sensor ermittelt.
Für jeden Sensor wird als Feature der Betrag dieser Differenz verwendet.
Um Muster aus einer Folge von Feature-Mengen zu inferieren hat Mian ein Datenfenster eingeführt, über das
hintereinander liegende Feature-Mengen zu einer Feature-Menge konkatiniert werden.
Zudem werden die zuletzt besuchten Standorte in Form einer exponentiell fallenden Funktion über die Zeit als weitere Features hinzugefügt.
\newpage
Mit dieser Eingabe trainierte Mian ein FFNN mit einer Rückwärtskante (FBNN) von der Ausgabe- zur Eingabeschicht,
um die zuletzt bestimmten Standorte als Features nutzen zu können.
Die Rückwärtskante wurde im Training simuliert, indem die Trainingsdaten in zwei Teilmengen partitioniert wurden.
Mit der ersten Teilmenge wurde das FFNN mit korrekt beschrifteten Trainingsdaten trainiert.
Das trainierte FFNN wurde dann genutzt, um die Standorte der zweiten Teilmenge zu bestimmen.
Daraufhin wurde das FFNN mit der zweiten Teilmenge trainiert, bevor es auf einer Testmenge validiert wurde.
\newline
\newline
Mian unterscheidet vier Modellarchitekturen: FFNN, FBNN, WFFNN (FFNN mit Sensorengedächtnis) und WFBNN (FBNN mit Sensorengedächtnis).
Er stellte fest, dass das FFNN nicht in der Lage war verschiedene Standorte zu unterscheiden,
unabhängig von der Anzahl der verdeckten Schichten und dessen Anzahl von Neuronen.
\newline
\newline
Das FBNN hingegen konnte bei einer Route mit einem Pfad und sechs Standorten Testgenauigkeiten von bis zu 98.12\% erzielen.
Allerdings bedarf es dafür zwei verdeckte Schichten mit jeweils 64 Neuronen.
Mit weniger Schichten oder Neuronen wurden deutlich schlechtere Ergebnisse erzielt.
Bei mehr Pfaden und Standorten wurden ebenfalls schlechtere Ergebnisse erzielt, obwohl Anzahl der Neuronen pro verdeckte Schicht auf 256 erhöht wurde.
Bei einer Route mit zwei Pfaden und neun Standorten wurden Testgenauigkeiten von 85.56\% erzielt und
bei drei Pfaden und 14 Standorten wurden Testgenauigkeiten von 33,57\% erzielt.
\newline
\newline
Mit der Einführung eines Datenfensters (WFFNN und WFBNN) hat sich die Klassifizierungsgenauigkeit signifikant erhöht.
Ein WFFNN mit zwei verdeckten Schichten mit jeweils 64 Neuronen und einem Sensorengedächtnis von 50 konnte eine Testgenauigkeit von 99,13\%
bei einem Pfad und sechs Standorten erreichen.
Die Testgenauigkeiten bei zwei Pfaden und 9 Standorten war 94,41\%.
Bei drei Pfaden und 14 Standorten wurde mit einer verdeckten Schicht mit 56 Neuronen und einem Sensorengedächtnis von 200 mit dem WFFNN 94,51\% erzielt
und mit dem WFBNN 93,26\%.
\newline
\newline
Mian stellte fest, dass sich die Klassifizierungsgenauigkeiten besser sind,
wenn eine Lichtquelle an Standorten gesetzt wird, an denen sich die neuronalen Netze unsicher sind.
Allerdings würden Fehler durch den Sensor oder Veränderungen der Lichtverhältnisse
einen größeren Einfluss auf die Klassifizierungsgenauigkeit des Modells haben.
\newline
\newline
Mian konkludierte, dass ein Kompromiss zwischen Klassifizierungsgenauigkeit und Modellgröße geschlossen werden müsste,
da die Modellgröße und Klassifizierungsgenauigkeit proportional mit der Anzahl der verdeckten Schichten und Neuronen,
sowie der Datenfenstergröße zusammenhinge.

