\section{Ressourcenbedarf auf dem Mikrocontroller}
\label{sec:dt_resource_usage}
Zukünftig soll das Modell auf einem Mikrocontroller ausgeführt werden \cite{antragForschungsprojekt}.
Mikrocontroller sind stark limitiert in ihrer Rechenleistung, Speicherkapazität, RAM und werden oft zudem mit einer Batterie betrieben.
Aus diesem Grund ist der Energieverbrauch zu minimieren und das Modell muss innerhalb dieser Limitierungen operieren können.

\newpage
\subsection{Ausführungszeit und Energieverbrauch}
\label{sub_sec:dt_ru_execution_time}
Der Energieverbrauch korreliert mit der Ausführungszeit.
Je länger die CPU ausgeschaltet ist, desto weniger Energie wird verbraucht.
Kurze Ausführungszeiträume vergrößern den Zeitraum, in dem die CPU ausgeschaltet sein kann.
Die Ausführungszeit ist die Zeit die benötigt wird, um alle Instruktionen auszuführen \cite{dymelThesis}.
Jede Instruktion bedarf eine bestimmte Anzahl an CPU-Zyklen.
Die Zeit pro Zyklus ist abhängig von der Taktrate der CPU.
\newline
\newline
Die Ausführungszeit eines Entscheidungswaldes setzt sich zusammen aus der Zeit für die Feature-Extrahierung, der Evaluierung aller im Ensemble enthaltenen Entscheidungsbäume
und der Aggregierungsfunktion.
Im schlimmsten Fall muss die gesamte Höhe eines Entscheidungsbaumes traversiert werden, um das Ergebnis zu bestimmen. Aus diesem Grund skaliert die Ausführungszeit mit der
traversierten Höhe jedes Baumes.
\newline
\newline
Um die Instruktionen zu minimieren sollten Datentypen verwendet werden, die von der CPU mit höchstens einem Wort dargestellt werden können.
Eine 8-Bit CPU würde zum Laden in Register eines 32-Bit Datentypen vier mal so viele Instruktionen benötigen wie bei einem 8-Bit Datentypen.
Außerdem sollten Operationen verwendet werden, die durch native Hardware-Operationen abgebildet werden können.
Ist dem nicht so, muss diese Operation durch Software ersetzt werden.
Dies erfordert mehr Zyklen als eine native Operation in Hardware.
\newline
\newline
Zu Beachten bei der Minimierung ist, dass Instruktionen unterschiedlich viele Zyklen benötigen und Funktionsaufrufe Overhead erzeugen.
Ein Beispiel dafür ist die Optimierung \textit{Function Inlining} \cite{leupers1999function}.
Der Aufruf von Funktionen kann einen hohen Overhead durch den Kontextwechsel erzeugen.
Aus diesem Grund verringert diese Optimierung die Ausführungszeit, erhöht aber die die Programmgröße signifikant.
Im Umkehrschluss könnten durch die Verwendung von Funktionen der nutzen des Programmspeichers verringert werden, Ausführungszeit und Energieverbrauch aber erhöht werden.

\newpage
\subsection{Programmgröße und RAM}
\label{sub_sec:dt_ru_programm_size}
Die Programmgröße ist die Gesamtheit aller Instruktionen die für das Programm benötigt werden \cite{dymelThesis}.
Dabei ist der Anteil für die Entscheidungswälder integral und der Anteil für die perifären Funktionalitäten zu vernachlässigen.
Die Programmgröße, die für einen Entscheidungswald benötigt wird, skaliert mit der Waldgröße und Höhe der einzelnen Entscheidungsbäume.
\newline
\newline
Die Höhe des Entscheidungsbaumes ist die Verzweigungstiefe der verschachtelten Tests.
Jeder Test ist ein Vergleich mit einem Schwellenwert.
Die Programmgröße für einen Vergleich setzt sich zusammen aus den Operationen, um die Operanden in die Register zu laden,
und die Instruktion, um den Vergleich durchzuführen, sowie Abzweiginstruktionen. Wie in Kapitel \ref{sub_sec:dt_ru_execution_time}
sind Instruktionen durch einen passenden Datentypen zu vermeiden.
\newline
\newline
Ein weiterer Faktor sind die Instruktionen, die zur Rückgabe des Klassifizierungsergebnis benötigt werden.
In Kapitel \ref{sec:dt_ensemble_methods} wurden verschiedene Möglichkeiten der Rückgabe diskutiert, die relevant bei dem Aggregierungsprozess eines Ensembles ist.
Einerseits kann die Rückgabe eine Wahrscheinlichkeitsverteilung sein und andererseits eine diskrete Klasse.
Bei $m$ möglichen Klassen würde die erste Variante $m$-mal so viele Instruktion benötigen, wie die zweite Variante, da der Rückgabevektor zuvor mit der Wahrscheinlichkeitsverteilung gefüllt werden muss.
In der Praxis werden aber weniger Instruktion benötigt, da es eine große Überschneidung der Wahrscheinlichkeitsverteilungen gibt, die zurück gegeben werden.
Die Instruktionen, um den Rückgabevektor zu befüllen, können durch \textit{Basic Blocks}, d. h. beschriftete Instruktionsblöcke, geschickt recycled werden.
Zudem können Zuweisungen ausgelassen werden, die die Wahrscheinlichkeit 0 zuweisen, da der Vektor mit Nullen initialisiert wird.
Dennoch werden signifikant mehr Instruktionen benötigt als bei der diskreten Variante.
Aus diesem Grund wurde ein hybrider Ansatz vorgeschlagen, der im Falle eines eindeutigen Ergebnisses mit einer Toleranz von $\epsilon\in [0, 1]$ die diskrete Klasse statt der Wahrscheinlichkeitsverteilung zurück gibt.
\newline
\newline
Der bentötigte RAM ist abhängig von der Größe der Feature-Menge und Anzahl zu klassifizierenden Klassen.
Für die Rückgabe wird zwischen ein Byte und $mQ$ Byte benötigt, wobei Q die größe des verwendeten Datentypen ist, um eine Gleitkommazahl zu speichern.