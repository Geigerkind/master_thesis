\section{Scikit-Learn}
\label{sec:dt_scikit_learn}
Diese Arbeit verwendet die Python ML-Bibliothek \textit{Scikit-Learn}. Scikit-Learn bietet verschiedene ML Algorithmen mit einem High-Level Interface an \cite{scikit-learn}.
Zur Konstruktion der Entscheidungsbäume wird der Algorithmus von CART verwendet \cite{ScikitLearnCART}.
Die Bibliothek kann Klassifizierer und Regressoren generieren.
\newline
\newline
Nachfolgend wird nur von Klassifizierern ausgegangen, da nur diese für diese Arbeit benötigt werden. Diese bieten zahlreiche Hyperparameter an, um die Konstruktion des
Entscheidungsbaumes zu steuern. In dieser Arbeit wird der Hyperparameter \texttt{max\_depth} verwendet, der die maximale Baumhöhe beschränkt und somit
den Programmspeicherverbrauch begrenzt.
\newpage
Scikit-Learn bietet Ensemble-Methoden an, um Entscheidungswälder zu trainieren. Ein wichtiger Parameter diese Methoden ist \texttt{n\_estimators}. Er steuert die Größe des Ensembles bzw.
die Waldgröße. Somit hat auch dieser Parameter Einfluss auf den Programmspeicherverbrauch.