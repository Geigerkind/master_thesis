\chapter{Künstliche Neuronale Netze}
Das mathematische Modell von künstlichen neuronalen Netzen wurde von McCulloch und Pitts im Jahre 1943 erfunden \cite{mcculloch1943logical}.
Dieses Modell ist eine Abstraktion des biologischen Neuronen als logischer Mechanismus.
Man unterscheided beim biologischen Neuronen zwischen \textit{Afferent}-Neuronen, \textit{Efferent}-Neuronen und \textit{Inter}-Neuronen (TODO: Quelle).
Afferent-Neuronen nehmen elektrische Signale von Organen entgegen und können als \textit{Input} interpretiert werden.
Efferent-Neuronen geben elektrische Signale an \textit{Effektorzellen} weiter und können als \textit{Output} interpretiert werden.
Inter-Neuronen nehmen elektrische Signale von Afferent-Neuronen oder Inter-Neuronen entgegen und geben sie an Inter-Neuronen oder Efferent-Neuronen weiter.
Wenn der Schwellenwert eines \textit{Dendrite} von einem Neuronen durch ein elektrisches Signal erreicht wurde, wird ein elektrisches Signal über den \textit{Axon} an ein anderes Neuron oder Effektorzellen übertragen.
Diese Charakteristiken werden mathematisch als ein Vergleich von einer gewichtete Summe von eingehenden Signalen mit einem Schwellenwert modelliert (TODO: Quelle).
Gleichung \ref{formular:neuron_activation} stellt diesen Zusammenhang dar.
\begin{align}
    \label{formular:neuron_activation}
    y = a(\textbf{w}^\intercal\textbf{x} + b)
\end{align}
Die Vergleichsoperation ist die \textit{Aktivierungsfunktion} $a: \mathbb{R}\mapsto\mathbb{R}$, die in diesem Fall die Eingabe mit 0 vergleicht und eine binäre Rückgabe abbildet (TODO: Quelle).
Die Eingabe $\textbf{x}\in\mathbb{R}^n$ wird mit $\textbf{w}\in[0, 1]$ gewichtet und der \textit{Bias} $b\in\mathbb{R}$ wird addiert. Der Bias stellt den Schwellenwert dar.
\newline
\newline
Das künstliche neuronale Netz approximiert eine arbiträre Funktion $f^*$. Dazu findet es eine Menge von Parametern $\boldsymbol\theta$, wodurch $f^*(\textbf{x})\approx f(\textbf{x}, \boldsymbol\theta)$
möglichst gut von der Approximationsfunktion $f$ abgebildet wird.
\newline
\newline
Das KNN ist in Schichten organisiert. Analog zu den biologischen Neuron, gibt es eine \textit{Eingabeschicht (engl. Input-Layer)},
\textit{Ausgabeschicht (engl. Output-Layer)} und \textit{verdeckte Schicht (engl. Hidden-Layer)} (TODO: Quelle).
Analog zur Aktivierung eines einzelnen Neuronen, dargestellt in Gleichung \ref{formular:neuron_activation}, stellt Gleichung \ref{formular:layer_activation} die Aktivierung einer Schicht dar.
\begin{align}
    \label{formular:layer_activation}
    \textbf{a}_{i+1} = a_i(\textbf{z}_i), \hspace{2cm} \textbf{z}_i := \textbf{W}_i\textbf{a}_i + \textbf{b}_i
\end{align}
Das allgemeine KNN verfügt über $m\in\mathbb{N}$ Schichten. Jede Schicht $i$ verfügt über $n_i\in\mathbb{N}$ Neuronen.
Die Aktivierungsfunktion $a_i:\mathbb{R}^{n_{i+1}}\mapsto\mathbb{R}^{n_{i+1}}$ berechnet die Aktivierung mit der gewichteten Summe $\textbf{z}_i$.
Die gewichtete Summe setzt sich zusammen aus der Aktivierung der vorherigen Schicht $\textbf{a}_i$ die mit $W_i\in\mathbb{R}^{n_{i+1}x{n_{i}}}$ gewichtet wird.
Die Schwellenwerte jedes Neuronen werden durch die Biase $\textbf{b}_i\in\mathbb{R}^{n_{i+1}}$ dargestellt.
Gleichung \ref{formular:general_knn} zeigt, wie das allgemeine KNN mit einer rekursiven Funktion modelliert werden kann.
\begin{align}
    \label{formular:general_knn}
    \textbf{a}_1 := \textbf{x}, \hspace{1cm} \textbf{z}_i := \textbf{W}_i\textbf{a}_i + \textbf{b}_i, \hspace{1cm} \textbf{a}_{i+1} := a_i(\textbf{z}_i), \hspace{1cm} \textbf{f(x)} := \textbf{a}_m
\end{align}
Diese Arbeit nutzt ausschließelich \textit{Feed Forward neuronale Netzwerke}.
Diese werden durch \textit{dichte Schichten (engl. Dense-Layer)} charakterisiert, d. h. Schichten in denen alle Neuronen einer Schicht mit allen Neuronen der folgenden Schicht verbunden sind.

\section{Keras}
TODO

\subsection{Aktivierungsfunktionen}
TODO

\subsection{Verlustfunktionen}
TODO

\subsection{Optimierer}
TODO

\subsection{Regularisierung}
TODO

\section{Training von Neuronalen Netzen}
TODO

\section{Ressourcenbedarf auf dem Mirkocontroller}
\begin{itemize}
    \item Energieverbrauch
    \item Speicherverbrauch
    \item RAM verbrauch
    \item Siehe Kubik und Co.
\end{itemize}