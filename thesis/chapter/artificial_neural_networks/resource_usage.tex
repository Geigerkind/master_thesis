\section{Ressourcenbedarf auf dem Mikrocontroller}
Der Speicherverbrauch eines neuronalen Netzes ist abhängig von der Anzahl der Gewichte und Biase, sowie der Größe des verwendeten Datentypen \cite{kubikThesis}.
Ein FFNN mit 3 Schichten der Größe $n_1, n_2, n_3$ hat $n_1 n_2 + n_2 n_3$ Gewichte und $n_2 + n_3$ Biase.
Die Gewichte und Biase sind für gewöhnlich Gleitkommazahlen. Diese können mit 4 und 8 Byte dargestellt werden.
Als Alternative können auch Festkommazahlen verwendet werden, die 2 Byte benötigen \cite{gieseThesis}.
\newline
\newline
Die Ausführungszeit des KNN ist stark abhängig von der Hardware.
Das KNN kann sowohl im RAM abgelegt werden oder im Flash-Speicher \cite{engelhardtThesis}.
Wenn es im Flash-Speicher abgelegt werden muss, müssen die Gewichte bei der Ausführung in den RAM und die Register geladen werden.
In Yao's Experimenten hat dies die Ausführungszeit um bis zu 74\% verlangsamt \cite{yaoThesis}.
\newline
\newline
Außerdem wird Multiplikation und Division zur Evaluierung des KNNs benötigt \cite{engelhardtThesis}.
Wenn die Hardware keine Unterstützung dafür anbietet, müssen diese Operationen durch Software ergänzt werden.
Diese sind dann signifikant komplexer zu berechnen im Vergleich zu hardwareuntersützten Operationen.
\newline
\newline
Die Ausführungszeit ist auch abhängig von der Struktur des Netzwerkes \cite{gieseThesis}.
Je mehr Gewichte ein KNN hat, desto mehr Multiplikationen müssen durchgeführt werden.
Zudem wird in jeder Schicht eine Aktivierungsfunktion angewendet, die ebenfalls sehr aufwendige Berechnungen benötigen kann \cite{venzkeArticle}.
\newline
\newline
Es gibt verschiedene Optimierungen, die die Ausführungszeit und den Speicherbedarf verringern.
Giese hat in seiner Arbeit festgestellt, dass das Löschen von Gewichten (engl. pruning) oder das Gruppieren von ähnlichen Gewichten (engl. quantization)
signifikante Verbesserungen sowohl des Speicherverbrauchs als auch bei der Ausführungszeit zeigt \cite{gieseThesis}.
Denkbar sind auch Compiler-Optimierungen, die sich sowohl auf Ausführungszeit, Speicherbedarf und Energiebedarf auswirken können.
Giese stellte fest, dass sowohl der Speicherbedarf verringert werden kann, als auch die Ausführungszeit.