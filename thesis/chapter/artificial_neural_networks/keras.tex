\section{Keras}
Keras ist die am meisten genutzte \textit{deep learning} API und wurde in Python geschrieben \cite{kerasDoc}.
Dadurch ist sie kompatibel mit allen gängigen Betriebssystemen.
Ihr Fokus ist eine intuitive und simple API anzubieten, sodass schnelle Iterationen im Entwicklungsprozess möglich sind.
Trotzdem ist sie effizient und skalierbar, um die Kapazitäten großer Rechenverbunde auszunutzen.
\newline
\newline
Keras abstrahiert das ML System \textit{Tensorflow}. Tensorflow implementiert ML Algorithmen, die dem Stand der Forschung entsprechen.
Der Fokus ist auf effizientes Training der Modelle \cite{abadi2016tensorflow} gerichtet.
Dafür nutzt es die Multikernarchitektur von CPUs, GPUs und spezialisierter Hardware, sogenannten TPUs (\textbf{T}ensor \textbf{P}rocessing \textbf{U}nit), aus.
Es wurde als open-source Projekt veröffentlicht und ist weit verbreitet.
\newline
\newline
Keras bietet die in dieser Arbeit benötigten Algorithmen an, weshalb es zum Trainieren von FFNNs verwendet wird.